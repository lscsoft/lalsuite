\documentclass[oneside]{book}
\usepackage{../lal,fancyhdr,epsfig,psfig,color,hyperref,makeidx}
\includeonly{std,support,hello,factories,tdfilters}

%define page size
\setlength{\textheight}{9.0in}
\setlength{\textwidth}{6.0in}
\setlength{\topmargin}{-0.00in}
\setlength{\oddsidemargin}{-0.25in}
\setlength{\evensidemargin}{\oddsidemargin}
\sloppy

\pagestyle{fancy}
\fancyhf{}
\lhead{\bf\nouppercase\rightmark}
\rhead{ \bf Pg \thepage}

\makeindex

\newfont{\lsdfont}{cmbx10 at 72pt}

\begin{document}

% \reversemarginpar
% \let\marginpar\mparorig
% \providecommand{\marginpar}[1]{\mbox{}\mparorig{\raggedleft\hspace{0pt}#1}}

% The title page:
\title{\sffamily\bfseries\Huge
\textcolor{red}{\lsdfont L}AL
\raisebox{-2.5ex}{\textcolor{green}{\lsdfont S}\hspace{-0.1em}oftware}
\hspace{-2em}
\raisebox{-0.5ex}{\textcolor{blue}{\lsdfont D}\hspace{-0.2em}ocumentation}}
\author{\bf Members of the LSC}
\maketitle



% The table of contents.
\tableofcontents

\chapter*{Preface}
\textbf{The LAL specification and style guide has precedence over any
statements about programming practice stated in this document, period.}

\part{Coding and Documentation Instructions}

\chapter{Instructions for code documentation}
\section{How to import code and comments from the source code to 
your \LaTeX\ documentation}

In addition to a knowledge of \LaTeX, there are only four additional
instructions you need to know to extract a code fragment, a comment block, or
a piece of \LaTeX\ source from your source-code files and insert it into your
documentation file.  They are:
\begin{verbatim}
 <lalVerbatim file="FileName"> 
 </lalVerbatim> 

 <lalLaTeX file="FileName"> 
 </lalLaTeX > 
\end{verbatim}
Use these commands (and the standard \LaTeX\ command \verb@\input{}@), changes
in the in-line documentation of the source-code will be captured---exactly as
they appear in the source---in the comprehensive manual.  This allows the
programmers to simultaneously serve several difficult masters:
\begin{description}
\item[$\bullet$ ] 
Produce a comprehensive, coherent manual of the entire software package.
\vspace*{-0.05in}
\item[$\bullet$ ] 
Keep the primary source of the documentation close to the code.
\item[$\bullet$] 
Insure that changes in the code and comments are immediately and automatically
captured in the comprehensive manual.
\item[$\bullet$ ] 
Use \LaTeX\ which is suitable for equation writing.
\vspace*{-0.051in}
\end{description}


Using the first pair of instructions, everything between
\texttt{<lalVerbatim file="FileName">} and \texttt{</lalVerbatim>} will be
wrapped in a \LaTeX\ verbatim environment and written to the file
\texttt{FileName.tex} for later insertion in the documentation.  The insertion
can be done with the \LaTeX\ command \verb@\input{FileName}@.

As an example, look at the code fragment in the source file 
\texttt{packages/hello/src/LALHello.c}.
\begin{verbatim}
#include <stdio.h>
#include "LALStdlib.h"
#include "LALHello.h"

/* <lalVerbatim file="LALHelloNRCSID"> */
NRCSID( LALHELLOC, "$Id$" );
/* </lalVerbatim> */
\end{verbatim}
When you run \texttt{make dvi} (this will be automated) in the package
directory, a parser sifts the code file. It wraps what is in between
\texttt{<lalVerbatim file="LALHelloNRCSID"> } and the \texttt{</lalVerbatim>}
in the \LaTeX\ "verbatim environment", and writes the result to the file
\texttt{LALHelloNRCSID.tex }.  This file can be put into the documentation
with the \LaTeX\ command {\verb@\input{LALHelloNRCSID}@}. The result is: 
\input{LALHelloNRCSID}
The marginpar is the line number and file name of where the fragment came
from.

The second pair of commands (\texttt{<lalLaTeX file="FileName">} and the
\texttt{</lalLaTeX>}) writes the intervening \LaTeX\ source to the file (no
verbatim wrapping this time) for later insertion.

\part{Documentation of the LAL packages}
\chapter{Package \texttt{std}}

This package contains headers providing basic datatypes, constants,
and macros that support the LAL standard.

\newpage\input{LALStdlibH}
\newpage\input{LALRCSIDH}
\newpage\input{LALDatatypesH}
\newpage\input{LALStatusMacrosH}
\newpage\input{LALConstantsH}
\newpage\input{LALStdioH}
\newpage\input{LALVersionH}
\newpage\input{LALMallocH}
\newpage\input{LALErrorH}
\newpage\input{LALGSLH}
\newpage\input{StringInputH}


\newpage\begin{thebibliography}{0}
\bibitem{Barnet:1996}
  Particle Data Group, R.~M. Barnett et al., Phys. Rev. D\textbf{54},
  1 (1996)
\bibitem{Lang:1992}
  K.~R. Lang, \textit{Astrophysical Data: Planets and Stars}.
  Springer-Verlag, New York (1992)
\end{thebibliography}

\chapter{Package \texttt{support}}

%% $Id$

This package covers LAL support routines.

These routines do not conform to LAL requirements, and many of them should be
used only for debugging and in test code, not in production code.  These are
compiled and installed as a separate library \texttt{lalsupport}.

\newpage\input{LALStdioH}
\newpage\input{LALVersionH}
\newpage\input{LALMallocH}
\newpage\input{LALErrorH}
\newpage\input{GridH}
\newpage\input{PrintVectorH}
\newpage\input{PrintFTSeriesH}
\newpage\input{ReadFTSeriesH}
\newpage\input{ReadNoiseSpectrumH}
\newpage\input{StringInputH}
\newpage\input{StreamInputH}
\newpage\input{StreamOutputH}
\newpage\input{LALInitBarycenterH}
\newpage\input{LALXMGRInterfaceH}
\newpage\input{LIGOLwXMLH}
\newpage\input{LIGOLwXMLReadH}
\newpage\input{SFTfileIOH}
\newpage\input{ConfigFileH}
\newpage\input{UserInputH}
\newpage\input{LALMathematicaH}

\section{Program \texttt{lalapps\_hello}}
\label{program:lalapps-hello}
\idx[Program]{lalapps\_hello}

\begin{entry}

\item[Name]
\verb$lal_hello$ --- prints ``hello LSC!''

\item[Synopsis]
\verb$lal_hello$ [\verb$-h$] [\verb$-V$] [\verb$-v$]
[\verb$-d$ \textit{dbglvl}] [\verb$-o$ \textit{outfile}]

\item[Description]
\verb$lal_hello$ prints ``hello LSC!'' to the screen or to an output file.

\item[Options]\leavevmode
\begin{entry}
\item[\texttt{-h}]
Print a help message.
\item[\texttt{-V}]
Print the version information.
\item[\texttt{-v}]
Verbose output.
\item[\texttt{-d} \textit{dbglvl}]
Set LAL debug level to \textit{dbglvl}.
\item[\texttt{-o} \textit{outfile}]
Write the output to file \textit{outfile}.
\end{entry}

\item[Debug levels]
The LAL debug level can be specified as an integer or as a string of flags:
\begin{entry}
\item[\texttt{NDEBUG}]
No debugging information is printed and memory debugging code is disabled.
\item[\texttt{ERROR}]
Error messages are printed.
\item[\texttt{WARNING}]
Warning messages are printed.
\item[\texttt{INFO}]
Information messages are printed.
\item[\texttt{TRACE}]
Function call tracing messages are printed.
\item[\texttt{MEMINFO}]
Memory  allocation  information messages are printed.
\item[\texttt{MEMDBG}]
Debugging of memory allocation routines is enabled but no messages are printed.
\end{entry}
The following composite levels are available:
\begin{entry}
\item[\texttt{MSGLVL1}]
Equivalent to \verb$ERROR$
\item[\texttt{MSGLVL2}]
Equivalent to \verb$ERROR | WARNING$
\item[\texttt{MSGLVL3}]
Equivalent to \verb$ERROR | WARNING | INFO$
\item[\texttt{ALLDBG}]
All debugging messages are printed.
\end{entry}

For example, the command
\begin{indented}
\verb$lal_hello -d "ERROR | INFO"$
\end{indented}
will set the debug level so that error and information messages are printed.

\item[Environment]\leavevmode

\begin{entry}
\item[\texttt{LAL\_DEBUG\_LEVEL}]
Default LAL debug level to use.
\end{entry}

\item[Author]
Jolien Creighton

\end{entry}

\chapter{Package \texttt{factories}}

This package provides routines for creating and destroying the LAL aggregate
datatypes.

\newpage\input{AVFactoriesH}
\newpage\input{SeqFactoriesH}

%\chapter{Package \texttt{vectorops}}

This package contains routines for manipulating vectors.

\newpage\input{VectorOpsH}
\newpage\input{VectorIndexRangeH}
\newpage\begin{thebibliography}{0}
\bibitem{ptvf:1992}
  W. H. Press, S. A. Teukolsky, W. T. Vetterling, and B. P. Flannery,
  \textit{Numerical Recipes in C: The Art of Scientific Computing}, 2nd ed.
  (Cambridge University Press, Cambridge, England, 1992).
\end{thebibliography}

%\chapter{Package \texttt{utilities}}

This package contains various numerical utilities for use in LAL.

\newpage\input{RandomH}
\newpage\input{FindRootH}
\newpage\input{IntegrateH}
\newpage\input{InterpolateH}
\newpage
\section{Header \texttt{Sort.h}}

Provides routines for sorting, indexing, and ranking real vector
elements.

\subsection{Synopsis}
\begin{verbatim}
#include "Sort.h"
\end{verbatim}


\subsection{Error conditions}
\begin{tabular}{|c|l|l|}
\hline
status & status                      & Explanation                      \\
 code  & description                 &                                  \\
\hline
\tt 1  & \tt Null pointer            & Missing a required pointer.      \\
\tt 2  & \tt Length mismatch         & Vectors are of different length. \\
\tt 3  & \tt Memory allocation error & Could not allocate memory.       \\
\hline
\end{tabular}

\subsection{Structures}
\newpage
\subsection{Module \texttt{HeapSort.c}}

Sorts, indexes, or ranks vector elements using the heap sort
algorithm.

\subsubsection{Prototypes}
\vspace{0.1in}
\input{HeapSortD}

\subsubsection{Description}

These routines sort a vector \verb@*data@ (of type \verb@REAL4Vector@
or \verb@REAL8Vector@) into ascending order using the in-place
heapsort algorithm, or construct an index vector \verb@*index@ that
indexes \verb@*data@ in increasing order (leaving \verb@*data@
unchanged), or construct a rank vector \verb@*rank@ that gives the
rank order of the corresponding \verb@*data@ element.

The relationship between sorting, indexing, and ranking can be a bit
confusing.  One way of looking at it is that the original array is
ordered by index, while the sorted array is ordered by rank.  The
index array gives the index as a function of rank; i.e.\ if you're
looking for a given rank (say the 0th, or smallest element), the index
array tells you where to look it up in the unsorted array:
\begin{verbatim}
unsorted_array[index[i]] = sorted_array[i]
\end{verbatim}
The rank array gives the rank as a function of index; i.e.\ it tells
you where a given element in the unsorted array will appear in the
sorted array:
\begin{verbatim}
unsorted_array[j] = sorted_array[rank[j]]
\end{verbatim}
Clearly these imply the following relationships, which can be used to
construct the index array from the rank array or vice-versa:
\begin{verbatim}
index[rank[j]] = j
rank[index[i]] = i
\end{verbatim}

\subsubsection{Algorithm}

These routines use the standard heap sort algorithm described in
Sec.~8.3 of Ref.~\cite{ptvf:1992}.

The \verb@SHeapSort()@ and \verb@DHeapSort()@ routines are entirely
in-place, with no auxiliary storage vector.  The \verb@SHeapIndex()@
and \verb@DHeapIndex()@ routines are also technically in-place, but
they require two input vectors (the data vector and the index vector),
and leave the data vector unchanged.  The \verb@SHeapRank()@ and
\verb@DHeapRank()@ routines require two input vectors (the data and
rank vectors), and also allocate a temporary index vector internally;
these routines are therefore the most memory-intensive.  All of these
algorithms are $N\log_2(N)$ algorithms, regardless of the ordering of
the initial dataset.


\subsubsection{Uses}
\begin{verbatim}
I4CreateVector()
I4DestroyVector()
\end{verbatim}

\subsubsection{Notes}


\newpage
\subsection{Program \texttt{SortTest.c}}

A program to test sorting routines.

\subsubsection{Usage}
\begin{verbatim}
SortTest [-s seed] [-d [debug-level]]
\end{verbatim}

\subsubsection{Description}

This test program creates rank and index arrays for an unordered list
of numbers, and then sorts the list.  The data for the list are
generated randomly, and the output is to \verb@stdout@ unless
redirected.  \verb@SortTest@ returns 0 if it executes successfully,
and 1 if any of the subroutines fail.

The \verb@-s@ option sets the seed for the random number generator; if
\verb@seed@ is set to zero (or if no \verb@-s@ option is given) then
the seed is taken from the processor clock.  The \verb@-d@ option
increases the default debug level from 0 to 1, or sets it to the
specified value \verb@debug-level@.


\subsubsection{Algorithm}

\subsubsection{Uses}
\begin{verbatim}
debuglevel
CreateI4Vector()
CreateSVector()
DestroyI4Vector()
DestroySVector()
CreateRandomParams()
DestroyRandomParams()
UniformDeviate()
LALPrintError()
\end{verbatim}

\subsubsection{Notes}



\newpage\input{ODEH}
\newpage\input{DirichletH}
\newpage\input{CoarseGrainFrequencySeriesH}
\newpage\input{LALRunningMedianH}
\newpage\input{RngMedBiasH}
\newpage\begin{thebibliography}{0}
\bibitem{ptvf:1992}
  W. H. Press, S. A. Teukolsky, W. T. Vetterling, and B. P. Flannery,
  \textit{Numerical Recipes in C: The Art of Scientific Computing}, 2nd ed.
  (Cambridge University Press, Cambridge, England, 1992).
\input{DirichletCB}
\end{thebibliography}

\include{tdfilters}
%\chapter{Package \texttt{window}}

This package contains a function to create a vector
containing a window (also called
a taper, lag window, or apodization function).  The choices
currently available are:
\begin{itemize}
\item Rectangular
\item Hann
\item Welch
\item Bartlett
\item Parzen
\item Papoulis
\item Hamming.
\end{itemize}
Using window functions is well documented in many places.  Their
principal purpose is to reduce the {\it bias} in power spectrum
estimation.  For example, if a sinusoidal signal is present, it will
give rise to a spike in a power spectrum.  If such a  signal is present
exactly at the frequency of a particular bin, the spike will have some
height.  If a signal at the same amplitude but slighly different
frequency is present, and the frequency is not exactly the same
as one of the bins, then
the spike will be broader and lower. Without widowing, this effect introduces
bias into spectral estimates.

If the signal is first multiplied by the window function, the
relative difference between the two resulting power spectra will be
less evident.  The signal which would have given a single-bin spike
will now be spread over several bins.  The signal that would have
been spread over several bins will now be somewhat more peaked.
In short, the two power spectra will appear more similar (apart from
the difference in frequency of the two signals).  Hence the spectra
is less biased.

Definitions of most of the window functions above may be found in {\it Numerical Recipes} \cite{numrec} equations 13.4.13-13.4.15.  Definitions
of the remaining windows can be found in {\it Spectral analysis for physical applications} \cite{pw} Section 6.11.


\newpage\input{WindowH}
\newpage\begin{thebibliography}{0}
\bibitem{numrec}
W. H. Press, S. A. Teukolsky, W. T. Vetterling, and B. P. Flannery,
  \textit{Numerical Recipes in C: The Art of Scientific Computing}, 2nd ed.
  (Cambridge University Press, Cambridge, England, 1992).
\bibitem{pw}
D.B. Percival and A.T. Walden, {\it Spectral analysis for physical applications}, first edition, Cambridge University Press, (1993).
\bibitem{pm}
J.G. Proakis and D.G. Manolakis, {\it Digital
Signal Processing: principles, algorithms and applications},
3rd Edition, 1995.
Prentice Hall
\end{thebibliography}

%\chapter{Package \texttt{fft}}

This package contains various routines for performing FFTs.

\newpage\input{RealFFTH}
\newpage\input{ComplexFFTH}

\newpage\begin{thebibliography}{0}
\bibitem{fj:1998}
  M. Frigo and S. G. Johnson,
  \textit{FFTW User's Manual},
  (Massachusetts Institute of Technology, Cambridge, USA, 1998).
  URL: \texttt{http://www.fftw.org/doc}
\bibitem{ptvf:1992}
  W. H. Press, S. A. Teukolsky, W. T. Vetterling, and B. P. Flannery,
  \textit{Numerical Recipes in C: The Art of Scientific Computing}, 2nd ed.
  (Cambridge University Press, Cambridge, England, 1992).
\end{thebibliography}

%\include{timefreq}
%\section{Program \prog{lalapps\_stochastic}}
\label{program:lalapps-stochastic}
\idx[Program]{lalapps\_stochastic}

\begin{entry}
\item[Name]
\prog{lalapps\_stochastic} --- standalone stochastic analysis code.

\item[Synopsis]
\prog{lalapps\_stochastic} \newline \hspace*{0.5in}
[\option{--help}] \newline \hspace*{0.5in}
[\option{--version}] \newline \hspace*{0.5in}
[\option{--verbose}] \newline \hspace*{0.5in}
[\option{--debug-level}~\parm{N}] \newline \hspace*{0.5in}
[\option{--user-tag}~\parm{STRING}] \newline \hspace*{0.5in}
[\option{--comment}~\parm{STRING}] \newline \hspace*{0.5in}
[\option{--output-dir}~\parm{DIR}] \newline \hspace*{0.5in}
[\option{--cc-spectra}] \newline \hspace*{0.5in}
\option{--gps-start-time}~\parm{N} \newline \hspace*{0.5in}
\option{--gps-end-time}~\parm{N} \newline \hspace*{0.5in}
\option{--interval-duration}~\parm{N} \newline \hspace*{0.5in}
\option{--segment-duration}~\parm{N} \newline \hspace*{0.5in}
\option{--resample-rate}~\parm{N} \newline \hspace*{0.5in}
\option{--f-min}~\parm{N} \newline \hspace*{0.5in}
\option{--f-max}~\parm{N} \newline \hspace*{0.5in}
\option{--ifo-one}~\parm{IFO} \newline \hspace*{0.5in}
\option{--ifo-two}~\parm{IFO} \newline \hspace*{0.5in}
\option{--channel-one}~\parm{CHANNEL} \newline \hspace*{0.5in}
\option{--channel-two}~\parm{CHANNEL} \newline \hspace*{0.5in}
\option{--frame-cache-one}~\parm{FILE} \newline \hspace*{0.5in}
\option{--frame-cache-two}~\parm{FILE} \newline \hspace*{0.5in}
\option{--calibration-cache-one}~\parm{FILE} \newline \hspace*{0.5in}
\option{--calibration-cache-two}~\parm{FILE} \newline \hspace*{0.5in}
\option{--calibration-offset}~\parm{N} \newline \hspace*{0.5in}
[\option{--apply-mask}~\parm{N} \newline \hspace*{0.5in}
\option{--mask-bin}~\parm{N}] \newline \hspace*{0.5in}
[\option{--overlap-hann} \newline \hspace*{0.5in}
\option{--hann-duration}~\parm{N}] \newline \hspace*{0.5in}
[\option{--high-pass-filter} \newline \hspace*{0.5in}
\option{--hpf-frequency}~\parm{N} \newline \hspace*{0.5in}
\option{--hpf-attenuation}~\parm{N} \newline \hspace*{0.5in}
\option{--hpf-order}~\parm{N}] \newline \hspace*{0.5in}
\option{--recentre} \newline \hspace*{0.5in}
\option{--middle-segment} \newline \hspace*{0.5in}
[\option{--geo-hpf-frequency}~\parm{N} \newline \hspace*{0.5in}
\option{--geo-hpf-attenuation}~\parm{N} \newline \hspace*{0.5in}
\option{--geo-hpf-order}~\parm{N}] \newline \hspace*{0.5in}
[\option{--alpha}~\parm{N}] \newline \hspace*{0.5in}
[\option{--f-ref}~\parm{N}] \newline \hspace*{0.5in}
[\option{--omega0}~\parm{N}]

\item[Description]
\prog{lalapps\_stochastic} runs the standalone stochastic analysis code.

\item[Options]\leavevmode
\begin{entry}
\item[\option{--help}]
Display usage information and exit.

\item[\option{--version}]
Display version information and exit.

\item[\option{--verbose}]
Enable the output of informational messages.

\item[\option{--debug-level}~\parm{N}]
Sets the LAL debug level to \parm{N}. The default value is
\texttt{LALMSGLVL2}, displaying error and warning messages. A useful
setting is 65 which turns off memory padding, but keeps memory tracking
and error messages. If you want to turn off memory tracking completly,
then use 33.

\item[\option{--user-tag}~\parm{STRING}]
Set the user tag to the string \parm{STRING}. This string must not
contain spaces or dashes (``-''). This string will appear in the name of
the file to which output information is written, and is recorded in the
various XML tables within the file.

\item[\option{--comment}~\parm{STRING}]
Set the process table comment to \parm{STRING}

\item[\option{--output-dir}~\parm{DIR}]
Set the output directory for search results to \parm{DIR}

\item[\option{--cc-spectra}]
Save out cross correlation spectra as frame files.

\item[\option{--gps-start-time}~\parm{N}]
Sets the GPS time from which data should be read to \parm{N}

\item[\option{--gps-end-time}~\parm{N}]
Sets the GPS time to which data should be read to \parm{N}

\item[\option{--interval-duration}~\parm{N}]
Sets the interval duration to \parm{N}

\item[\option{--segment-duration}~\parm{N}]
Sets the segment duration to \parm{N}

\item[\option{--resample-rate}~\parm{N}]
Down-convert the input data stream to a sample rate of \parm{N} samples
per second prior to analysis.  This can be used to reduce the number of CPU
cycles required to analyze a given quantity of input data.

\item[\option{--f-min}~\parm{N}]
Sets the minimum frequency of the search band to \parm{N}

\item[\option{--f-max}~\parm{N}]
Sets the maximum frequency of the search band to \parm{N}

\item[\option{--ifo-one}~\parm{IFO}]
Sets the IFO for the first stream to be \parm{IFO}, currently supported
IFO's are H1, H2, L1 and G1

\item[\option{--ifo-two}~\parm{IFO}]
Sets the IFO for the second stream to be \parm{IFO}, currently supported
IFO's are H1, H2, L1 and G1

\item[\option{--channel-one}~\parm{CHANNEL}]
Sets the channel for the first stream to be \parm{CHANNEL}

\item[\option{--channel-two}~\parm{CHANNEL}]
Sets the channel for the second stream to be \parm{CHANNEL}

\item[\option{--frame-cache-one}~\parm{FILE}]
Obtain the locations of input \texttt{.gwf} frame files from the LAL frame
cache file \parm{FILE} for the first detector.  LAL frame cache files
are explained in the ``framedata'' package in LAL and can be constructed
by making calls to \prog{LSCdataFind} on some systems.

\item[\option{--frame-cache-two}~\parm{FILE}]
Obtain the locations of input \texttt{.gwf} frame files from the LAL frame
cache file \parm{FILE} for the second detector.  LAL frame cache files
are explained in the ``framedata'' package in LAL and can be constructed
by making calls to \prog{LSCdataFind} on some systems.

\item[\option{--calibration-cache-one}~\parm{FILE}]
Specify the location of calibration information for the first detector.
\parm{FILE} gives the path to a LAL-format frame cache file describing
locations of \texttt{.gwf} frame files that provide the calibration data
($\alpha$ and $\beta$ coefficients) for the analysis.  Frame cache files
are explained in the ``framedata'' package in LAL.

\item[\option{--calibration-cache-two}~\parm{FILE}]
Specify the location of calibration information for the second detector.
\parm{FILE} gives the path to a LAL-format frame cache file describing
locations of \texttt{.gwf} frame files that provide the calibration data
($\alpha$ and $\beta$ coefficients) for the analysis.  Frame cache files
are explained in the ``framedata'' package in LAL.

\item[\option{--calibration-offset}~\parm{N}]
Sets the calibration offset to \parm{N}

\item[\option{--apply-mask}]
Apply frequency masking

\item[\option{--mask-bin}~\parm{N}]
Set the number of bins to mask per frequency to \parm{N}

\item[\option{--overlap-hann}]
Use overlapping Hann windows for data segments

\item[\option{--hann-duration}~\parm{N}]
Set the Hann duration of the data segment window to \parm{N}, 0 for
Rectangular windowing, 1 for Tukey windowing and 60 for Hann windowing

\item[\option{--high-pass-filter}]
Apply a high pass filter to the input data

\item[\option{--hpf-frequency}~\parm{N}]
Set the knee frequency of the high pass filter to \parm{N}

\item[\option{--hpf-attenuation}~\parm{N}]
Set the attenuation coefficent for the high pass filter to \parm{N}

\item[\option{--hpf-order}~\parm{N}]
Sets the high pass filter order to \parm{N}

\item[\option{--recentre}]
Centre the data

\item[\option{--middle-segment}]
Include the middle segment in the power spectra estimation

\item[\option{--geo-hpf-frequency}~\parm{N}]
Set the knee frequency for the GEO high pass filter to \parm{N}

\item[\option{--geo-hpf-attenuation}~\parm{N}]
Set the attenuation coefficient for the GEO high pass filter to \parm{N}

\item[\option{--geo-hpf-order}~\parm{N}]
Set the GEO high pass filter order to \parm{N}

\item[\option{--alpha}~\parm{N}]
Exponent for $\Omega_{\mathrm{GW}}$ for construction of the optimal
filter.

\item[\option{--f-ref}~\parm{N}]
Reference frequency for $\Omega_{\mathrm{GW}}$ for the construction of
the optimal filter.

\item[\option{--omega0}~\parm{N}]
Reference $\Omega_0$ for $\Omega_{\mathrm{GW}}$ for the construction of
the optimal filter.
\end{entry}

\item[Example]
\prog{lalapps\_stochastic} is generally run as part of a DAG, as created
by \prog{lalapps\_stochastic\_pipe} or
\prog{lalapps\_stochastic\_bayes}, however an example usage can be seen
below.

\begin{verbatim}
> lalapps_stochastic --debug-level 33 --verbose \
>   --gps-start-time 752242398 --gps-end-time 752242758 \
>   --interval-duration 180 --segment-duration 60 \
>   --resample-rate 1024 --f-min 50 --f-max 250 --ifo-one H1 \
>   --ifo-two H2 --channel-one LSC-AS_Q --channel-two LSC-AS_Q \
>   --frame-cache-one H1.cache --frame-cache-two H2.cache \
>   --calibration-cache-one H1-CAL-V02-751651244-757699245.cache \
>   --calibration-cache-two H2-CAL-V02-751651244-757699245.cache \
>   --calibration-offset 0 --hann-duration 1 --cc-spectra
\end{verbatim}

\item[Author] 
Adam Mercer, Tania Regimbau
\end{entry}

%\chapter{Package \texttt{framedata}}

Package for reading frame-format data files.

\newpage\input{FrameDataH}
\newpage\input{SpecBufferH}
\newpage\input{DataBufferH}

%\chapter{Package \texttt{comm}}

This package contains routines for using MPI to exchange LAL data structures.

\newpage\input{CommH}

%\clearpage

\section{Findchirp Programs}
\label{section:findchirp}

This section of \textsc{LALApps} contains programs part of the findchirp packages
related to inspiral studies. 



%\chapter{Package \texttt{pulsar}: common routines}
Teviet Creighton
\bigskip

This package provides routines for timing, metric calculation and
mesh-generation relevant for pulsar searches.

\newpage\input{PulsarTimesH}
\newpage\input{LALBarycenterH}
\newpage\input{FlatMeshH}
\newpage\input{TwoDMeshH}
\newpage\input{TwoDMeshPlotH}
\newpage\input{ResampleH}

\newpage\begin{thebibliography}{0}
\bibitem{Brady_P:2000}
  P.~R. Brady and T. Creighton, Phys. Rev. D\textbf{61}, 082001
  (2000).
\end{thebibliography}

\chapter{Package \texttt{pulsar}: amplitude folding routines}
Greg Mendell
\bigskip

Contains function LALFoldAmplitudes: folds amplitudes into phase bins.

\begin{verbatim}
Files:

FoldAmplitudes.h        header file
FoldAmplitudes.c        source code
FoldAmplitudesTest.c    test code
foldamplitudes.tex      overview
\end{verbatim}

Periodic sources of gravitational radiation will produce measured strains of the following form:
$$
c[i] = A(t_i,\vec{\lambda}) \sin[\Phi(t_i,\vec{\lambda})] + n(t_i)
$$
In this equation $c[i]$ is the discrete time series output of the detector (perhaps after some data conditioning, such as
being resampled, narrow banded, or with instrument line noise removed).
The amplitude, $A(t_i,\vec{\lambda})$, is assumed roughly constant at the gravity wave source,
but is modulated by variation in the detector's response due to the Earth's motion.  The phase, $\Phi(t_i,\vec{\lambda})$,
is modulated by both the intrinsic spin down of the source, and the changes in relative motion between the source
and the detector.  This can be calculated for known pulsars.  The vector $\vec{\lambda}$ is a vector of parameters
that describe the sky position, etc., of the source and location, etc., of the detector.
Finally, $n(t_i)$ is the noise, which also includes any other signals that are not coherent with
the phase $\Phi(t_i,\vec{\lambda})$.

The folded amplitude is given by
$$
c_{\rm F} [j] = \sum_{i'}
\left \{ A(t_i,\vec{\lambda})\sin[\Phi(t_i,\vec{\lambda})] + n(t_i) \right \} ,
$$
where the sum over $i'$ means sum over all $i$'s with $\Phi$ in phase bin $j$.
If the bin sizes are sufficiently small, then $c_{\rm F} [j]$ can be approximated as
$$
c_{\rm F} [j] = \sin\Phi_j\sum_{i'} A(t_i,\vec{\lambda}) + \sum_{i'} n(t_i) ,
$$
where $\Phi_j$ is representive of the phase for bin $j$ (e.g., the phase corresponding to the midpoint of the bin).
However, because of amplitude modulation, the amplitudes that are added to a phase bin are not guaranteed to enter
with the same sign.  Thus, some sort of amplitude demodulation should be done.

If we demodulate $A(t_i,\vec{\lambda})$ (for example, in a minimum way such as multiplying by the sign
of the response function) we multiply each element of the vector $c[i]$ by an amplitude demodulation factor $D(t_i)$
$$
c_{\rm D\, , F} [j] = \sin\Phi_j \sum_{i'} D(t_i) A(t_i,\vec{\lambda}) + \sum_{i'} D(t_i) n(t_i) ,
$$
If the average value of $D(t_i)$ is zero, and is not correlated with the noise, then
$$
\sum_{i'} D(t_i) n(t_i) \approx 0
$$
However, the average value of $D(t_i)$ is probably not zero.
The following is a very preliminary suggestion of how to further reduce the noise.
Consider folding the measured strains, $c[i]$, again,
but this time shifting the phase bins by $\pi$.  Define this phase shifted folded amplitude as:
$$
c_{\pi, \, \rm D\, , F} [j] = \sin(\Phi_j + \pi) \sum_{i''} D(t_i) A(t_i,\vec{\lambda}) + \sum_{i''} D(t_i) n(t_i) ,
$$
where the sum over $i''$ means sum over all $i$'s with $\Phi + \pi$ in phase bin $j$.
This will reverse the sign of the sum of the amplitudes that enter into each phase bin, but
the sum of the noise contributions into each bin should be roughly the same.
If the signal we are searching for is present, then amplitudes, $A(t_i,\vec{\lambda})$ are correlated with $D(t_i)$ such that
$$
\sum_{i'} D(t_i) A(t_i,\vec{\lambda}) \approx \bar{A} = {\rm constant}
$$
Thus,
$$
c_{\rm D\, , F} [j] - c_{\pi, \, \rm D\, , F} [j] \approx 2 \bar{A}\sin\Phi_j ,
$$
plus residual noise.  In practice, one needs to fold the amplitudes only once, and then make the replacement
$$
c_{\rm D\, , F} [j] \rightarrow c_{\rm D\, , F} [j] - c_{\rm D\, , F} [(j + N/2) \, \% \, j] ,
$$
where $N$ is the number of phase bins.  We can then statistically analyze the hypothesis that
the demodulated folded amplitudes correspond to a sinusoid.

\newpage\input{FoldAmplitudesH}

\chapter{Package \texttt{pulsar}: Coherent search routines}
Steven Berukoff, M. Alessandra Papa
\bigskip

This package provides a routine to perform a demodulation on a set of data.
In particular, this routine works with frequency domain data by combining
short timescale Fourier Transforms (SFTs) into longer time baseline
demodulated Fourier Transforms (DeFTs). If the assumptions under which the
method was developed are met (\cite{Williams:1999}), then the demodulation
procedure concentrates the total power (within $5\%-10\%$) in a single
frequency bin. In practice, due to the discretization of frequency space, this
power may be shared between two neighbouring bins.

The procedure follows that outlined in \cite{Williams:1999} and is part of the
continuous-wave search algorithm outlined in \cite{Schutz:1999}.  Briefly, the
routine takes input SFTs, corrects for modulation effects due to intrinsic
frequency spindown and Earth's motion, and outputs a DeFT of long time
baseline. In general the routine can be easily adapted to correct for an
arbitrary modulation effect simply by the use of a suitable timing routine,
here \verb+tdb()+ .

The package is organized under the headers \verb+LALDemod.h+, \verb+LALComputeAM.h+, and
\verb+ComputeSky.h+ and the modules \verb+LALDemod.c+ and \verb+ComputeSky.c+.


\newpage\input{LALDemodH}
\newpage\input{ComputeSkyH}
\newpage\input{ComputeSkyBinaryH}
\newpage\input{LALComputeAMH}

\newpage\begin{thebibliography}{0}
\bibitem{Williams:1999}
        Peter R. Williams, Bernard F. Schutz.  gr-qc/9912029.
\bibitem{Schutz:1999}
        Bernard F. Schutz, M. A. Papa.  gr-qc/9905018.
\end{thebibliography}

\chapter{Package \texttt{pulsar}: known pulsar time-domain search routines}

This package provides routines for a time domain search of gravitational wave
signals from known pulsars.  The documentation and functionality of this
package is \textbf{incomplete}.
\newpage\input{BinaryPulsarTimingH}
\newpage\begin{thebibliography}{0}
\bibitem{TaylorWeisberg:1989}
        J. Taylor and J. Weisberg, \it{Ap. J.}, \bf{345}, 1989.
\bibitem{BlandfordTeukolsky:1976}
        R. Blandford and S. Teukolsky, \it{Ap. J.}, \bf{205}, 1976.
\bibitem{ChLangeetal:2001}
        Ch. Lange {\it et. al.}, \it{Mon. Not. R. Astron. Soc.}, \bf{326}, 2001.
\bibitem{DamourDeruelle:1985}
        T. Damour and N. Deruelle, \it{Ann. Inst. H. Poincar\'e (Phys. Th\'eorique)}, \bf{43}, 1985.
\bibitem{Wex:1998}
        N. Wex, \it{Mon. Not. R. Astron. Soc.}, \bf{298}, 1998.
\end{thebibliography}
\newpage\input{FitToPulsarH}
\newpage\input{PulsarCatH}



\part{Some Coding Details:}
\chapter{The RCSID goes in every code file}
In every code file (i.e., \texttt{file.h, file.c}) there is macro that looks
like
\input{LALHelloNRCSID}
This macro assign a character variable (in this case \texttt{LALHELLOC} in the
file \texttt{LALHello.c}) the string with the cvs-supplied revision numbers.
The reason we use this macro is that it gives no "unused variable" warnings
form the \texttt{.h}-files when the code is compiled with the \texttt{-Wall}
option.

\chapter{The status macro}

\printindex

\end{document}
