The detection process for binary inspiral chirps is simply the construction of
some statistic of the data followed by some test of the hypotheses ``there is
a signal present in the data'' and ``there is no signal present in the data.''
When these hypotheses are considered to be exclusive, the reception process
reduces to the comparison of the statistic with some pre-assigned threshold.
Thus, there are two issues: first, what is the \emph{optimal} statistic to
construct, and second, how is the threshold to be determined.

We show the construct the statistic for the maximum likelihood reciever
used in the \texttt{findchirp} package. It can be shown \cite{wz} that this is
the optimal statistic for detection of inspiral chirps when the waveform of
the chirp is known.

The modules under this header provide a way to constuct the statistic
$\rho^2(t)$ from data pre-conditioned using other functions from the
\texttt{findchirp} package and to perform a comparison with the pre-assigned
thresold to generate events, calling the \texttt{FindChirpChisq} function to
perform a $\chi^2$ veto on the events, if requested.

\subsubsection*{Construction of the Matched Filter}

Suppose that the detector output is a dimensionless strain $h(t)$. We denote
by $s(t)$ the waveform of the signal (i.e. chirp) that we wish to find in the
data stream hidden in detector noise, $n(t)$. We assume that the noise samples
are drawn from a stationary Normal distribution, though there may be
correlations amongst the noise events (colored noise). The noise correlations
may be expressed in terms of the one-sided noise power spectral density,
\ospsd, defined in equation [\ref{findchirp:eqn:ospsd}]. The detector output
is then
\begin{equation}
h(t) = s(t) + n(t).
\end{equation}
These functions are members of the Banach space $L_2$, so we may construct an
inner product on this space $(\cdot\mid\cdot) : L_2 \times L_2 \rightarrow
\mathcal{C}$ by
\begin{equation}
  (a\mid b) = \int_{-\infty}^\infty df\,
  \frac{\tilde{a}^\ast(f)\tilde{b}(f)+\tilde{a}(f)\tilde{b}^\ast(f)}
       {S_h(|f|)}.
\end{equation}

The signal we wish to recieve is, in general, arbitrary up to an unknown phase
and amplitude. The signal is given by
\begin{equation}
s(t)= F_{+}h_{+}(t) + F_{\times}h_{\times}(t),
\end{equation}
where $h_{+}(t)$ and $h_{\times}(t)$ are the two polarizations of the
gravitational wave signal and $F_{+}$ and $F_{\times}$ are the detector beam
pattern funtions. The ``$+$'' (plus) polarization is given by
\begin{equation}
h_{+}(t) = -\frac{T_\odot c}{D}(1 + \cos^2 i)h_c(t)
\end{equation}
and the ``$\times$'' (cross) polarization is given by
\begin{equation}
h_{+}(t) = -\frac{2 T_\odot c}{D}(\cos i)h_s(t).
\end{equation}
where $D$ is the distance to the source and $i$ is the inclination angle
(radians) of the angular momentum axis relative to the line-of-sight. The
waveforms $h_c(t)$ and $h_s(t)$ are refered to as the cosine and sine
polarizations of the inspiral signal.

We suppose further that the sine and cosine polarizatuons of the inspiral
signal are orthogonal, that is
\begin{equation}
\left(h_c\mid h_s\right) = 0
\end{equation}
and
\begin{equation}
\left(h_c\mid h_c\right) = \left(h_s\mid h_s\right) = \sigma^2
\end{equation}
which defines the normalisation constant $\sigma$.

The inspiral signals that we wish to recieve are much shorter than the
observation time, and we do not know when the signal occour. We wish not only
to detect these signals, but also to maesure their arrival time. To do so, we
constuct the time series
\begin{equation}
x(t) = 2 \int_{\infty}^{\infty}df\,e^{2\pi i f t} 
\frac{\tilde{h}(f) \tilde{h_c}^\ast(f)}{S_h\left(\left|f\right|\right)}
\end{equation}
and
\begin{equation}
y(t) = 2 \int_{\infty}^{\infty}df\,e^{2\pi i f t} 
\frac{\tilde{h}(f) \tilde{h_s}^\ast(f)}{S_h\left(\left|f\right|\right)}.
\end{equation}
For each observation period, we construct the statistic
\begin{equation}
\rho(t) = \frac{1}{\sigma}\sqrt{x^2(t) + y^2(t)}
\end{equation}
and threshold on this statistic, which we call the signal-to-noise ratio
(SNR).

Since the time input data are real time series, we have $h(f) = h^\ast(-f)$
and the inner product becomes
\begin{equation}
\left(a\mid b\right) = 2 \int_{-\infty}^{\infty}df\,
\frac{\tilde{a}^\ast(f)\tilde{b}(f)}{S_h\left(\left|f\right|\right)}
\end{equation}
so the normalisation constant $\sigma$ is
\begin{eqnarray}
\sigma^2 &=& 2 \int_{-\infty}^{\infty}df\,
\frac{\tilde{h_c}^\ast(f)\tilde{h_c}(f)}{S_h\left(\left|f\right|\right)} \\
&=& 2 \int \frac{\tilde{h_s}^\ast(f)\tilde{h_s}(f)}{S_h\left(\left|f\right|\right)}.
\end{eqnarray}

The filter that we have constructed is normalised according to the convention
of Cutler and Flanagan \cite{cutflan}, so that in the case when the detector
output is Gaussian noise, the filter output averaged over an ensemble of
detectors with different realisatons of the noise is
\begin{equation}
\left\langle \rho^2 \right\rangle = \left\langle x^2 + y^2 \right\rangle = 2.
\end{equation}

Since the two chirp waveforms $\tilde{h_c}$ and $\tilde{h_s}$ are assumed to
be orthogonal, it is possible to increase the efficency of the algorithm by
contructing the single time $\rho(t)$ directly using a single complex inverse
FFT rather than computing it from $x(t)$ and $y(t)$ which requires two real
inverse FFTs. Let us write $x(t)$ in the following way
\begin{eqnarray}
x(t) 
&=& 2 \left[ 
\int_{-\infty}^{0} df\, e^{2\pi i f t} \frac{\tilde{h}(f)
\tilde{h_c}^\ast(f)}{S_h\left(\left|f\right|\right)}
+ \int_{0}^{\infty} df\, e^{2\pi i f t} \frac{\tilde{h}(f)
\tilde{h_c}^\ast(f)}{S_h\left(\left|f\right|\right)} 
\right] \\
&=& 2 \left[ 
\int_{0}^{\infty} df\, e^{-2\pi i f t} \frac{\tilde{h}^\ast(f)
\tilde{h_c}(f)}{S_h\left(\left|f\right|\right)}
+ \int_{0}^{\infty} df\, e^{2\pi i f t} \frac{\tilde{h}(f)
\tilde{h_c}^\ast(f)}{S_h\left(\left|f\right|\right)} 
\right] \\
&=& 2 \left[ I^\ast + I \right],
\end{eqnarray}
similarly for $y(t)$
\begin{eqnarray}
y(t) 
&=& 2 \left[ 
\int_{-\infty}^{0} df\, e^{2\pi i f t} \frac{\tilde{h}(f)
\tilde{h_s}^\ast(f)}{S_h\left(\left|f\right|\right)}
+ \int_{0}^{\infty} df\, e^{2\pi i f t} \frac{\tilde{h}(f)
\tilde{h_s}^\ast(f)}{S_h\left(\left|f\right|\right)} 
\right] \\
&=& 2 \left[ 
\int_{0}^{\infty} df\, e^{-2\pi i f t} \frac{\tilde{h}^\ast(f)
\tilde{h_s}(f)}{S_h\left(\left|f\right|\right)}
+ \int_{0}^{\infty} df\, e^{2\pi i f t} \frac{\tilde{h}(f)
\tilde{h_c}^\ast(f)}{S_h\left(\left|f\right|\right)} 
\right] \\
&=& 2 \left[ 
\int_{0}^{\infty} df\, e^{-2\pi i f t} \frac{\tilde{h}^\ast(f)
i\tilde{h_c}(f)}{S_h\left(\left|f\right|\right)}
+ \int_{0}^{\infty} df\, e^{2\pi i f t} \frac{\tilde{h}(f)
(-i)\tilde{h_s}^\ast(f)}{S_h\left(\left|f\right|\right)} 
\right] \\
&=& -2i \left[ 
- \int_{0}^{\infty} df\, e^{-2\pi i f t} \frac{\tilde{h}^\ast(f)
\tilde{h_c}(f)}{S_h\left(\left|f\right|\right)}
+ \int_{0}^{\infty} df\, e^{2\pi i f t} \frac{\tilde{h}(f)
\tilde{h_s}^\ast(f)}{S_h\left(\left|f\right|\right)} 
\right] \\
&=& -2i \left[ - I^\ast + I \right],
\end{eqnarray}
where $I$ is defined to be
\begin{equation}
I = \int_{0}^{\infty} df\, e^{2\pi i f t}\frac{\tilde{h}(f)
\tilde{h_s}^\ast(f)}{S_h\left(\left|f\right|\right)}.
\end{equation}
Now we define the quantity $z(t)$ to be
\begin{equation}
z(t) = 4 I = 4 \int_{0}^{\infty} df\, e^{2\pi i f t}\frac{\tilde{h}(f)
\tilde{h_s}^\ast(f)}{S_h\left(\left|f\right|\right)}
\end{equation}
and then
\begin{eqnarray}
x(t) &=& \Re z(t) \\
y(t) &=& \Im z(t)
\end{eqnarray}
so the statistic $\rho(t)$ can be written as
\begin{equation}
\rho(t) = \frac{1}{\sigma} \sqrt{x^2(t) + y^2(t)} 
= \frac{1}{\sigma} \sqrt{|z^2(t)|}.
\end{equation}
In fact the statistic that we use in \texttt{findchirp} to threshold events on
is the signal to noise squared, $\rho^2 (t)$, which is
\begin{equation}
\rho^2(t) = \frac{1}{\sigma^2} \left|z^2(t)\right|.
\end{equation}
We can then compare this statistic to some pre-assigned threshold and decide
whether or not events are present in the data or construct a veto statistic
as desired.

\subsubsection*{Construction of the Digital Filter}
