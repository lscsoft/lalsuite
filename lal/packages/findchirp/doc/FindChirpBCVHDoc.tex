For the binary black hole inspiral we use the BCV templates.

\subsubsection*{BCV Templates}
\label{BCV}

The signal-to-noise ratio (SNR) for a signal $s$ and a template $h$ is given by
\begin{equation}
<s,h> = 2 \int_{-\infty}^{\infty} \frac{\tilde{s}^{\ast}(f) \tilde{h}(f)}
{S_h(|f|)} df =
4 \Re \int_0^{\infty} \frac{\tilde{s}^{\ast}(f) \tilde{h}(f)}{S_h(f)} df
\label{SNRdef}
\end{equation}
and $S_h(f)$ being the one-sided noise power spectral density.
The last equality in Eq.~(\ref{SNRdef}) holds only if $\tilde{s}(f)$ and
$\tilde{h}(f)$ are the Fourier-transforms of real time-functions.

The effective frequency-domain template given by Buonanno, Chen and Vallisneri
\cite{BCVPaper} is
\begin{equation}
\tilde{h}(f) = A(f) e^{i \psi(f)}
\label{tmplt}
\end{equation}
where
\begin{equation}
A(f) = f^{-7/6} (1-\alpha f^{2/3}) \theta (f_{cut}-f),
\end{equation}
\begin{equation}
\psi(f) = \phi_0 + 2 \pi f t_0 + f^{-5/3} (\psi_0 + \psi_{1/2} f^{1/3} +
\psi_1 f^{2/3} + \psi_{3/2} f + \ldots).
\end{equation}
In these expressions, $t_0$ and $\phi_0$ are the time of arrival and the
frequency-domain phase offset respectively,
and $\theta$ is the Heaviside step function.
For most inspiral templates approximated by the template (\ref{tmplt}),
it is sufficient to use the parameters $\psi_0$ and $\psi_{3/2}$ and set all
other $\psi$ coefficients equal to 0.
So in the following:
\begin{eqnarray}
\psi(f) &=& \phi_0 + 2 \pi f t_0 + f^{-5/3} (\psi_0 + \psi_{3/2} f)\\
&=& \phi_0 + \psi'(f) . \\
\end{eqnarray}

To simplify the equations, the abbreviation
\begin{equation}
I_k \equiv \int_0^{f_{cut}} \frac{df}{f^k S_h(f)}
\end{equation}
is used in the following.

Notice that in the code, $\psi_0$ is \texttt{psi0} and $\psi_{3/2}$ is
\texttt{psi3}.

\subsubsection*{Normalization of the Template}
\label{NormOfTemplate}

The amplitude $A(f)$ can be written as:
\begin{equation}
A(f) = A_1(f) + \alpha A_2(f)
\end{equation}
where
\begin{eqnarray}
A_1(f) &=& f^{-7/6} \theta(f_{cut}-f) \\
A_2(f) &=& - f^{-1/2} \theta(f_{cut}-f).
\end{eqnarray}
Then $\tilde{h}(f)$ can be written as
\begin{equation}
\tilde{h}(f) = \tilde{h}_1(f) + \alpha \tilde{h}_2(f)
\end{equation}
where
\begin{eqnarray}
\tilde{h}_1(f) &=& A_1(f) e^{i \psi(f)} = f^{-7/6} \theta(f_{cut}-f)
e^{i \psi(f)} \\
\tilde{h}_2(f) &=& A_2(f) e^{i \psi(f)} = - f^{-1/2} \theta(f_{cut}-f)
e^{i \psi(f)}.
\end{eqnarray}

We begin by orthonormalizing $\tilde{h}_1(f)$ and $\tilde{h}_2(f)$, which 
results in
$\hat{h}_1(f)$ and $\hat{h}_2(f)$.
Specifically:
\begin{eqnarray}
\hat{h}_1(f) &=& c_1 h_1(f) = (a_1 + i b_1) h_1(f) \\
\hat{h}_2(f) &=& c_2 h_2(f) = (a_2 + i b_2) h_2(f)
\end{eqnarray}
where $a_1$, $b_1$, $a_2$ and $b_2$ are real numbers.
Normalizing $\tilde{h}_1$ gives:
\begin{eqnarray}
&& <\hat{h}_1,\hat{h}_1> = 1  \quad \Rightarrow \quad
 4 \Re \int_0^{\infty} \frac{c_1^{\ast} \tilde{h}_1^{\ast}(f) c_1 
\tilde{h}_1(f)}{S_h(f)} df
= 1 \:  \Rightarrow \\
&& |c_1|^2 \int_0^{\infty} \frac{f^{-7/3} \theta(f_{cut} -f)}{S_h(f)} df
= \frac{1}{4} \quad \Rightarrow \quad
 a_1^2 + b_1^2 = \frac{1}{4 I_{7/3}}. \\
\label{Normalization1}
\end{eqnarray}
Normalizing $\tilde{h}_2$ gives:
\begin{eqnarray}
&& <\hat{h}_2,\hat{h}_2> = 1  \quad \Rightarrow \quad
 4 \Re \int_0^{\infty} \frac{c_2^{\ast} \tilde{h}_2^{\ast}(f) c_2 
\tilde{h}_2(f)}{S_h(f)} df
= 1 \:  \Rightarrow \\
&& |c_2|^2 \int_0^{\infty} \frac{f^{-1} \theta(f_{cut} -f)}{S_h(f)} df
= \frac{1}{4} \quad \Rightarrow \quad
 a_2^2 + b_2^2 = \frac{1}{4 I_{1}}. \\
\label{Normalization2}
\end{eqnarray}
Orthonormalizing $\hat{h}_1$ and $\hat{h}_2$ gives:
\begin{eqnarray}
&&<\hat{h}_1,\hat{h}_2> = 0 \quad \Rightarrow \quad
4 \Re \int_0^{\infty} \frac{c_1^{\ast} \tilde{h}_1^{\ast}(f) c_2 
\tilde{h}_2(f)}{S_h(f)} df
= 0 \quad \Rightarrow \quad \\
&& \Re \big \{ (a_1-i b_1)(a_2 +i b_2) \Big \} \int_0^{\infty} \frac{f^{-5/3}
\theta(f_{cut}-f)}{S_h(f)} df = 0 \quad \Rightarrow a_1 a_2+b_1 b_2 =0
. \\
\label{Normalization3}
\end{eqnarray}
In solving Eqs (\ref{Normalization1}), (\ref{Normalization2}) and
(\ref{Normalization3}), one of the unknowns
 can be set equal to an arbitrary number.
It is convenient to set $b_2=0$.
Then the remaining 3 unknowns are calculated:
\begin{eqnarray}
a_1 &=& 0 \\
b_1 &=& \frac{1}{2 \sqrt{I_{7/3}}} \label{b1}\\
a_2 &=& \frac{1}{2 \sqrt{I_1}}.
\label{a2}
\end{eqnarray}

The template $\tilde{h}(f)$ can also be normalized.
Specifically, it is assumed that the normalized template is
\begin{equation}
\hat{h}(f) = g \tilde{h}(f) = (g_1 + i g_2) \tilde{h}(f)
\end{equation}
where $g_1$ and $g_2$ are real numbers.
Taking into account the fact that $\hat{h}_1$ and $\hat{h}_2$ are
orthonormalized,
the normalization of $h$ gives
\begin{eqnarray}
&& <\hat{h}, \hat{h}> = 1 \: \Rightarrow \:
4 \Re |g|^2 \int_0^{\infty} \frac{\hat{h}_1^{\ast} \hat{h}_1 +
\alpha \hat{h}_1^{\ast} \hat{h}_2 + \alpha \hat{h}_1 \hat{h}_2^{\ast} +
\alpha^2 \hat{h}_2^{\ast} \hat{h}_2 }{S_h(f)} df = 1 \: \Rightarrow \\
&& g_1^2 + g_2^2 = \frac{1}{1+\alpha^2} \\
\label{Normalize}
\end{eqnarray}
We set $g_2 = 0$ for simplicity, so
\begin{equation}
g = g_1 = \frac{1}{\sqrt{1 + \alpha^2}}.
\end{equation}

Finally, the normalized template is
\begin{eqnarray}
\hat{h}(f) &=& \frac{1}{\sqrt{1+\alpha^2}} \Big [ \hat{h}_1(f) +
\alpha \hat{h}_2(f) \Big ] \\
    &=& \frac{1}{\sqrt{1+\alpha^2}} \Big [ i b_1 A_1(f) e^{i \psi(f)} +
\alpha \: a_2 A_2(f) e^{i \psi(f)} \Big ] \\
\label{NormalizedTmplt}
\end{eqnarray}
where $b_1$ and $a_2$ are given by Eqs (\ref{b1}) and (\ref{a2}) respectively.


\subsubsection*{Maximization of the SNR}
\label{SNRMaximization}

The Fourier-transformed data can, in general, be written as:
\begin{equation}
\tilde{s}(f) = R(f) + i I(f).
\end{equation}
The following brakets are defined:
\begin{eqnarray}
\hat{z}_1 &=& <\tilde{s}, \hat{h}_1> \\
\hat{z}_2 &=& <\tilde{s}, \hat{h}_2> .
\end{eqnarray}
Calculation of those gives:
\begin{eqnarray}
\hat{z}_1 &=&  4 \Re \int_{0}^{\infty} \frac{\tilde{s}^{\ast}(f)
\hat{h}_1(f)}{S_h(f)} df
    =  4 \Re \int_{0}^{\infty} \frac{ (R-iI) b_1 A_1 e^{i (\psi+\pi/2)}}
{S_h} df \\
    &=& \int_0^{\infty} 4 \frac{b_1 A_1}{S_h} (-R \sin\psi + I\cos\psi) df \\
    &=&\int_0^{\infty} \frac{4 b_1 A_1}{S_h} \Big [ -R (\sin\psi'\cos\phi_0
+\sin\phi_0 \cos\psi') + I(\cos\psi'\cos\phi_0-\sin\psi'\sin\phi_0) \Big ] df\\
    &=& \cos\phi_0 \int_0^{\infty} \frac{4 b_1 A_1}{S_h} (-R \sin\psi'+
I\cos\psi') df \\
    &&+\sin\phi_0 \int_0^{\infty} \frac{4 b_1 A_1}{S_h} (-R\cos\psi'
-I\sin\psi') df \\
    &=&\cos\phi_0 K_1 + \sin\phi_0 K_3 
\label{z1}
\end{eqnarray}
where the integrals $K_1$ and $K_3$ are defined by the above equation.
By a similar calculation we obtain
\begin{eqnarray}
\hat{z}_2 &=&  4 \Re \int_{0}^{\infty} \frac{\tilde{s}^{\ast}(f) \hat{h}_2(f)}
{S_h(f)} df \\
    &=& \cos\phi_0 \int_0^{\infty} \frac{4 a_2 A_2}{S_h} (R\cos\psi'
+I\sin\psi') df  \\
    &&+\sin\phi_0 \int_0^{\infty} \frac{4 a_2 A_2}{S_h} (-R\sin\psi'
+I\cos\psi')df\\
    &=& \cos\phi_0 K_2 + \sin\phi_0 K_4.
\label{z2}
\end{eqnarray}
Notice that
\begin{eqnarray}
K_1 &=& \int_0^{\infty} \frac{4 b_1 A_1}{S_h}(-R\sin\psi'+I\cos\psi')df \\
    &=&\int_0^{\infty} \frac{4 b_1 A_1}{S_h} \bigg \{ - \Im\Big [(R-iI)
	 (\cos\psi'+i\sin\psi')\Big ] \bigg \} df \\
	&=&-\Im \Big \{\int_0^{\infty} df \frac{4 b_1 A_1}{S_h} \tilde{s}^{\ast}
	 e^{i \psi'}  \Big \} 
\label{K1}
\end{eqnarray}
and by the same reasoning
\begin{eqnarray}
\label{K2}
K_2 &=& \Re \Big \{ \int_0^{\infty} df \frac{4 a_2 A_2}{S_h} \tilde{s}^{\ast}
	e^{i \psi'} \Big \} \\
\label{K3}
K_3 &=& - \Re \Big \{ \int_0^{\infty} df \frac{4 b_1 A_1}{S_h} \tilde{s}^{\ast}
	e^{i \psi'} \Big \} \\
\label{K4}
K_4 &=& -\Im \Big \{ \int_0^{\infty} df \frac{4 a_2 A_2}{S_h} \tilde{s}^{\ast}
	e^{i \psi'} \Big \}.
\end{eqnarray}

For the normalized template $\hat{h}$, the SNR is
\begin{eqnarray}
\rho &=& <\tilde{s},\hat{h}> = <\tilde{s}, \frac{1}{\sqrt{1+\alpha^2}}
\hat{h}_1 + \frac{\alpha}
{\sqrt{1+\alpha^2}} \hat{h}_2 > \\
    &=& \frac{1}{\sqrt{1+\alpha^2}} \hat{z}_1 + \frac{\alpha}
{\sqrt{1+\alpha^2}} \hat{z}_2 .\\
    &=& \frac{1}{\sqrt{1+\alpha^2}} (\cos\phi_0 K_1 + \sin\phi_0 K_3) +
 \frac{\alpha}{\sqrt{1+\alpha^2}} (\cos\phi_0 K_2 + \sin\phi_0 K_4) \\
\end{eqnarray}
Set:
\begin{eqnarray}
\alpha &=&\tan\omega \quad \Rightarrow \quad \frac{\alpha}{\sqrt{1+\alpha^2}} =
\sin\omega \\
    && \frac{1}{\sqrt{1+\alpha^2}} = \cos\omega.
\end{eqnarray}
Then:
\begin{eqnarray}
\rho &=& K_1 \cos\omega \cos\phi_0 + K_3 \cos\omega \sin\phi_0 + K_2 \sin\omega
\cos\phi_0 + K_4 \sin\omega \sin\phi_0 \: \Rightarrow \\
\rho &=& \frac{1}{2} K_1 [\cos(\omega+\phi_0) + \cos(\omega-\phi_0)] +
    \frac{1}{2} K_2 [\sin(\omega+\phi_0) + \sin(\omega-\phi_0)]  \\
&& +\frac{1}{2} K_3 [\sin(\omega+\phi_0) - \sin(\omega-\phi_0)] +
    \frac{1}{2} K_4 [\cos(\omega-\phi_0) - \cos(\omega+\phi_0)] \:
\Rightarrow \\
2 \rho &=& [K_1 - K_4] \cos(\omega+\phi_0) + [K_1+K_4] \cos(\omega-\phi_0) +
    [K_2+K_3] \sin(\omega+\phi_0)  \\
    && \quad \quad + [K_2-K_3] \sin(\omega-\phi_0). \\
\end{eqnarray}
Now set
\begin{eqnarray}
\label{OmegaMinusPhi}
A &=& \omega - \phi_0 \\
B &=& \omega + \phi_0 \label{OmegaPlusPhi}
\end{eqnarray}
so that the expression for the SNR becomes
\begin{equation}
2 \rho = (K_1+K_4) \cos A + (K_2-K_3) \sin A + (K_1-K_4) \cos B + (K_2+K_3)
\sin B.
\label{MaxSNR}
\end{equation}

To maximize with respect to $A$ we take the first derivative
\begin{equation}
\frac{\partial(2 \rho)}{\partial A} = - (K_1+K_4) \sin A
+(K_2-K_3) \cos A
\end{equation}
and set that equal to 0, which gives
\begin{equation}
\frac{\partial(2 \rho)}{\partial A} \Big |_{A_0} = 0 \: \Rightarrow
\: \tan A_0 =\frac{K_2-K_3}{K_1+K_4}. \\
\label{TANA}
\end{equation}
Then the sine and cosine of $A_0$ can be found:
\begin{eqnarray}
\sin A_0 &=& \pm \frac{\tan A_0}{\sqrt{1 + \tan^2 A_0}} = \pm
\frac{K_2-K_3}{\sqrt{(K_1+K_4)^2 +(K_2-K_3)^2}},  \label{SINA} \\
\cos A_0 &=& \pm \frac{1}{\sqrt{1 + \tan^2 A_0}} = \pm
\frac{K_1+K_4}{\sqrt{(K_1+K_4)^2 +(K_2-K_3)^2}}.
\label{COSA}
\end{eqnarray}
Notice that for Eq.~(\ref{TANA}) to be satisfied, the same sign must be kept
in Eqs (\ref{SINA}) and (\ref{COSA}).
To find the values that correspond to the maximum, we take the second
derivative of $\rho$ with respect to $A$:
\begin{equation}
\frac{\partial^2(2 \rho)}{\partial A^2}\Big |_{A_0} < 0 \: \Rightarrow
\: \Big [ - (K_1+K_4) \cos A - (K_2-K_3) \sin A \Big ]_{A_0} < 0
\end{equation}
which is satisfied if the $+$ sign is considered in Eqs (\ref{SINA}) and
(\ref{COSA}).

To maximize with respect to $B$ we take the first derivative
\begin{equation}
\frac{\partial(2 \rho)}{\partial B}  = - (K_1-K_4) \sin B
+(K_2+K_3) \cos B
\end{equation}
and set that equal to 0, which gives
\begin{equation}
\frac{\partial(2 \rho)}{\partial B} \Big |_{B_0} = 0 \: \Rightarrow
\: \tan B_0 =\frac{K_2+K_3}{K_1-K_4}. \\
\label{TANB}
\end{equation}
Then the sine and cosine of $B_0$ can be found:
\begin{eqnarray}
\sin B_0 &=& \pm \frac{\tan B_0}{\sqrt{1 + \tan^2 B_0}} = \pm
\frac{K_2+K_3}{\sqrt{(K_1-K_4)^2 +(K_2+K_3)^2}},  \label{SINB} \\
\cos B_0 &=& \pm \frac{1}{\sqrt{1 + \tan^2 B_0}} = \pm
\frac{K_1-K_4}{\sqrt{(K_1-K_4)^2 +(K_2+K_3)^2}}.
\label{COSB}
\end{eqnarray}
Again, the same sign must be kept in Eqs (\ref{SINB}) and (\ref{COSB}).
To find the values that correspond to the maximum, we take the second
derivative of $\rho$ with respect to $B$:
\begin{equation}
\frac{\partial^2(2 \rho)}{\partial B^2}\Big |_{B_0} < 0 \: \Rightarrow
\: \Big [ - (K_1-K_4) \cos B - (K_2+K_3) \sin B \Big ]_{B_0} < 0
\end{equation}
which is satisfied if the $+$ sign is considered in Eqs (\ref{SINB}) and
(\ref{COSB}).

Substituting the expressions for the sines and cosines of $A_0$ and
$B_0$
into Eq.~(\ref{MaxSNR}), the maximum SNR is:
\begin{equation}
 \rho_{max} = \frac{1}{2} \sqrt{(K_1+K_4)^2 + (K_2-K_3)^2} +
\frac{1}{2} \sqrt{(K_1-K_4)^2 +(K_2+K_3)^2}.
\end{equation}
To achieve a simpler form for the SNR, we can use Eqs (\ref{K1})-(\ref{K4}) to 
combine the integrals $K_1$, $K_2$, $K_3$ and
$K_4$.  
That gives
\begin{eqnarray}
K_1 + K_4 &=& - \Im \int_0^{\infty}  \frac{4 [b_1 A_1(f) + a_2 A_2(f)]}{S_h(f)}
\tilde{s}^{\ast}(f) e^{i \psi'(f)} df, \\
K_2 - K_3 &=& \Re \int_0^{\infty}  \frac{4[b_1 A_1(f) + a_2 A_2(f)]}{S_h(f)}
\tilde{s}^{\ast}(f) e^{i \psi'(f)} df,\\
K_1 - K_4 &=& - \Im \int_0^{\infty}  \frac{4[b_1 A_1(f) - a_2 A_2(f)]}{S_h(f)}
\tilde{s}^{\ast}(f) e^{i \psi'(f)} df, \\
K_2 + K_3 &=& - \Re \int_0^{\infty}  \frac{4[b_1 A_1(f) - a_2 A_2(f)]}{S_h(f)}
\tilde{s}^{\ast}(f) e^{i \psi'(f)} df.
\end{eqnarray}
Finally, the maximum SNR can be written as:
\begin{eqnarray}
\rho_{max} &=&  \Bigg | \int_0^{\infty} \frac{2 [b_1
A_1(f) + a_2 A_2(f)]}{S_h(f)} \tilde{s}^{\ast}(f) e^{i \psi'(f)} df \Bigg | \\
 && + \Bigg |
\int_0^{\infty} \frac{2[b_1 A_1(f) - a_2 A_2(f)]}{S_h(f)} \tilde{s}^{\ast}(f)
e^{i \psi'(f)} df \Bigg | \\
        &=& 2| F_1 -  F_2 | + 2| F_1 + F_2 |
\label{rhomax}
\end{eqnarray}
where $F_1$ and $F_2$ are the two integrals 
\begin{equation}
F_{1} = b_1 \int_0^{f_{cut}} \frac{ f^{-7/6}}
{S_h(f)} \tilde{s}^{\ast}(f) e^{i \psi'(f)} df , \quad
F_{2} = a_2 \int_0^{f_{cut}} \frac{ f^{-1/2}}
{S_h(f)} \tilde{s}^{\ast}(f) e^{i \psi'(f)} df .
\end{equation}

\subsubsection*{Final Form of the Template}
\label{FinalForm}

Combining the results given in the previous sections, the final form of the 
template is:
\begin{eqnarray}
\hat{h}(f) &=& \frac{1}{\sqrt{1+\alpha^2}} i b_1 A_1(f) e^{i \phi_0}
e^{i \psi'(f)} + \frac{\alpha}{\sqrt{1+\alpha^2}} a_2 A_2(f)
e^{i \phi_0} e^{i \psi'(f)} \\
    &=& \cos\omega (\cos\phi_0 + i \sin\phi_0)
i b_1 A_1(f) e^{i \psi'(f)}  + \sin\omega (\cos\phi_0
+ i \sin\phi_0)  a_2 A_2(f) e^{i \psi'(f)}  \\
    &=& \cos\omega \cos\phi_0 \big ( i b_1 A_1 e^{i \psi'} \big ) -
\cos\omega \sin\phi_0 \big(b_1 A_1 e^{i \psi'}\big ) + \sin\omega \cos\phi_0
\big ( a_2 A_2 e^{i \psi'} \big ) \\
    && + \sin\omega \sin\phi_0 (i a_2 A_2
e^{i \psi'} \big )\\
    &=& \Big [ \cos(\omega+\phi_0) + \cos(\omega-\phi_0) \Big ]
\frac{i b_1}{2} A_1 e^{i \psi'} - \Big [ \sin(\omega +\phi_0) -
\sin(\omega-\phi_0) \Big ] \frac{b_1}{2} A_1 e^{i \psi'}\\
    && +\Big[ \sin(\omega+\phi_0)+\sin(\omega-\phi_0)\Big]
\frac{a_2}{2} f^{2/3} A_1 e^{i \psi'} 
    + \Big[ \cos(\omega-\phi_0) -
\cos(\omega+\phi_0) \Big] \frac{i a_2}{2} f^{2/3} A_1 e^{i \psi'} \\
    &=&\big ( \cos B_0 + \cos A_0 \big )
\frac{i b_1}{2} A_1 e^{i \psi'} - \big ( \sin B_0 -
\sin A_0 \big ) \frac{b_1}{2} A_1 e^{i \psi'}\\
        && +\big( \sin B_0+\sin A_0\big)
\frac{a_2}{2} f^{2/3} A_1 e^{i \psi'}
        + \big( \cos A_0 -
\cos B_0\big) \frac{i a_2}{2} f^{2/3} A_1 e^{i \psi'} \: , \\
\end{eqnarray}
or, finally:
\begin{eqnarray}
\hat{h}(f) &=& f^{-7/6} \: \theta(f_{cut}-f)\: e^{i \psi'(f)} \Big \{ \: 
\frac{b_1}{2} \big [ i (\cos A_0
+ \cos B_0) + (\sin A_0 - \sin B_0) \big ]  \\
&& +f^{2/3} \frac{a_2}{2} \big[ i (\cos A_0 - \cos B_0 ) +
(\sin A_0 + \sin B_0) \big ] \Big \}.
\end{eqnarray}

\subsubsection*{The $\chi^2$-veto}
\label{ChisquaredVeto}

If we are working with $p$ bins the maximum SNR for the template must be 
divided into $p$ equal parts. In this case, since we have two different 
amplitude-parts of the template, we have to calculate two sets of bin
boundaries.
For the first set, the quantity
\begin{displaymath}
\int_0^{\infty} \frac{4 [b_1 A_1(f) - a_2 A_2(f) ]^2 }{S_h(f)} df =
\int_0^{f_{cut}} \frac{4 [b_1 f^{-7/6} + a_2 f^{-1/2}]^2}{S_h(f)} df
\end{displaymath}
must be divided into $p$ equal pieces.
For the second set, the quantity
\begin{displaymath}
\int_0^{\infty} \frac{4 [b_1 A_1(f) + a_2 A_2(f) ]^2 }{S_h(f)} df =
\int_0^{f_{cut}} \frac{4 [b_1 f^{-7/6} - a_2 f^{-1/2}]^2}{S_h(f)} df
\end{displaymath} 
must be divided into $p$ equal pieces.

To check if the total SNR is smoothly distributed over the bins,
take:
\begin{eqnarray}
\chi^2 &=& p \sum_{l=1}^p  \Big | \int_{f_l}^{f_{l+1}} \frac{2 [b_1
     A_1(f) - a_2 A_2(f)]}{S_h(f)} \tilde{s}^{\ast}(f) e^{i \psi'(f)} df
     - \frac{1}{p} \int_{0}^{\infty} \frac{2 [b_1A_1(f) - a_2 A_2(f)]}
{S_h(f)} \tilde{s}^{\ast}(f) e^{i \psi'(f)} df \Big |^2 \\
	 &&+ p\sum_{l=1}^p\Big |\int_{f_l}^{f_{l+1}} \frac{2 [b_1 A_1(f) + a_2
     A_2(f)]}{S_h(f)}\tilde{s}^{\ast}(f)e^{i \psi'(f)} df-\frac{1}{p} 
     \int_{0}^{\infty} \frac{2[b_1 A_1(f) + a_2 A_2(f)]}{S_h(f)}
\tilde{s}^{\ast}(f)e^{i \psi'(f)} df\Big |^2 \\
\end{eqnarray}
or
\begin{equation}
\chi^2 =  p \sum_{l=1}^p  \Big |2 (F_1^l+F_2^l) 
     - \frac{2}{p} (F_1+F_2) \Big |^2 + p \sum_{l=1}^p  \Big |2 (F_1^l-F_2^l)
     - \frac{2}{p} (F_1-F_2) \Big |^2.
\end{equation}

%\bibliographystyle{unsrt}
%\bibliography{References}

