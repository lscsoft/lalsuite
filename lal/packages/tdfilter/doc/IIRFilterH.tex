
\section{Header \texttt{IIRFilter.h}}

Provides routines to make and apply IIR filters.

\subsection{Synopsis}
\begin{verbatim}
#include "IIRFilter.h"
\end{verbatim}

\noindent This header covers routines that create, destroy, and apply
generic time-domain filters, given by objects of type
\verb@<datatype>IIRFilter@, where \verb@<datatype>@ is either
\verb@REAL4@ or \verb@REAL8@.

An IIR (Infinite Impulse Response) filter is a generalized linear
causal time-domain filter, in which the filter output $y_n=y(t_n)$ at
any sampled time $t_n=t_0+n\Delta t$ is a linear combination of the
input $x$ \emph{and} output $y$ at previous sampled times:
$$
y_n = \sum_{k=0}^M c_k x_{n-k} + \sum_{l=1}^N d_l y_{n-l} \; .
$$
The coefficients $c_k$ are called the direct filter coefficients, and
the coefficients $d_l$ are the recursive filter coefficients.  The
filter order is the larger of $M$ or $N$, and determines how far back
in time the filter must look to determine its next output.  However,
the recursive nature of the filter means that the output can depend on
input arbitrarily far in the past; hence the name ``infinite impulse
response''.  Nonetheless, for a well-designed, stable filter, the
actual filter response to an impulse should diminish rapidly beyond
some characteristic timescale.

Note that nonrecursive FIR (Finite Impulse Response) filters are
considered a subset of IIR filters, having $N=0$.

For practical implementation, it is convenient to express the bilinear
equation above as two linear equations involving an auxiliary sequence
$w$:
$$
w_n = x_n + \sum_{l=1}^N d_l w_{n-l} \; ,
$$
$$
y_n = \sum_{k=0}^M c_k w_{n-k} \; .
$$
The equivalence of this to the first expression is not obvious, but
can be proven by mathematical induction.  The advantage of the
auxiliary variable representation is twofold.  First, when one is
feeding data point by point to the filter, the filter needs only
``remember'' the previous $M$ or $N$ (whichever is larger) values of
$w$, rather than remembering the previous $M$ values of $x$ \emph{and}
the previous $N$ values of $y$.  Second, when filtering a large stored
data vector, the filter response can be computed in place: one first
runs forward through the vector replacing $x$ with $w$, and then
backward replacing $w$ with $y$.

Although the IIR filters in these routines are explicitly real, one
can consider formally their complex response.  A sinusoidal input can
thus be written as $x_n=X\exp(2\pi ifn\Delta t)=z^n$, where $X$ is a
complex amplitude and $z=\exp(2\pi if\Delta t)$ is a complex
parametrization of the frequency.  By linearity, the output must also
be sinusoidal: $y_m=Y\exp(2\pi ifm\Delta t)=z^m$.  Putting these into
the bilinear equation, one can easily compute the filter's complex
transfer function:
$$
T(z) = \frac{Y}{X} = \frac{\sum_{k=0}^M c_k z^{-k}}
                      {1 - \sum_{l=1}^N d_l z^{-l}}
$$
This can be readily converted to and from the ``zeros, poles, gain''
representation of a filter, which expresses $T(z)$ as a factored
rational function of $z$.

It should also be noted that, in the routines covered by this header,
I have adopted the convention of including a redundant recursive
coefficient $d_0$, in order to make the indexing more intuitive.  For
formal correctness $d_0$ should be set to $-1$, although the filtering
routines never actually use this coefficient.


\subsection{Error conditions}
\begin{tabular}{|c|l|l|}
\hline
status & status      & Explanation \\
 code  & description & \\
\hline
\tt 1  & \tt Null pointer            & Missing a required pointer.           \\
\tt 2  & \tt Output already exists   & Can't allocate to a non-null pointer. \\
\tt 3  & \tt Memory allocation error & Could not allocate memory.            \\
\tt 4  & \tt Input has unpaired      & For real filters, complex poles or    \\
       & \tt nonreal poles or zeros  & zeros must come in conjugate pairs.   \\
\hline
\end{tabular}


\subsection{Structures}
\begin{verbatim}
<datatype>IIRFilter
\end{verbatim}

\noindent This structure stores the direct and recursive filter
coefficients, as well as the history of the auxiliary sequence $w$.
The length of the history vector gives the order of the filter.  The
fields are:

\begin{description}
\item[\texttt{CHAR *name}] A user-assigned name.

\item[\texttt{<datatype>Vector *directCoef}] The direct filter
  coefficients.

\item[\texttt{<datatype>Vector *recursCoef}] The recursive filter
  coefficients.

\item[\texttt{<datatype>Vector *history}] The previous values of $w$.
\end{description}

\newpage
\subsection{Module \texttt{CreateIIRFilter.c}}

Creates IIR filter objects.

\subsubsection{Prototypes}
\vspace{0.1in}
\marginpar{\texttt{\tiny l.127}\\{\tiny CreateIIRFilter.c}}
\vspace{-0.1in}
\begin{verbatim}
void CreateREAL4IIRFilter(Status            *stat,
			  REAL4IIRFilter    **output,
			  COMPLEX8ZPGFilter *input)
\end{verbatim}
\marginpar{\texttt{\tiny l.368}\\{\tiny CreateIIRFilter.c}}
\vspace{-0.1in}
\begin{verbatim}
void CreateREAL8IIRFilter(Status             *stat,
			  REAL8IIRFilter     **output,
			  COMPLEX16ZPGFilter *input)
\end{verbatim}


\subsubsection{Description}

These functions create an object \verb@**output@ of type
\verb@<datatype>IIRFilter@, where \verb@<datatype>@ is \verb@REAL4@ or
\verb@REAL8@.  The filter coefficients are computed from the zeroes,
poles, and gain of an input object \verb@*input@ of type
\verb@COMPLEX8ZPGFilter@ or \verb@COMPLEX16ZPGFilter@, respectively.
The ZPG filter should express the factored transfer function in the
$z=\exp(2\pi if)$ plane.  Initially the output handle must be a valid
handle (\verb@output@$\neq$\verb@NULL@) but should not point to an
existing object (\verb@*output@=\verb@NULL@)

\subsubsection{Algorithm}

An IIR filter is a real time-domain filter, which imposes certain
constraints on the zeros, poles, and gain of the filter transfer
function.  The function \verb@Create<datatype>IIRFilter()@ deals with
the constraints either by aborting if they are not met, or by
adjusting the filter response so that they are met.  In the latter
case, warning messages will be issued if the external parameter
\verb@debuglevel@ is 1 or more.  The specific constraints, and how
they are dealt with, are as follows:

First, the filter must be \emph{causal}; that is, the output at any
time can be generated entirely from the input at previous times.  In
practice this means that the number of (finite) poles in the $z$ plane
must equal or exceed the number of (finite) zeros.  If this is not the
case, \verb@Create<datatype>IIRFilter()@ will add additional poles at
$z=0$.  Effectively this is just adding a delay to the filter response
in order to make it causal.

Second, the filter should be \emph{stable}, which means that all poles
should be located on or within the circle $|z|=1$.  This is not
enforced by \verb@Create<datatype>IIRFilter()@, which can be used to
make unstable filters; however, warnings will be issued if
\verb@debuglevel@ is 1 or more.  (In some sense the first condition is
a special case of this one, since a transfer function with more zeros
than poles actually has corresponding poles at infinity.)

Third, the filter must be \emph{physically realizable}; that is, the
transfer function should expand to a rational function of $z$ with
real coefficients.  Necessary and sufficient conditions for this are
that the gain be real, and that all zeros and poles either be real or
come in complex conjugate pairs.  The routine
\verb@Create<datatype>IIRFilter()@ enforces this by using only the
real part of the gain, and only the real or positive-imaginary zeros
and poles; it assumes that the latter are paired with
negative-imaginary conjugates.  The routine will abort if this
assumption results in a change in the given number of zeros or poles,
but will otherwise simply modify the filter response.  This allows
\verb@debuglevel@=0 runs to proceed without lengthy and usually
unnecessary error trapping; when \verb@debuglevel@ is 1 or more, the
routine checks to make sure that each nonreal zero or pole does in
fact have a complex-conjugate partner.

\subsubsection{Uses}
\begin{verbatim}
debuglevel
LALMalloc()
SCreateVector()
DCreateVector()
\end{verbatim}

\subsubsection{Notes}


\newpage
\subsection{Module \texttt{DestroyIIRFilter.c}}

Destroys IIR filter objects.

\subsubsection{Prototypes}
\vspace{0.1in}
\marginpar{\texttt{\tiny l.64}\\{\tiny DestroyIIRFilter.c}}
\vspace{-0.1in}
\begin{verbatim}
void DestroyREAL4IIRFilter(Status         *stat,
			   REAL4IIRFilter **input)
\end{verbatim}
\marginpar{\texttt{\tiny l.92}\\{\tiny DestroyIIRFilter.c}}
\vspace{-0.1in}
\begin{verbatim}
void DestroyREAL8IIRFilter(Status         *stat,
			   REAL8IIRFilter **input)
\end{verbatim}


\subsubsection{Description}

These functions destroy an object \verb@**input@ of type
\texttt{REAL4IIRFilter} or \texttt{REAL8IIRFilter}, and set
\verb@*input@ to \verb@NULL@.

\subsubsection{Algorithm}

\subsubsection{Uses}
\begin{verbatim}
void LALFree()
void SDestroyVector()
void DDestroyVector()
\end{verbatim}

\subsubsection{Notes}


\newpage
\subsection{Module \texttt{IIRFilter.c}}

Computes an instant-by-instant IIR filter response.

\subsubsection{Prototypes}
\vspace{0.1in}
\marginpar{\texttt{\tiny l.64}\\{\tiny IIRFilter.c}}
\vspace{-0.1in}
\begin{verbatim}
void IIRFilterREAL4(Status         *stat,
		    REAL4          *output,
		    REAL4          input,
		    REAL4IIRFilter *filter)
\end{verbatim}
\marginpar{\texttt{\tiny l.117}\\{\tiny IIRFilter.c}}
\vspace{-0.1in}
\begin{verbatim}
void IIRFilterREAL8(Status         *stat,
		    REAL8          *output,
		    REAL8          input,
		    REAL8IIRFilter *filter)
\end{verbatim}
\marginpar{\texttt{\tiny l.170}\\{\tiny IIRFilter.c}}
\vspace{-0.1in}
\begin{verbatim}
REAL4 SIIRFilter(REAL4          x,
		 REAL4IIRFilter *filter)
\end{verbatim}
\marginpar{\texttt{\tiny l.210}\\{\tiny IIRFilter.c}}
\vspace{-0.1in}
\begin{verbatim}
REAL8 DIIRFilter(REAL8          x,
		 REAL8IIRFilter *filter)
\end{verbatim}


\subsubsection{Description}

These functions pass a time-domain datum to an object \verb@*filter@
of type \verb@REAL4IIRFilter@ or \verb@REAL8IIRFilter@, and return the
filter response.  This is done using the auxiliary data series
formalism described in the header \verb@IIRFilter.h@.

There are two pairs of routines in this module.  The functions
\verb@IIRFilterReal4()@ and \verb@IIRFilterREAL8()@ conform to the LAL
standard, with status handling and error trapping; the input datum is
passed in as \verb@input@ and the response is returned in
\verb@*output@.  The functions \verb@SIIRFilter()@ and
\verb@DIIRFilter()@ are non-standard lightweight routines, which may
be more suitable for multiple callings from the inner loops of
programs; they have no status handling or error trapping.  The input
datum is passed in by the variable \verb@x@, and the response is
returned through the function's return statement.

\subsubsection{Algorithm}

\subsubsection{Uses}

\subsubsection{Notes}


\newpage
\subsection{Module \texttt{IIRFilterVector.c}}

Applies an IIR filter to a data stream.

\subsubsection{Prototypes}
\vspace{0.1in}
\marginpar{\texttt{\tiny l.61}\\{\tiny IIRFilterVector.c}}
\vspace{-0.1in}
\begin{verbatim}
void IIRFilterREAL4Vector(Status         *stat,
			  REAL4Vector    *vector,
			  REAL4IIRFilter *filter)
\end{verbatim}
\marginpar{\texttt{\tiny l.147}\\{\tiny IIRFilterVector.c}}
\vspace{-0.1in}
\begin{verbatim}
void IIRFilterREAL8Vector(Status         *stat,
			  REAL8Vector    *vector,
			  REAL8IIRFilter *filter)
\end{verbatim}


\subsubsection{Description}

These functions apply a generic time-domain filter given by an object
\verb@*filter@ of type \verb@REAL4IIRFilter@ or \verb@REAL8IIRFilter@
to a list \verb@*vector@ of data representing a time series.  This is
done in place using the auxiliary data series formalism described in
\verb@IIRFilter.h@, so as to accomodate potentially large data series.
To filter a piece of a larger dataset, the calling routine may pass a
vector structure whose data pointer and length fields specify a subset
of a larger vector.

\subsubsection{Algorithm}

\subsubsection{Uses}
\begin{verbatim}
LALMalloc()
LALFree()
\end{verbatim}

\subsubsection{Notes}


\newpage
\subsection{Module \texttt{IIRFilterVectorR.c}}

Applies a time-reversed IIR filter to a data stream.

\subsubsection{Prototypes}
\vspace{0.1in}
\marginpar{\texttt{\tiny l.61}\\{\tiny IIRFilterVectorR.c}}
\vspace{-0.1in}
\begin{verbatim}
void IIRFilterREAL4VectorR(Status         *stat,
			   REAL4Vector    *vector,
			   REAL4IIRFilter *filter)
\end{verbatim}
\marginpar{\texttt{\tiny l.120}\\{\tiny IIRFilterVectorR.c}}
\vspace{-0.1in}
\begin{verbatim}
void IIRFilterREAL8VectorR(Status         *stat,
			   REAL8Vector    *vector,
			   REAL8IIRFilter *filter)
\end{verbatim}


\subsubsection{Description}

These functions apply a generic time-domain filter \verb@*filter@ to a
time series \verb@*vector@, as with the routines
\verb@IIRFilterREAL4Vector()@ and \verb@IIRFilterREAL8Vector()@, but
do so in a time-reversed manner.  By successively applying normal and
time-reversed IIR filters to the same data, one squares the magnitude
of the frequency response while canceling the phase shift.  This can
be significant when one wishes to preserve phase correlations across
wide frequency bands.

\subsubsection{Algorithm}

Because these filter routines are inherently acausal, the
\verb@filter->history@ vector is meaningless and unnecessary.  These
routines neither use nor modify this data array.

\subsubsection{Uses}

\subsubsection{Notes}


