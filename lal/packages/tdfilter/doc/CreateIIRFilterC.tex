
\subsection{Module \texttt{CreateIIRFilter.c}}

Creates IIR filter objects.

\subsubsection{Prototypes}
\vspace{0.1in}
\marginpar{\texttt{\tiny l.127}\\{\tiny CreateIIRFilter.c}}
\vspace{-0.1in}
\begin{verbatim}
void CreateREAL4IIRFilter(Status            *stat,
			  REAL4IIRFilter    **output,
			  COMPLEX8ZPGFilter *input)
\end{verbatim}
\marginpar{\texttt{\tiny l.368}\\{\tiny CreateIIRFilter.c}}
\vspace{-0.1in}
\begin{verbatim}
void CreateREAL8IIRFilter(Status             *stat,
			  REAL8IIRFilter     **output,
			  COMPLEX16ZPGFilter *input)
\end{verbatim}


\subsubsection{Description}

These functions create an object \verb@**output@ of type
\verb@<datatype>IIRFilter@, where \verb@<datatype>@ is \verb@REAL4@ or
\verb@REAL8@.  The filter coefficients are computed from the zeroes,
poles, and gain of an input object \verb@*input@ of type
\verb@COMPLEX8ZPGFilter@ or \verb@COMPLEX16ZPGFilter@, respectively.
The ZPG filter should express the factored transfer function in the
$z=\exp(2\pi if)$ plane.  Initially the output handle must be a valid
handle (\verb@output@$\neq$\verb@NULL@) but should not point to an
existing object (\verb@*output@=\verb@NULL@)

\subsubsection{Algorithm}

An IIR filter is a real time-domain filter, which imposes certain
constraints on the zeros, poles, and gain of the filter transfer
function.  The function \verb@Create<datatype>IIRFilter()@ deals with
the constraints either by aborting if they are not met, or by
adjusting the filter response so that they are met.  In the latter
case, warning messages will be issued if the external parameter
\verb@debuglevel@ is 1 or more.  The specific constraints, and how
they are dealt with, are as follows:

First, the filter must be \emph{causal}; that is, the output at any
time can be generated entirely from the input at previous times.  In
practice this means that the number of (finite) poles in the $z$ plane
must equal or exceed the number of (finite) zeros.  If this is not the
case, \verb@Create<datatype>IIRFilter()@ will add additional poles at
$z=0$.  Effectively this is just adding a delay to the filter response
in order to make it causal.

Second, the filter should be \emph{stable}, which means that all poles
should be located on or within the circle $|z|=1$.  This is not
enforced by \verb@Create<datatype>IIRFilter()@, which can be used to
make unstable filters; however, warnings will be issued if
\verb@debuglevel@ is 1 or more.  (In some sense the first condition is
a special case of this one, since a transfer function with more zeros
than poles actually has corresponding poles at infinity.)

Third, the filter must be \emph{physically realizable}; that is, the
transfer function should expand to a rational function of $z$ with
real coefficients.  Necessary and sufficient conditions for this are
that the gain be real, and that all zeros and poles either be real or
come in complex conjugate pairs.  The routine
\verb@Create<datatype>IIRFilter()@ enforces this by using only the
real part of the gain, and only the real or positive-imaginary zeros
and poles; it assumes that the latter are paired with
negative-imaginary conjugates.  The routine will abort if this
assumption results in a change in the given number of zeros or poles,
but will otherwise simply modify the filter response.  This allows
\verb@debuglevel@=0 runs to proceed without lengthy and usually
unnecessary error trapping; when \verb@debuglevel@ is 1 or more, the
routine checks to make sure that each nonreal zero or pole does in
fact have a complex-conjugate partner.

\subsubsection{Uses}
\begin{verbatim}
debuglevel
LALMalloc()
SCreateVector()
DCreateVector()
\end{verbatim}

\subsubsection{Notes}

