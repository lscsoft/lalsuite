
\section{Header \texttt{ZPGFilter.h}}

Provides routines to manipulate ZPG filters.

\subsection{Synopsis}
\begin{verbatim}
#include "ZPGFilter.h"
\end{verbatim}

\noindent This header covers routines that create, destroy, and
transform objects of type \verb@<datatype>ZPGFilter@, where
\verb@<datatype>@ is either \verb@COMPLEX8@ or \verb@COMPLEX16@.
Generically, these data types can be used to store any rational
complex function in a factored form.  Normally this function is a
filter response, or ``transfer function'' $T(z)$, expressed in terms
of a complex frequency parameter $z=\exp(2\pi if\Delta t)$, where
$\Delta t$ is the sampling interval.  The rational function is
factored as follows:
$$
T(f) = g\times\frac{\prod_k (z-a_k)}{\prod_l (z-b_l)}
$$
where $g$ is the gain, $a_k$ are the (finite) zeros, and $b_l$ are the
(finite) poles.  It should be noted that rational functions always
have the same number of zeros as poles if one includes the point
$z=\infty$; any excess in the number of finite zeros or poles in the
rational expression simply indicates that there is a corresponding
pole or zero of that order at infinity.  It is also worth pointing out
that the ``gain'' is just the overall prefactor of this rational
function, and is not necessarily equal to the actual gain of the
transfer function at any particular frequency.

Another common complex frequency space is the $w$-space, obtained
from the $z$-space by the bilinear transformation:
$$
w = i\left(\frac{1-z}{1+z}\right) = \tan(\pi f\Delta t) , \quad
z = \frac{1+iw}{1-iw} \; .
$$
Other variables can also be used to represent the complex frequency
plane.  The \verb@<datatype>ZPGFilter@ structure can be used to
represent the transfer function in any of these spaces by transforming
the coordinates of the zeros and poles, and incorporating any residual
factors into the gain.  Care must be taken to include any zeros or
poles that are brought in from infinity by the transformation, and to
remove any zeros or poles which were sent to infinity.  Thus the
number of zeros and poles of the \verb@<datatype>ZPGFilter@ is not
necessarily constant under transformations!  Routines invoking the
\verb@<datatype>ZPGFilter@ data types should document which complex
variable is assumed.


\subsection{Error conditions}
\begin{tabular}{|c|l|l|}
\hline
status & status      & Explanation \\
 code  & description &             \\
\hline
\tt 1  & \tt Null pointer            & Missing a required pointer.           \\
\tt 2  & \tt Output already exists   & Can't allocate to a non-null pointer. \\
\tt 3  & \tt Memory allocation error & Could not allocate memory.            \\
\tt 4  & \tt Bad filter parameters   & Filter creation parameters outside of \\
       &                             & acceptable ranges.                    \\
\hline
\end{tabular}

\subsection{Structures}
\newpage
\subsection{Module \texttt{CreateZPGFilter.c}}

Creates ZPG filter objects.

\subsubsection{Prototypes}
\vspace{0.1in}
\marginpar{\texttt{\tiny l.68}\\{\tiny CreateZPGFilter.c}}
\vspace{-0.1in}
\begin{verbatim}
void CreateCOMPLEX8ZPGFilter(Status            *stat,
			     COMPLEX8ZPGFilter **output,
			     INT4              numZeros,
			     INT4              numPoles)
\end{verbatim}
\marginpar{\texttt{\tiny l.106}\\{\tiny CreateZPGFilter.c}}
\vspace{-0.1in}
\begin{verbatim}
void CreateCOMPLEX16ZPGFilter(Status             *stat,
			      COMPLEX16ZPGFilter **output,
			      INT4               numZeros,
			      INT4               numPoles)
\end{verbatim}


\subsubsection{Description}

These functions create an object \verb@**output@, of type
\verb@COMPLEX8ZPGFilter@ or \verb@COMPLEX16ZPGFilter@, having
\verb@numZeros@ zeros and \verb@numPoles@ poles.  The values of those
zeros and poles are not set by these routines (in general they will
start out as garbage).  The handle passed into the functions must be a
valid handle (i.e.\ \verb@output@$\neq$\verb@NULL@), but must not
point to an existing object (\i.e.\ \verb@*output@=\verb@NULL@).

\subsubsection{Algorithm}

\subsubsection{Uses}
\begin{verbatim}
LALMalloc()
CCreateVector()
ZCreateVector()
\end{verbatim}

\subsubsection{Notes}


\newpage
\subsection{Module \texttt{DestroyZPGFilter.c}}

Destroys ZPG filter objects.

\subsubsection{Prototypes}
\vspace{0.1in}
\marginpar{\texttt{\tiny l.63}\\{\tiny DestroyZPGFilter.c}}
\vspace{-0.1in}
\begin{verbatim}
void DestroyCOMPLEX8ZPGFilter(Status            *stat,
			      COMPLEX8ZPGFilter **input)
\end{verbatim}
\marginpar{\texttt{\tiny l.90}\\{\tiny DestroyZPGFilter.c}}
\vspace{-0.1in}
\begin{verbatim}
void DestroyCOMPLEX16ZPGFilter(Status             *stat,
			       COMPLEX16ZPGFilter **input)
\end{verbatim}


\subsubsection{Description}

These functions destroy an object \verb@**output@ of type
\verb@COMPLEX8ZPGFilter@ or \verb@COMPLEX16ZPGFilter@, and set
\verb@*output@ to \verb@NULL@.

\subsubsection{Algorithm}

\subsubsection{Uses}
\begin{verbatim}
LALFree()
CDestroyVector()
ZDestroyVector()
\end{verbatim}

\subsubsection{Notes}


\newpage
\subsection{Module \texttt{BilinearTransform.c}}

Transforms the complex frequency coordinate of a ZPG filter.

\subsubsection{Prototypes}
\vspace{0.1in}
\marginpar{\texttt{\tiny l.151}\\{\tiny BilinearTransform.c}}
\vspace{-0.1in}
\begin{verbatim}
void WToZCOMPLEX8ZPGFilter(Status            *stat,
			   COMPLEX8ZPGFilter *filter)
\end{verbatim}
\marginpar{\texttt{\tiny l.386}\\{\tiny BilinearTransform.c}}
\vspace{-0.1in}
\begin{verbatim}
void WToZCOMPLEX16ZPGFilter(Status             *stat,
			    COMPLEX16ZPGFilter *filter)
\end{verbatim}


\subsubsection{Description}

These functions perform an in-place bilinear transformation on an
object \verb@*filter@ of type \verb@<datatype>ZPGFilter@, transforming
from $w$ to $z=(1+iw)/(1-iw)$.  Care is taken to ensure that zeros and
poles at $w=\infty$ are correctly transformed to $z=-1$, and zeros and
poles at $w=-i$ are correctly transformed to $z=\infty$.  In addition
to simply relocating the zeros and poles, residual factors are also
incorporated into the gain of the filter (i.e.\ the leading
coefficient of the rational function).

\subsubsection{Algorithm}

The vectors \verb@filter->zeros@ and \verb@filter->poles@ only record
those zeros and poles that have finite value.  If one includes the
point $\infty$ on the complex plane, then a rational function always
has the same number of zeros and poles: a number \verb@num@ that is
the larger of \verb@z->zeros->length@ or \verb@z->poles->length@.  If
one or the other vector has a smaller length, then after the
transformation that vector will receive additional elements, with a
complex value of $z=-1$, to bring its length up to \verb@num@.
However, each vector will then \emph{lose} those elements that
previously had values $w=-i$, (which are sent to $z=\infty$,) thus
possibly decreasing the length of the vector.  These routines handle
this by simply allocating a new vector for the transformed data, and
freeing the old vector after the transformation.

When transforming a zero $w_k$ on the complex plane, one makes use of
the identity:
$$
(w - w_k) = -(w_k + i)\times\frac{z-z_k}{z+1} \; ,
$$
and similarly, when transforming a pole at $w_k$,
$$
(w - w_k)^{-1} = -(w_k + i)^{-1}\times\frac{z+1}{z-z_k} \; ,
$$
where $z=(1+iw)/(1-iw)$ and $z_k=(1+iw_k)/(1-iw_k)$.  If there are an
equal number of poles and zeros being transformed, then the factors of
$z+1$ will cancel; otherwise, the remaining factors correspond to the
zeros or poles at $z=-1$ brought in from $w=\infty$.  The factor
$(z-z_k)$ represents the transformation of the zero or pole at $w_k$.
The important factor to note, though, is the factor $-(w_k+i)^{\pm1}$.
This factor represents the change in the gain \verb@filter->gain@.
When $w_k=-i$, the transformation is slightly different:
$$
(w + i) = \frac{2i}{z+1} \; ;
$$
thus the gain correction factor is $2i$ (rather than 0) in this case.

The algorithms in this module computes and stores all the gain
correction factors before applying them to the gain.  The correction
factors are sorted in order of absolute magnitude, and are multiplied
together in small- and large-magnitude pairs.  In this way one reduces
the risk of overrunning the floating-point dynamical range during
intermediate calculations.

As a similar precaution, the routines in this module use the algorithm
discussed in the \verb@VectorOps@ package whenever they perform
complex division, to avoid intermediate results that mey be the
product of two large numbers.  When transforming $z=(1+iw)/(1-iw)$,
these routines also test for special cases (such as $w$ purely
imaginary) that have qualitatively significant results ($z$ purely
real), so that one doesn't end up with, for instance, an imaginary
part of $10^{-12}$ instead of 0.

\subsubsection{Uses}
\begin{verbatim}
I4CreateVector()
SCreateVector()
DCreateVector()
CCreateVector()
ZCreateVector()
I4DestroyVector()
SDestroyVector()
DDestroyVector()
CDestroyVector()
ZDestroyVector()
SHeapIndex()
DHeapIndex()
\end{verbatim}

\subsubsection{Notes}


