\documentclass[12pt]{article}
\usepackage{amsmath}

\begin{document}
\huge
\begin{center}
MakeTemplate.c
\end{center}
\normalsize
\vspace{10mm}

\section{Purpose}

The code \texttt{MakeTemplate.c} takes from the user the parameters which specify the waveform they would like to generate, and it calls the appropriate waveform generation function which will actually calculate the waveform.


\section{Algorithms}

The code uses no algorithms


\section{Arguments}

The function header is of the form:

\vspace{5mm}

\begin{tabular}{ll}
void \texttt{MakeTemplate}&(\texttt{Status $\ast$status},     \\
                                   &\texttt{REAL8Vector $\ast$waveform}, \\
                                   &\texttt{InspiralTemplate $\ast$params})
\end{tabular}

\vspace{5mm}

The structure which is of type \texttt{Status}, which is pointed to by the pointer \texttt{status} writes information to the screen should the code encounter
a problem. The output structure is of the form \texttt{REAL8Vector} and is pointed to by the pointer \texttt{waveform}. This is the required waveform.
The inputs needed come from the input structure which is of type \texttt{InspiralTemplate}, and which is pointed to by the pointer \texttt{params}.

The output structure has the form

\vspace{5mm}

\begin{tabular}{ll}
\texttt{typedef struct} & \texttt{tagREAL8Vector} \{ \\
                        & \texttt{INT4 length;} \\
                        & \texttt{REAL8 $\ast$data;}  \\
                        & \} \texttt{REAL8Vector;}
\end{tabular}

\vspace{5mm}

\vspace{5mm}


The input structure is of the form

\vspace{5mm}

\begin{tabular}{ll}
\texttt{typedef struct} & \texttt{tagInspiralTemplate} \{ \\
                        & \texttt{INT4 number;} \\
                        & \texttt{REAL4 mass1;} \\
                        & \texttt{REAL4 mass2;}  \\
                        & \texttt{REAL4 spin1[3];}  \\
                        & \texttt{REAL4 spin2[3];}  \\
                        & \texttt{REAL4 inclination;} \\
                        & \texttt{REAL4 eccentricity;} \\
                        & \texttt{REAL4 totalMass;} \\
                        & \texttt{REAL4 mu;}  \\
                        & \texttt{REAL4 eta;}  \\
                        & \texttt{REAL4 fLower;}  \\
                        & \texttt{REAL4 fCutoff;} \\
                        & \texttt{REAL4 tSampling;} \\
                        & \texttt{REAL4 phaseShift;} \\
                        & \texttt{REAL4 nStartPad;} \\
                        & \texttt{REAL4 nEndPad;} \\
                        & \texttt{InputMasses massChoice;}  \\
                        & \texttt{InspiralMethod method;}  \\
                        & \} \texttt{InspiralTemplate;}
\end{tabular}

\vspace{5mm}




The parameters which are represented by these input are as follows: \texttt{number} is a label for each template, \texttt{mass1} and \texttt{mass2} are the masses of the compact objects in solar masses, \texttt{spin1} and \texttt{spin2} are the spins of the objects, \texttt{inclination} is the angle of inclination which the binary system makes to the observer, \texttt{eccentricity} is the eccentricity of the objects' orbit, \texttt{totalMass} is their combined mass $m=m_{1}+m_{2}$, \texttt{mu} is the reduced mass $\mu=m_{1}m_{2}/(m_{1}+m_{2})$, \texttt{eta} is the symmetric mass ratio $\eta=m_{1}m_{2}/(m_{1}+m_{2})^{2}$, \texttt{fLower} is the frequency at which the detectors' noise curve rises steeply (the seismic limit), \texttt{fCutoff} is the frequency at which the user can choose to terminate the waveform, \texttt{tSampling} is the time interval between samples in units of seconds, \texttt{phaseShift} is the initial phase given to the signal, \texttt{nStartPad} is the number of zeros which are added at the start of the waveform and \texttt{nEndPad} is the number of zeros which are added at the end of the waveform.
The parameter \texttt{MassChoice} is of type \texttt{enum InputMasses}, which determines which pair of input masses the user has defined. This typedef is as follows:

\vspace{5mm}

\begin{tabular}{ll}
\texttt{typedef enum} & \{ \\
                      & \texttt{m1Andm2,} \\
                      & \texttt{totalMassAndEta,}  \\
                      & \texttt{totalMassAndMu} \\
                      & \} \texttt{InputMasses;}
\end{tabular}

\vspace{5mm}

The parameter \texttt{method} is of type \texttt{enum InspiralMethod}, which tells the function which code is to be used to generate the waveform. This typedef is (at the moment) as follows:

\vspace{5mm}

\begin{tabular}{ll}
\texttt{typedef enum} & \{ \\
                      & \texttt{Best,} \\
                      & \texttt{TappRpnTdomFreq20,}  \\
                      & \texttt{TappRpnTdomTime20,} \\
                      & \} \texttt{InspiralMethod;}
\end{tabular}

\vspace{5mm}
More choices will be added as more codes are written.



\section{Operating Instructions}

When this function is called each of the following input structure members must be defined (although the last three may be set to zero):
\texttt{fLower}, \texttt{fCutoff}, \texttt{tSampling}, \texttt{phaseShift}, \texttt{nStartPad} and \texttt{nEndPad}
From the following five input members (\texttt{m1,m2,totalMass,mu,eta}), it is not necessary to define all five. Instead, only  one of the following pairs needs to be defined: (\texttt{m1,m2}), (\texttt{totalMass,mu}), or (\texttt{totalMass,eta}). The pair which the user decides to use is as determined by choosing one of the following inputs \\ \texttt{params.massChoice=m1Andm2}, \\ \texttt{params.massChoice=totalMassAndMu} or \\ \texttt{params.massChoice=totalMassAndEta}.

The choice of code which the user would like to use to calculate the waveform is made by choosing one of the following inputs:\\
\texttt{params.method=Best}, \\
\texttt{params.method=TaylorTime20}, \\
\texttt{params.method=TaylorFreq20}, \\
\texttt{params.method=Papprox20}


Here is an example of a code fragment which shows how the members of the input structure (which is pointed to by the pointer \texttt{templateIn}) are initialized, and how the function is then called. In this example, we have chosen to define the pair \texttt{mass1} and \texttt{mass2} as inputs, and we generate the waveform using Taylor approximants, up to second post--Newtonian order, in the time domain with the phase being calculated as a function of frequency.


\vspace{5mm}

\noindent
\begin{verbatim}
/* Declare the structures to be used  */
\end{verbatim}
\texttt{InspiralTemplate templateIn;} \\
\texttt{REAL8Vector $\ast$waveform;} \\
\texttt{Status status;} \\
\begin{verbatim}
/* Allocate a pointer for the output structure  */

waveform=(REAL8Vector *)LALMalloc(sizeof(REAL8Vector));

\end{verbatim}
\begin{verbatim}
/* Initialize the inputs  */
\end{verbatim}
\texttt{templateIn.number} = 1;\\
\texttt{templateIn.mass1} = 10.0; \\
\texttt{templateIn.mass2} = 10.0; \\
\texttt{templateIn.fLower} = 40.0; \\
\texttt{templateIn.fCutoff} = 1000.0; \\
\texttt{templateIn.tSampling} = 1.0/4000.0; \\
\texttt{templateIn.phaseShift} = 0.0; \\
\texttt{templateIn.nStartPad} = 100; \\
\texttt{templateIn.nEndPad} = 0; \\
\texttt{templateIn.massChoice} = \texttt{m1Andm2} \\
\texttt{templateIn.method} = \texttt{TaylorFreq20} \\

\begin{verbatim}
/* Call the function  */
\end{verbatim}
\texttt{MakeTemplate (\&status, waveform, \&templateIn);}

\vspace{5mm}

Now the waveform will be pointed to by the pointer \texttt{waveform->data}.

Inside the function \texttt{MakeTemplate()}, error checks are made upon its arguments, using the ASSERT macro. Because each of the arguments to the function involves a pointer being passed to the function, we first of all check that each of the pointers are not NULL pointers.

Inside the function \texttt{MakeTemplate()}, this looks like:

\vspace{5mm}

\begin{tabular}{ll}
void \texttt{MakeTemplate}&(\texttt{Status $\ast$status},     \\
                                   &\texttt{REAL8Vector $\ast$waveform}, \\
                                   &\texttt{InspiralTemplate $\ast$params})
\end{tabular}

\vspace{5mm}

\begin{tabular}{ll}
ASSERT & (waveform!=NULL, \\
       &  status,  \\
       &  MAKETEMPLATE\_ENULL, \\
       &  MAKETEMPLATE\_MSGENULL1);
\end{tabular}

\vspace{5mm}

This above example checks whether the pointer \texttt{waveform} is a NULL pointer or not. If it is a NULL pointer, then an error message which is defined by the character string \texttt{MAKETEMPLATEC\_MSGENULL1} is sent to the screen.


\section{Options}

The options available to the user are the choice of input parameters. As explained above, from the following list of five $(m_{1},m_{2},m,\mu,\eta)$, the user need only specify any one of the following pairs $(m_{1},m_{2})$, $(m,\eta)$ or $(m,\mu)$. The user may also choose which code is to be used to generate the waveform.

\section{Accuracy}

All variables are decalred to be REAL8, which means that they are double precision.
Each double precision variable has an approximate precision of 15 significant figures.



\section{Error conditions}

We check that each of the pointers passed to the function \\ \texttt{MakeTemplate()} as an argument , i.e.\ \texttt{Status}, \texttt{waveform} and \texttt{params}, are not NULL pointers. If any of them are NULL, then an error message is sent to the screen.



\section{Tests}

This function is tested as a part of the test of the function TappRpnTdomFreq(). For a discussion of this test, see the documentation for that function.

\section{Uses}

This function calls the particular function which is going to calculate the waveform.


\section{References}
For a fuller description of how this function is used in the generation of an inspiral waveform, see the documentation for the function \texttt{TappRpnTdomFreq()}.
The nomenclature adopted is the same as that used in Sathyaprakash, PRD, 50, R7111, 1994, which may be consulted for further details.









\end{document}
