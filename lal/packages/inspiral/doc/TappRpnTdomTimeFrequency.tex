\documentclass[12pt]{article}
\usepackage{amsmath}

\begin{document}
\huge
\begin{center}
TappRpnTdomTimeFrequency.c
\end{center}
\normalsize
\vspace{10mm}

\section{Purpose}

The code \texttt{TappRpnTdomTimeFrequency.c} calculates the frequency the waveform from an inspiralling binary system as a function of time up to second post-Nowtonian order. The method used is as follows.


The frequency of a gravitational wave is related to the parameter $x$, which is defined as 
\begin{equation}
x(t) \equiv \left( \frac{Gm \omega(t)}{c^{3}_{0}} \right)^{2/3}
\end{equation}
in the following way

\begin{equation}
f_{GW}(t) = \frac{c^{3}_{0}}{G m \pi} \, x(t)^{3/2} \,\,.
\label{fofx}
\end{equation}
Now, $x(t)$ is given by
\begin{equation}
\begin{split}
x(t) =  &  \frac{\Theta^{-1/4}}{4} \left\{  1 + \left(\frac{743}{4032} + \frac{11}{48} \eta \right) \Theta^{-1/4} - \frac{\pi}{5} \Theta^{-3/8} \right. \\
     &  + \left. \left( \frac{19583}{254016} + \frac{24401}{193536} \eta + \frac{31}{288} \eta^{2} \right) \Theta^{-1/2} \right\}
\end{split}
\label{xoft}
\end{equation}

All of these equations have included explicitly their dependence upon $G$ and $c$. The code uses units where $G=c=1$.



\section{Algorithms}

This code uses no algorithms.


\section{Arguments}

The function header is of the form:

\vspace{5mm}

\begin{tabular}{ll}
void \texttt{TappRpnTdomTimeFrequency}&(\texttt{Status $\ast$status},     \\
                                   &\texttt{InspiralwaveFrequencyOutput $\ast$output}, \\
                                   &\texttt{InspiralwaveFrequencyInput $\ast$params})
\end{tabular}

\vspace{5mm}

The structure which is of type \texttt{Status}, which is pointed to by the pointer \texttt{status} writes information to the screen should the code encounter a problem. The output structure is of the form \texttt{InspiralwaveFrequencyOutput} and is pointed to by the pointer \texttt{output}.
The inputs come from the input structure which is of type \texttt{InspiralwaveFrequencyInput}, and which is pointed to by the pointer \texttt{params}.

The output structure has the form

\vspace{5mm}

\begin{tabular}{ll}
\texttt{typedef struct} & \texttt{tagInspiralwaveFrequencyInput} \{ \\
                        & \texttt{REAL8 frequency;} \\
                        & \} \texttt{InspiralwaveFrequencyInput;}
\end{tabular}

\vspace{5mm}

The input structure is of the form

\vspace{5mm}

\begin{tabular}{ll}
\texttt{typedef struct} & \texttt{tagInspiralwaveFrequencyInput} \{ \\
                        & \texttt{REAL8 eta;} \\
                        & \texttt{REAL8 td;}  \\
                        & \texttt{REAL8 m;}  \\
                        & \} \texttt{InspiralwaveFrequencyInput;}
\end{tabular}

\vspace{5mm}

The parameters which are represented by these input are as follows: \texttt{eta} is the symettric mass ratio $\eta$, \texttt{td} is $\Theta$ above and $m$ is the total mass of the binary. 



\section{Operating Instructions}

Here is an example of a code fragment which shows how the members of the input structure are initialized, and how the function is then called.

\vspace{5mm}

\noindent
\begin{verbatim}
/* Declare the structures to be used  */
\end{verbatim}
\texttt{InspiralwaveFrequencyInput input;} \\
\texttt{InspiralwaveFrequencyOutput output;} \\
\texttt{Status status;} \\
\begin{verbatim}
/* Initialize the inputs  */
\end{verbatim}
\texttt{input.eta} = $m_{1}m_{2}/(m_{1}+m_{2})^{2}$ \\
\texttt{input.td} = $\frac{c^{3}_{0} \eta}{5Gm} (t_{c} - t)$ \\
\texttt{input.m} = $m_{1}+m_{2}$ \\
\\
\begin{verbatim}
/* Call the function */
\end{verbatim}
\texttt{TappRpnTdomTimeFrequency (\&status, \&output, \&input);}
\begin{verbatim}
/* Write the data to the screen  */
  fprintf(stderr,"%e\n",output.frequency); 
\end{verbatim}

Inside the function \texttt{TappRpnTdomTimeFrequency()}, error checks are made upon its arguments, using the ASSERT macro. Because each of the arguments to the function involves a pointer being passed to the function (e.g.\ \texttt{output, input}), we check that each of the pointers are not NULL pointers.
Inside the function \texttt{TappRpnTdomTimeFrequency()}, this looks like:

\vspace{5mm}

\begin{tabular}{ll}
void \texttt{TappRpnTdomTimeFrequency}&(\texttt{Status $\ast$status},     \\
                                   &\texttt{InspiralwaveFrequencyOutput $\ast$output}, \\
                                   &\texttt{InspiralwaveFrequencyInput $\ast$params})
\end{tabular}

\vspace{5mm}

\begin{tabular}{ll}
ASSERT & (output!=NULL,  \\
       &  status,    \\
       &  TAPPRPNTDOMTIMEFREQUENCY\_ENULL, \\
       &  TAPPRPNTDOMTIMEFREQUENCY\_MSGENULL1);
\end{tabular}

\vspace{5mm}

This above example checks whether the pointer \texttt{output} is a NULL pointer or not. If it is a NULL pointer, then an error message which is defined by the character string \texttt{TAPPRPNTDOMTIMEFREQUENCY\_MSGENULL1} is sent to the screen.


\section{Options}

There are no options available to the user.

\section{Accuracy}

All variables are decalred to be REAL8, which means that they are double precision.
Each double precision variable has an approximate precision of 15 significant figures.


\section{Error conditions}

We first of all check that each of the pointers passed to the function \\ \texttt{TappRpnTdomTimeFrequency()} as an argument , i.e.\ \texttt{Status}, \texttt{output} and \texttt{params}, are not NULL pointers. If any of them are NULL, then an error message is sent to the screen.

Checks are performed upon the other inputs to make sure that they have values with the expected ranges. We check that \texttt{eta} is greater than zero and less than or equal to 0.25, that \texttt{td} is greater than zero and that the total mass of the system is greater than 0.4 solar masses and less than 100 solar masses.

\section{Tests}

This function was tested along with the code \texttt{TappRpnTdomTime()}. For a description of this test, see the documentation for the code called \texttt{TappRpnTdomFreq()}.

\section{Uses}

This function does not call any other functions.


\end{document}
