\documentclass[12pt]{article}
\usepackage{amsmath}

\begin{document}
\huge
\begin{center}
ChooseModel.c
\end{center}
\normalsize
\vspace{10mm}

\section{Purpose}

The code \texttt{ChooseModel.c} sets the pointers to the correct energy and flux functions $E^{\prime}(v)$ and $\mathcal{F}(v)$. 

\section{Algorithms}

The code uses no algorithms


\section{Arguments}

The function header is of the form:

\vspace{5mm}

\begin{tabular}{ll}
void \texttt{ChooseModel}&(\texttt{Status $\ast$status},     \\
                                   &\texttt{expnFunc $\ast$f}, \\
                                   &\texttt{expnCoeffs $\ast$ak}, \\
                                   &\texttt{InspiralTemplate $\ast$params})
\end{tabular}

\vspace{5mm}

The structure which is of type \texttt{Status}, which is pointed to by the pointer \texttt{status} writes information to the screen should the code encounter
a problem. The main output structure is of the form \texttt{expnFunc} and is pointed to by the pointer \texttt{f}.
The inputs needed come from the input structure which is of type \texttt{InspiralTemplate}, and which is pointed to by the pointer \texttt{params}. The structure which is of type \texttt{expnCoeffs} and is pointed to by the pointer \texttt{ak} is also an output structure, although most of its members are defined by the function \texttt{InspiralSetup}, some additional ones are set here.

The main output structure has the form

\vspace{5mm}

\begin{tabular}{ll}
\texttt{typedef struct} & \texttt{tagexpnFunc} \{ \\
                        & \texttt{EnergyFunction $\ast$dEnergy;}  \\
                        & \texttt{FluxFunction $\ast$flux;}  \\
                        & \} \texttt{expnFunc;}
\end{tabular}

\vspace{5mm}

Now \texttt{dEnergy} is a pointer to the function which represents $E^{\prime}(v)$. This function is of the form

\vspace{5mm}

\texttt{typedef REAL8 EnergyFunction(REAL8, expnCoeffs *)};

\vspace{5mm}

Similarly, \texttt{flux} is a pointer to the function which represents $\mathcal{F}(v)$. This function has the form

\vspace{5mm}

\texttt{typedef REAL8 FluxFunction(REAL8, expnCoeffs *)};

\vspace{5mm}

The input structure is of the form

\vspace{5mm}

\begin{tabular}{ll}
\texttt{typedef struct} & \texttt{tagInspiralTemplate} \{ \\
                        & \texttt{INT4 number;} \\
                        & \texttt{REAL8 mass1;} \\
                        & \texttt{REAL8 mass2;}  \\
                        & \texttt{REAL8 spin1[3];}  \\
                        & \texttt{REAL8 spin2[3];}  \\
                        & \texttt{REAL8 inclination;} \\
                        & \texttt{REAL8 eccentricity;} \\
                        & \texttt{REAL8 totalMass;} \\
                        & \texttt{REAL8 mu;}  \\
                        & \texttt{REAL8 eta;}  \\
                        & \texttt{REAL8 fLower;}  \\
                        & \texttt{REAL8 fCutoff;} \\
                        & \texttt{REAL8 tSampling;} \\
                        & \texttt{REAL8 startPhase;} \\
                        & \texttt{REAL8 startTime;} \\
                        & \texttt{REAL8 signalAmplitude;} \\
                        & \texttt{REAL8 nStartPad;} \\
                        & \texttt{REAL8 nEndPad;} \\
                        & \texttt{INT4 ieta;} \\
                        & \texttt{InspiralMethod method;}  \\
                        & \texttt{InputMasses massChoice;}  \\
                        & \texttt{Order order;}  \\
                        & \texttt{Domain domain;}  \\
                        & \texttt{Approximant approximant;}  \\
                        & \} \texttt{InspiralTemplate;}
\end{tabular}

\vspace{5mm}

The only members of this structure which are of interest to us are \texttt{approximant} and \texttt{order}. The parameter \texttt{approximant} tells us whether we use T or P--Approximants, and the parameter \texttt{order} tells us to which order of post--Newtonain expansion we go.

The function \texttt{ChooseModel} contains a list of energy and flux functions, all of which are \texttt{static}, and so can only been seen by \texttt{ChooseModel}. However, \texttt{ChooseModel} is able to pass a pointer to any of these individual functions to any function which calls \texttt{ChooseModel}.


\section{Operating Instructions}

Here is an example of a code fragment which shows how the members of the input structure are initialized, and how the function is then called.


\vspace{5mm}

\noindent
\begin{verbatim}
/* Declare the structures to be used  */
\end{verbatim}
\texttt{expnFunc func;} \\
\texttt{expnCoeffs ak;} \\
\texttt{Status status;} \\
\texttt{InspiralTemplate params};
\begin{verbatim}
/* Initialize the inputs  */
\end{verbatim}
\texttt{params.approximant=taylor}; \\
\texttt{params.ortder = newtonain};

\begin{verbatim}
/* Call the function  */
\end{verbatim}
\texttt{ChooseModel (\&status, \&func, \&ak, \&params);} \\


Now, inside the structure \texttt{func}, pointers have been set to the appropriate energy and flux functions $E^{\prime}(v)$ and $\mathcal{F}(v)$.


Inside the function \texttt{ChooseModel()}, error checks are made upon its arguments, using the ASSERT macro. Because each of the arguments to the function involves a pointer being passed to the function (e.g.\ \texttt{f, ak}), we first of all check that each of the pointers are not NULL pointers.

Inside the function \texttt{ChooseModel()}, this looks like:

\vspace{5mm}

\begin{tabular}{ll}
void \texttt{ChooseModel}&(\texttt{Status $\ast$status},     \\
                                   &\texttt{expnFunc $\ast$f}, \\
                                   &\texttt{expnCoeffs $\ast$ak}, \\
                                   &\texttt{InspiralTemplate $\ast$params})
\end{tabular}

\vspace{5mm}

\begin{tabular}{ll}
ASSERT & (f,  \\
       &  status,    \\
       &  CHOOSEMODEL\_ENULL, \\
       &  CHOOSEMODEL\_MSGENULL1);
\end{tabular}

\vspace{5mm}

This above example checks whether the pointer \texttt{f} is a NULL pointer or not. If it is a NULL pointer, then an error message which is defined by the character string \texttt{CHOOSEMODEL\_MSGENULL1} is sent to the screen using the \texttt{StatusHandler} function.


\section{Options}

 

\section{Accuracy}

All variables are decalred to be REAL8, which means that they are double precision.
Each double precision variable has an approximate precision of 15 significant figures.



\section{Error conditions}

We first of all check that each of the pointers passed to the function \\ \texttt{ChooseModel()} as an argument , i.e.\ \texttt{Status}, \texttt{f}, \texttt{ak} and \texttt{params} are not NULL pointers. If any of them are NULL, then an error message is sent to the screen.




\section{Tests}

This code has been extensively tested by B. Sathyaprakash. This test included a comparison to the routines in the GRASP library.

\section{Uses}

This function does not call any other functions.


\section{References}









\end{document}
