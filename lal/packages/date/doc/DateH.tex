%%% $Id$

\section{Header \texttt{Date.h}}
\label{sec:dateh:header}
Provides routines for manipulating date and time information.

\subsection{Synopsis}
\label{sec:dateh:synopsis}
\begin{verbatim}
#include "Date.h"
\end{verbatim}

\subsubsection{Datatypes}
\label{sec:dateh:datatypes}
\begin{verbatim}
/* The standard Unix tm structure */
typedef struct
tm
LALUnixDate;

/* This time object is exactly like LIGOTimeGPS, except for the name */
typedef struct
tagLIGOTimeUTC
{
    INT4 utcSeconds;     /* no. of seconds since Unix epoch */
    INT4 utcNanoSeconds; /* residual no. of nanoseconds */
}
LIGOTimeUTC;

/* Encode timezone information */
typedef struct
tagLALTimezone
{
    INT4 secondsWest; /* seconds West of UTC */
    INT4 dst;         /* Daylight Savings Time correction to apply */
}
LALTimezone;    

/* Date and time structure */
typedef struct
tagLALDate
{
    LALUnixDate unixDate;
    INT4        residualNanoSeconds; /* residual nanoseconds */
    LALTimezone timezone;            /* timezone information */
}
LALDate;
\end{verbatim}

\subsubsection{Routine prototypes}
\begin{verbatim}

void JulianDay (Status*, INT4*, const LALDate*);
void ModJulianDay(Status*, REAL8*, const LALDate*);
void JulianDate(Status*, REAL8*, const LALDate*);
void ModJulianDate(Status*, REAL8*, const LALDate*);
void UTCtoGPS(Status*, LIGOTimeGPS*, const LIGOTimeUTC*);
void GPStoUTC(Status*, LIGOTimeUTC*, const LIGOTimeGPS*);
void UTCtime(Status*, LALDate*, const LIGOTimeUTC*);
void DateString(Status*, CHARVector*, const LALDate*);
void GMST1(Status*, REAL8*, const LALDate*, INT4);
void LMST1(Status*, REAL8*, const LALDate*, REAL8, INT4);
void SecsToLALDate(Status*, LALDate*, REAL8);

\end{verbatim}



\subsection{Error conditions}

%%% Local Variables: 
%%% mode: latex
%%% TeX-master: "Date"
%%% End: 
