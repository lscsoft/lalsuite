%%% $Id$
\section{Module \texttt{DateString.c}}
\label{sec:datestringc}

\subsection{Prototypes}
\begin{verbatim}
void
DateString (Status        *status,
            CHARVector    *timestamp,
            const LALDate *date)
\end{verbatim}


\subsection{Description}

Forms a timestamp string in ISO 8601 format, stored in
\texttt{timestamp->data}.  See \cite{iso8601} available via the Web
\texttt{http://www.iso.ch/markete/8601.pdf}.

\subsection{Operating Instructions}
See Section~\ref{sec:dateh:datatypes} for the definitions of the
data structures used.  Suppose we would like to form a timestamp string for
the current time.  The following program would accomplish this:

\begin{verbatim}
/* Taken from GRASP */
void printone(Status *status, const LIGOTimeUnix *time1)
{
    LIGOTimeGPS  gpstime;
    LALDate      laldate;
    LIGOTimeUnix tmp;
    LALUnixDate  utimestruct;
    CHARVector *utc = NULL;

    INITSTATUS (status, "printone", TESTUTOGPSC);

    /*
     * Allocate space for CHARVectors
     */
    CHARCreateVector(status, &utc, (UINT4)64);

    /* compute GPS time */
    UtoGPS(status, &gpstime, time1);

    /*
     * construct strings with appropriate time stamps
     */
    /* Utime() */
    Utime(status, &laldate, time1);
    DateString(status, utc, &laldate);

    printf("%s\n", utc->data);

    /*
     * House cleaning
     */
    CHARDestroyVector(status, &utc);
    
    RETURN (status);
}
\end{verbatim}

\subsubsection{Arguments}

\paragraph{\texttt{DateString}}

\subparagraph{Inputs}

\begin{itemize}
    \item \texttt{date}: A pointer to an \texttt{LALDate} structure
        containing the date and time for which a timestamp string is to be
        generated.
\end{itemize}

\subparagraph{Outputs}

\begin{itemize}
    \item \texttt{timestamp}: A pointer to a \texttt{CHARVector} structure
    which will contain the timestamp string.  The length of this vector
    should be at least 26.
\end{itemize}

\subsubsection{Error Conditions}
\begin{tabular}{|c|l|l|}
  \hline
  Status & Status       & Explanation \\
  code   & description  &             \\
  \hline
  \tt 1  & \tt Null pointer & Pointer to input data is NULL. \\
  \hline
\end{tabular}

\subsection{Algorithm}

\texttt{DateString()} uses \texttt{strftime (3)} to format the timestamp
string. 

\subsection{Accuracy}

\subsection{Tests}

\subsection{Uses}

% Dependencies, LAL or otherwise

\subsection{Notes}



%%% Local Variables: 
%%% mode: latex
%%% TeX-master: "Date"
%%% End: 
