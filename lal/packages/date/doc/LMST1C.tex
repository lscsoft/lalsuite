%%% $Id$
\section{Module \texttt{LMST1.c}}

\subsection{Prototypes}
\begin{verbatim}
void
GMST1 (Status        *status,
       REAL8         *gmst,
       const LALDate *date,
       INT4           outunits)

void
LMST1 (Status        *status,
       REAL8         *lmst,
       const LALDate *date,
       REAL8          longitude,
       INT4           outunits)
\end{verbatim}

\subsection{Description}

The routines in this module compute Mean Sidereal Time in a choice of
units: seconds, hours, degrees, or radians. \texttt{GMST1} computes GMST1,
and \texttt{LMST1} computes LMST1.  LMST1 is offset from GMST1 by
the longitude of the observing post.

\subsection{Operating Instructions}
Here is a simple example:

\begin{verbatim}
#include <stdlib.h>
#include "LALStdlib.h"
#include "Date.h"

INT4 debuglevel = 2;

NRCSID (TESTLMSTC, "Id");

int
main(int argc, char *argv[])
{
    LALDate date;
    LALDate mstdate;
    REAL8   gmsthours, lmsthours;
    REAL8   gmstsecs;
    REAL8   longitude;
    time_t  timer;
    CHAR    timestamp[64], tmpstr[64];
    Status  status = {0};

    if (argc == 1)
    {
        /*
         * Print help message and exit
         */
        printf("Usage: TestUTCtoGPS debug_level -- debug_level = [0,1,2]\n");
        return 0;
    }

    if (argc == 2)
        debuglevel = atoi(argv[1]);

    INITSTATUS(&status, "TestLMST", TESTLMSTC);

    printf("TEST of GMST1 routine\n");
    printf("=====================\n");

    /*
     * Check against the Astronomical Almanac:
     * For 1994-11-16 0h UT - Julian Date 2449672.5, GMST 03h 39m 21.2738s
     */
    date.unixDate.tm_sec  = 0;
    date.unixDate.tm_min  = 0;
    date.unixDate.tm_hour = 0;
    date.unixDate.tm_mday = 16;
    date.unixDate.tm_mon  = 10;
    date.unixDate.tm_year = 94;

    longitude = 0.; /* Greenwich */
    GMST1(&status, &gmsthours, &date, MST_HRS);
    LMST1(&status, &lmsthours, &date, longitude, MST_HRS);

    GMST1(&status, &gmstsecs, &date, MST_SEC);
    SecsToLALDate(&status, &mstdate, gmstsecs);
    strftime(timestamp, 64, "%Hh %Mm %S", &(mstdate.unixDate));
    sprintf(tmpstr, "%fs", mstdate.residualNanoSeconds * 1.e-9);
    strcat(timestamp, tmpstr+1); /* remove leading 0 */
    
    printf("gmsthours = %f = %s\n", gmsthours, timestamp);
    printf("    expect: 3.655728 = 03h 39m 20.6222s \n");
    /* printf("lmsthours = %f\n", lmsthours); */

    return(0);
}
\end{verbatim}


\subsubsection{Arguments}

\paragraph{\texttt{GMST1}}

\subparagraph{Inputs}

\begin{itemize}

   \item \texttt{date}: A pointer to an \texttt{LALDate} structure
     containing Gregorian date and UTC time for which to compute GMST1.

   \item \texttt{outunits}: An integer defining the units in which GMST1
     are to be expressed.  Preprocessor macros are defined in \texttt{Date.h}
     for selecting the units for MST1, which are passed in the argument
     \texttt{outunits}.  The following table lists the macros:
%
    \begin{center}
         \begin{tabular}{|c|l|}
           \hline 
           \verb MST_SEC & seconds \\
           \verb MST_HRS & hours   \\
           \verb MST_DEG & degrees \\
           \verb MST_RAD & radians \\
           \hline
         \end{tabular}
       \end{center}
%
\end{itemize}

\subparagraph{Outputs}

\begin{itemize}
  \item \texttt{gmst}: A pointer to a \texttt{REAL8} containing GMST1 in
    requested units.
\end{itemize}

\paragraph{\texttt{LMST1}}

\subparagraph{Inputs}

\begin{itemize}

   \item \texttt{date}: A pointer to an \texttt{LALDate} structure
     containing Gregorian date and UTC time for which to compute LMST1.

   \item \texttt{longitude}: A \texttt{REAL8} with the longitude of the
     observing station in decimal degrees (\textit{i.e.} as stored in a
     \texttt{GeodeticCoords} structure (see module \texttt{AM}).

   \item \texttt{outunits}: An integer defining the units in which LMST1
     are to be expressed.  Preprocessor macros are defined in \texttt{Date.h}
     for selecting the units for MST1, which are passed in the argument
     \texttt{outunits}.  The following table lists the macros:
%
    \begin{center}
         \begin{tabular}{|c|l|}
           \hline 
           \verb MST_SEC & seconds \\
           \verb MST_HRS & hours   \\
           \verb MST_DEG & degrees \\
           \verb MST_RAD & radians \\
           \hline
         \end{tabular}
       \end{center}
%
\end{itemize}

\subparagraph{Outputs}

\begin{itemize}
  \item \texttt{lmst}: A pointer to a \texttt{REAL8} containing the LMST1 in
    requested units.
\end{itemize}


\subsubsection{Error Conditions}

\paragraph{\texttt{LMST1}}

\subparagraph{}

\begin{tabular}{|c|l|l|}
  \hline
  Status & Status       & Explanation \\
  code   & description  &             \\
  \hline
  \tt 1  & \tt Null pointer & Pointer to input data is NULL. \\
  \hline
\end{tabular}

\paragraph{\texttt{GMST1}}

\subparagraph{}

\begin{tabular}{|c|l|l|}
  \hline
  Status & Status       & Explanation \\
  code   & description  &             \\
  \hline
  \tt 1  & \tt Null pointer & Pointer to input data is NULL. \\
  \hline
\end{tabular}

\subsection{Algorithm}

The formula used to compute GMST1 at 0h UT1 is from the \textit{Explanatory
Supplement to the Astronomical Almanac}, Ch.~2, Sec.~24 \cite{esaa:1992},
and also in Section B of the \textit{Astronomical Almanac}.  The formula is
%
\begin{eqnarray*}
    {\text{GMST at $0^{h}$ UT}} & = & 24\ 110^{s}\!.548\ 41 + 8\ 640\ 184^{s}\!.812\ 866 T_{U} \\
 & & + 0^{s}\!.093\ 104 T_{U}^{2} - 6^{s}\!.2\times10^{-6} T_{U}^{3}
\end{eqnarray*}
%
where $T_{U} = (\mathrm{JD} - 2\ 451\ 545.0)/36\ 525$.  To compute GMST at
some other time UT, the Mean Sidereal Time interval is added to the GMST
computed above.  The Mean Sidereal Time interval is obtained by multiplying
the UT time interval by $1.002737909350795 + 5.9006\times10^{-11}T_{U} +
5.9\times10^{-15}T_{U}^{2}$.


\subsection{Accuracy}
This implementation for computing GMST1 is accurate to about 1 sidereal
second.  To compute accurate locations of celestial objects, what is needed
is the Apparent Sidereal Time, which differs from Mean Sidereal Time by the
Equation of the Equinoxes:
%
\begin{eqnarray*}
  GAST = GMST + \textrm{equation of equinoxes}
\end{eqnarray*}
%
where the equation of equinoxes $= \frac{1}{15}(\Delta\psi \cos\epsilon +
0^{"}\!.002\ 64 \sin\Omega + 0^{"}\!.000\ 063 \sin
2\Omega$, and $\Delta\psi$ is the total nutation in longitude, $\epsilon$
is the mean obliquity of th eecliptic and $\Omega$ is the mean longitude of
the ascending node of the Moon.  The Equation of the Equinoxes is tabulated
in the \textit{Astronomical Almanac}, and can be interpolated.  It can also
be computed using software from the US Naval Observatory
(\texttt{http://aa.usno.navy.mil/AA/}). For the year 2000, the Equation of
the Equinoxes goes to an absolute maximum of about 1 sidereal second.

\subsection{Tests}

\subsection{Uses}

\subsection{Notes}


%%% Local Variables: 
%%% mode: latex
%%% TeX-master: "Date"
%%% End: 
