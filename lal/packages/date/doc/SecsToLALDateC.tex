%%% $Id$


\section{Module \texttt{SecsToLALDate.c}}

\subsection{Prototypes}
\begin{verbatim}
void
SecsToLALDate(Status  *status,
              LALDate *date,
              REAL8    seconds)
\end{verbatim}

\subsection{Description}

This routine converts a time of day in decimal seconds since 0h (midnight)
to an LALDate structure.  Of course, the date information is not present.

\subsection{Operating Instructions}

A simple example:

\begin{verbatim}
#include <stdlib.h>
#include "LALStdlib.h"
#include "Date.h"

INT4 debuglevel = 2;

NRCSID (TESTLMSTC, "Id");

int
main(int argc, char *argv[])
{
    LALDate date;
    LALDate mstdate;
    REAL8   gmstsecs;
    CHAR    timestamp[64], tmpstr[64];
    Status  status = {0};

    
    INITSTATUS(&status, "TestLMST", TESTLMSTC);

    printf("TEST of GMST1 routine\n");
    printf("=====================\n");

    /*
     * Check against the Astronomical Almanac:
     * For 1994-11-16 0h UT - Julian Date 2449672.5, GMST 03h 39m 21.2738s
     */
    date.unixDate.tm_sec  = 0;
    date.unixDate.tm_min  = 0;
    date.unixDate.tm_hour = 0;
    date.unixDate.tm_mday = 16;
    date.unixDate.tm_mon  = 10;
    date.unixDate.tm_year = 94;

    GMST1(&status, &gmstsecs, &date, MST_SEC);
    SecsToLALDate(&status, &mstdate, gmstsecs);
    strftime(timestamp, 64, "%Hh %Mm %S", &(mstdate.unixDate));
    sprintf(tmpstr, "%fs", mstdate.residualNanoSeconds * 1.e-9);
    strcat(timestamp, tmpstr+1); /* remove leading 0 */
    
    printf("gmst = %s\n", timestamp);
    printf("    expect: 03h 39m 20.6222s \n");

    return(0);
}

\end{verbatim}

\subsubsection{Arguments}

\paragraph{\texttt{SecsToLALDate}}

\subparagraph{Inputs}

\begin{itemize}
    \item \texttt{seconds}: A \texttt{REAL8} containing decimal number of
    seconds since 0h (midnight) to be converted to an \texttt{LALDate}
    structure.
\end{itemize}

\subparagraph{Outputs}

\begin{itemize}
    \item \texttt{date}: A pointer to an \texttt{LALDate} structure
    containing the converted time.
\end{itemize}

\subsubsection{Error Conditions}
\begin{tabular}{|c|l|l|}
  \hline
  Status & Status       & Explanation \\
  code   & description  &             \\
  \hline
  \tt 1  & \tt Null pointer & Pointer to output data is NULL. \\
  \hline
\end{tabular}

\subsection{Algorithm}

\subsection{Accuracy}

\subsection{Tests}

\subsection{Uses}

\subsection{Notes}


%%% Local Variables: 
%%% mode: latex
%%% TeX-master: "Date"
%%% End: 
