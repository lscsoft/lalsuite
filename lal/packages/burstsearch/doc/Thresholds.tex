% This file is meant to be included in another 
\documentclass{article}
\begin{document}
\section{Thresholds}

\subsection{Purpose}

Compute thresholds and quantities related to thresholds for
chi-squared and non-central chi-squared distributions \cite{ptvf:1992},
for use in the excess power search technique.

 
\subsection{Synopsis}

% Syntax: argument definitions, calling signature

\begin{verbatim}
#include "Thresholds.h"


typedef struct tagChisqCdfIn
{
  REAL8     chi2;               /* value of chi squared          */
  REAL8     dof;                /* number of degrees of freedom  */
  REAL8     nonCentral;         /* non-centrality parameter      */
}
ChisqCdfIn;

typedef struct tagChi2ThresholdIn
{
  REAL8     dof;                /* number of degrees of freedom  */
  REAL8     falseAlarm;         /* false alarm probability       */
}
Chi2ThresholdIn;

typedef struct tagRhoThresholdIn
{
  REAL8     chi2;               /* value of chi squared          */
  REAL8     dof;                /* number of degrees of freedom  */
  REAL8     falseDismissal;     /* false dismissal probability   */
}
RhoThresholdIn;


void
ChisqCdf (
          Status                        *status,
          REAL8                         *prob,
          ChisqCdfIn                    *input
          );

void
OneMinusChisqCdf (
                  Status                *status,
                  REAL8                 *prob,
                  ChisqCdfIn            *input
                  );

void
NoncChisqCdf (
              Status                    *status,
              REAL8                     *prob,
              ChisqCdfIn                *input
              );

void
Chi2Threshold (
              Status                    *status,
              REAL8                     *chi2,
              Chi2ThresholdIn           *input
               );

void
RhoThreshold (
              Status                    *status,
              REAL8                     *rho,
              RhoThresholdIn            *input
              );

\end{verbatim}


\subsection{Description}

This package provides a suite for functions for computing cumulative
probability functions and thresholds for the chi-squared and
non-central chi-squared distributions.

The routine \verb+ChisqCdf()+ computes as a function of $\chi^2$ the
probability $p = p(\chi^2,n)$ that $x_1^2 + x_2^2 + \ldots + x_n^2 \le \chi^2$,
where $x_1$, $x_2$, \ldots $x_n$ are independent Gaussian random
variables of zero mean and unit variance, and $n$ is the number of
degrees of freedom.  An integral expression for $p$ is
$$
p(\chi^2,n) = {1 \over \Gamma(n/2)} \int_0^{\chi^2/2} dx \, x^{n/2-1} e^{-x},
$$
where 
$$
\Gamma(n/2) \equiv \int_0^\infty dx \, x^{n/2-1} e^{-x}
$$
is the usual $\Gamma$ function.  \verb+ChisqCdf()+ takes as input a
structure \verb+ChisqCdfIn+ which has elements \verb+chi2+ or
$\chi^2$, \verb+dof+ or $n$, and \verb+nonCentral+ which is not used
by \verb+ChisqCdf()+.  The probability $p$ is returned by the routine 
in the output variable \verb+*prob+.


The routine \verb+OneMinusChisqCdf()+ takes the same arguments as
\verb+ChisqCdf()+ and computes $1-p$.  It is more accurate than
\verb+ChisqCdf()+ when $p$ is very close to $1$.  The quantity $1-p$
is returned in the output variable \verb+*prob+.


The routine \verb+NoncChisqCdf()+ takes the same arguments
\verb+ChisqCdf()+ and computes the non-central chi-squared
distribution.  It computes as a function of $\chi^2$ the
probability $p = p(\chi^2,n,\rho)$ that $(x_1+\rho)^2 + x_2^2 + \ldots
+ x_n^2 \le \chi^2$, 
where $x_1$, $x_2$, \ldots $x_n$ are independent Gaussian random
variables of zero mean and unit variance, $n$ is the number of
degrees of freedom, and $\rho^2$ is called the non-centrality
parameter.  It uses the series formula
$$
p(\chi^2,n,\rho) = \sum_{j=0}^\infty \, { \exp\left[ -{1 \over 2}
\rho^2 \right] \over j!} \ \left( {\rho^2 \over 2} \right)^j \
p(\chi^2,n+2j)
$$
and uses \verb+ChisqCdf()+.  The non-centrality parameter
$\rho^2$ is passed to \verb+NoncChisqCdf()+ through the element
\verb+nonCentral+ of the input structure \verb+ChisqCdfIn+.
Again $p$ is returned in the output variable \verb+*prob+.



The routine \verb+Chi2Threshold()+ returns the value $\chi^2 =
\chi^2(\alpha,n)$ of $\chi^2$ such that 
$$
\alpha = 1 - p(\chi^2,n),
$$
where $\alpha$ is a false alarm probability and $n$ is the number of
degrees of freedom.  \verb+Chi2Threshold()+ takes as input a
structure \verb+Chi2ThresholdIn+ which has elements \verb+dof+ or
$n$ and \verb+falseAlarm+ or $\alpha$.  The square $\chi^2$ of the
threshold $\chi$ is returned by the routine in the output variable
\verb+*chi2+. 




The routine \verb+RhoThreshold()+ returns the value $\rho =
\rho(\chi^2,n,\beta)$ of $\rho$ such that 
$$
\beta = p(\chi^2,n,\rho),
$$
where $\beta$ is a false dismissal probability and $n$ is the number
of degrees of freedom.  \verb+RhoThreshold()+ takes as input a
structure \verb+RhoThresholdIn+ which has elements \verb+chi2+ or
$\chi^2$, \verb+dof+ or $n$, and \verb+falseDismissal+ or $\beta$.
The result $\rho(\chi^2,n,\beta)$ is returned by the routine in the
output variable \verb+*rho+. 


Notes:
\begin{enumerate}
\item These are all double precision (REAL8) functions of double
precision variables.
\item The functions \verb+Chi2Threshold()+ and \verb+RhoThreshold()+
use the routines \verb+BracketRootReal8()+ and
\verb+BisectionFindRootReal8()+. 
\item The definitions given above for $p(\chi^2,n)$ and
$p(\chi^2,n,\rho)$ are for integral $n$ only.  However,
the integral formulae serve to define these quantities for all
positive real $n$, and the routines work for non-integral $n$.
\end{enumerate}

\subsection{Operating Instructions}

% Detailed usage 

\begin{verbatim}

static Status status;

{ 
  /* use of ChisqCdf(), OneMinusChisqCdf(), and NoncChisqCdf() */
  ChisqCdfIn         input;
  REAL8              prob;

  input.dof = 8.0;
  input.chi2 = 2.0;
  /* input.nonCentral not used by ChisqCdf() or OneMinusChisqCdf() */

  ChisqCdf (&status, &prob, &input);
  /* prob now contains required probability */
  OneMinusChisqCdf (&status, &prob, &input)
  /* prob is now 1 - (what it was before)   */

  input.nonCentral = 1.0;
  NoncChisqCdf (&status, &prob, &input);
  /* prob now contains required probability */
}

{
  /* use of Chi2Threshold()  */
  Chi2ThresholdIn      input;
  REAL8                chi2;

  input.dof = 3.0;
  input.falseAlarm = 1e-6;
  Chi2Threshold (&status, &chi2, &input);
  /* chi2 now contains required threshold */
}

{
  /* use of RhoThreshold()  */
  RhoThresholdIn       input;
  REAL8                rho;

  input.dof = 3.0;
  input.falseDismissal = 0.5;
  input.chi2 = 1.0;
  RhoThreshold (&status, &rho, &input);
  /* rho now contains required value */
}




\end{verbatim}

\subsubsection{Arguments}

% Describe meaning of each argument

\begin{itemize}
\item \texttt{status} is a universal status structure.  Its contents are
assigned by the functions.
\item \texttt{ChisqCdfIn} is a structure containing the input
parameters for the functions \verb+ChisqCdf()+,
\verb+OneMinusChisqCdf()+, and \verb+NoncChisqCdf()+.
\item \texttt{Chi2ThresholdIn} is a structure containing the input
parameters for the function \verb+Chi2Threshold()+.
\item \texttt{RhoThresholdIn} is a structure containing the input
parameters for the function \verb+RhoThreshold()+.
\item \texttt{dof} is the number of degrees of freedom.  It can be any
positive real number, and is represented as a REAL8.  It is an input
argument to the functions \verb+ChisqCdf()+,
\verb+OneMinusChisqCdf()+, \verb+NoncChisqCdf()+ as a member of the
input structure \verb+ChisqCdfIn+.  It is also an input argument to
the functions \verb+Chi2Threshold()+ and \verb+RhoThreshold()+ as a
member of the input structures \verb+Chi2ThresholdIn+ and
\verb+RhoThresholdIn+ respectively.
\item \texttt{chi2} is the value of $\chi^2$.  It is an input
argument to the functions \verb+ChisqCdf()+,
\verb+OneMinusChisqCdf()+, \verb+NoncChisqCdf()+ as a member of the 
input structure \verb+ChisqCdfIn+.  It is also an input argument to
the function \verb+RhoThreshold()+ as a member of the input structure
\verb+RhoThresholdIn+, and an output argument to the function
\verb+Chi2Threshold()+. 
\item \texttt{nonCentral} is the non-centrality parameter $\rho^2$.
It is an input argument to the function \verb+NoncChisqCdf()+,
as a member of the input structure \verb+ChisqCdfIn+.  
\item \texttt{rho} is the square root of the non-centrality parameter
$\rho^2$.  It is an output argument to the function
\verb+RhoThreshold()+.
\item \texttt{falseAlarm} is the false alarm probability $\alpha$.  It
is an input parameter for \verb+Chi2Threshold()+ as a member of the
input structure \verb+Chi2ThresholdIn+.
\item \texttt{falseDismissal} is the false dismissal probability $\beta$.  It
is an input parameter for \verb+RhoThreshold()+ as a member of the
input structure \verb+RhoThresholdIn+.

\end{itemize}

\subsubsection{Options}

None. 

\subsubsection{Error conditions}

% What constitutes an error condition? What do the error codes mean?

These functions all set the universal status structure on return.
Error conditions are described in the following table.

\begin{table}
\begin{tabular}{|r|l|p{2in}|}\hline
status  & status          & Description\\
code    & description     & \\\hline
THRESHOLDS\_ENULLP 1   & Null pointer
  & an argument is NULL or contains a NULL pointer\\
THRESHOLDS\_EPOSARG 2   & Arguments must be non-negative
  & Illegal zero or negative arguments encountered \\
THRESHOLDS\_EMXIT 4   & Maximum iterations exceeded
  & Error in internal computations\\
THRESHOLDS\_EBADPROB 8  & Supplied probability must 
  & probability outside legal range\\ 
 \, & be between 0 and 1
 & \, \\
THRESHOLDS\_ERANGE 16  & Arguments too large,
  & result is 0 or 1 to machine accuracy\\ 
 \, & cannot obtain finite probability
 & \, \\

\hline
\end{tabular}
\caption{Error conditions for all THRESHOLDS functions}\label{tbl:CV}
\end{table}
                                
\subsection{Algorithms}


% Describe algorithm by which work is done

\subsection{Accuracy}

% For numerical routines address issues related to accuracy:
% approximations, argument ranges, etc.


\subsection{Tests}

% Describe the tests that are part of the test suite

\subsection{Uses}

% What LAL, other routines does this one call?

\begin{itemize}
\item\texttt{BracketRootReal8()}
\item\texttt{BisectionFindRootReal8()}
\end{itemize}

\subsection{Notes}

\subsection{References}

% Any references for algorithms, tests, etc.
\begin{thebibliography}{0}

\bibitem{ptvf:1992}
  W. H. Press, S. A. Teukolsky, W. T. Vetterling, and B. P. Flannery,
  \textit{Numerical Recipes in C: The Art of Scientific Computing}, 2nd ed.
  (Cambridge University Press, Cambridge, England, 1992).
\end{thebibliography}

\end{document}

%%% Local Variables: 
%%% mode: latex
%%% TeX-master: t
%%% End: 

