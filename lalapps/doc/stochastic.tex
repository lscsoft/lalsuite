% $Id$

%
% API Documentation
% Module stochastic
%
% Generated by epydoc 2.1
% [Tue Sep 14 14:45:48 2004]
%

%%%%%%%%%%%%%%%%%%%%%%%%%%%%%%%%%%%%%%%%%%%%%%%%%%%%%%%%%%%%%%%%%%%%%%%%%%%
%%                          Module Description                           %%
%%%%%%%%%%%%%%%%%%%%%%%%%%%%%%%%%%%%%%%%%%%%%%%%%%%%%%%%%%%%%%%%%%%%%%%%%%%

    \index{stochastic \textit{(module)}|(}
\section{Python Module \texttt{stochastic}}

    \label{stochastic}
Classes needed for the stochastic analysis pipeline.
This script produced the necessary condor submit and dag files to run the 
standalone stochastic code on LIGO data.


%%%%%%%%%%%%%%%%%%%%%%%%%%%%%%%%%%%%%%%%%%%%%%%%%%%%%%%%%%%%%%%%%%%%%%%%%%%
%%                               Variables                               %%
%%%%%%%%%%%%%%%%%%%%%%%%%%%%%%%%%%%%%%%%%%%%%%%%%%%%%%%%%%%%%%%%%%%%%%%%%%%

  \subsection{Variables}

\begin{longtable}{|p{.30\textwidth}|p{.62\textwidth}|l}
\cline{1-2}
\cline{1-2} \centering \textbf{Name} & \centering \textbf{Description}& \\
\cline{1-2}
\endhead\cline{1-2}\multicolumn{3}{r}{\small\textit{continued on next page}}\\\endfoot\cline{1-2}
\endlastfoot\raggedright \_\-\_\-a\-u\-t\-h\-o\-r\-\_\-\_\- & \raggedright \textbf{Value:} 
{\tt '\-A\-d\-a\-m\-~\-M\-e\-r\-c\-e\-r\-~\-{\textless}\-r\-a\-m\-@\-s\-t\-a\-r\-.\-s\-r\-.\-b\-h\-a\-m\-.\-a\-c\-.\-u\-k\-{\textgreater}\-'\-}            \textit{(type=\texttt{str})}&\\
\cline{1-2}
\raggedright \_\-\_\-d\-a\-t\-e\-\_\-\_\- & \raggedright \textbf{Value:} 
{\tt '\-\$\-D\-a\-t\-e\-:\-~\-2\-0\-0\-4\-/\-0\-7\-/\-2\-3\-~\-1\-4\-:\-2\-3\-:\-3\-0\-~\-\$\-'\-}            \textit{(type=\texttt{str})}&\\
\cline{1-2}
\raggedright \_\-\_\-v\-e\-r\-s\-i\-o\-n\-\_\-\_\- & \raggedright \textbf{Value:} 
{\tt '\-1\-.\-8\-'\-}            \textit{(type=\texttt{str})}&\\
\cline{1-2}
\end{longtable}

    \index{stochastic \textit{(module)}!StochasticError \textit{(class)}|(}

%%%%%%%%%%%%%%%%%%%%%%%%%%%%%%%%%%%%%%%%%%%%%%%%%%%%%%%%%%%%%%%%%%%%%%%%%%%
%%                           Class Description                           %%
%%%%%%%%%%%%%%%%%%%%%%%%%%%%%%%%%%%%%%%%%%%%%%%%%%%%%%%%%%%%%%%%%%%%%%%%%%%

\subsection{Class StochasticError}

    \label{stochastic:StochasticError}
\begin{tabular}{cccccc}
% Line for exceptions.Exception, linespec=[False]
\multicolumn{2}{r}{\settowidth{\BCL}{exceptions.Exception}\multirow{2}{\BCL}{exceptions.Exception}}
&&
  \\\cline{3-3}
  &&\multicolumn{1}{c|}{}
&&
  \\
&&\multicolumn{2}{l}{\textbf{StochasticError}}
\end{tabular}


%%%%%%%%%%%%%%%%%%%%%%%%%%%%%%%%%%%%%%%%%%%%%%%%%%%%%%%%%%%%%%%%%%%%%%%%%%%
%%                                Methods                                %%
%%%%%%%%%%%%%%%%%%%%%%%%%%%%%%%%%%%%%%%%%%%%%%%%%%%%%%%%%%%%%%%%%%%%%%%%%%%

  \subsubsection{Methods}

    \label{stochastic:StochasticError:__init__}
    \index{stochastic \textit{(module)}!StochasticError \textit{(class)}!\_\_init\_\_ \textit{(method)}}
    \vspace{0.5ex}

    \begin{boxedminipage}{\textwidth}

    \raggedright \textbf{\_\_init\_\_}(\textit{self}, \textit{args}=\texttt{N\-o\-n\-e\-})

      Overrides: exceptions.Exception.\_\_init\_\_

    \end{boxedminipage}

  \textbf{Inherited from Exception:}
    \_\_getitem\_\_,
    \_\_str\_\_
    \index{stochastic \textit{(module)}!StochasticError \textit{(class)}|)}
    \index{stochastic \textit{(module)}!StochasticJob \textit{(class)}|(}

%%%%%%%%%%%%%%%%%%%%%%%%%%%%%%%%%%%%%%%%%%%%%%%%%%%%%%%%%%%%%%%%%%%%%%%%%%%
%%                           Class Description                           %%
%%%%%%%%%%%%%%%%%%%%%%%%%%%%%%%%%%%%%%%%%%%%%%%%%%%%%%%%%%%%%%%%%%%%%%%%%%%

\subsection{Class StochasticJob}

    \label{stochastic:StochasticJob}
\begin{tabular}{cccccccc}
% Line for pipeline.AnalysisJob, linespec=[False]
\multicolumn{4}{r}{\settowidth{\BCL}{pipeline.AnalysisJob}\multirow{2}{\BCL}{pipeline.AnalysisJob}}
&&
  \\\cline{5-5}
  &&&&\multicolumn{1}{c|}{}
&&
  \\
% Line for pipeline.CondorJob, linespec=[False, True]
\multicolumn{2}{r}{\settowidth{\BCL}{pipeline.CondorJob}\multirow{2}{\BCL}{pipeline.CondorJob}}
&&
&&\multicolumn{1}{|c}{}
  \\\cline{3-3}
  &&\multicolumn{1}{c|}{}
&&
&\multicolumn{1}{|c}{}&
  \\
% Line for pipeline.CondorDAGJob, linespec=[True]
\multicolumn{4}{r}{\settowidth{\BCL}{pipeline.CondorDAGJob}\multirow{2}{\BCL}{pipeline.CondorDAGJob}}
&&\multicolumn{1}{|c}{}
  \\\cline{5-5}
  &&&&\multicolumn{1}{c|}{}
&\multicolumn{1}{|c}{}&
  \\
&&&&\multicolumn{2}{l}{\textbf{StochasticJob}}
\end{tabular}

A lalapps\_stochastic job used by the stochastic pipeline. The static 
options are read from the section [stochastic] in the ini file. The 
stdout and stderr from the job are directed to the logs directory. The 
path to the executable and the universe is determined from the ini file.


%%%%%%%%%%%%%%%%%%%%%%%%%%%%%%%%%%%%%%%%%%%%%%%%%%%%%%%%%%%%%%%%%%%%%%%%%%%
%%                                Methods                                %%
%%%%%%%%%%%%%%%%%%%%%%%%%%%%%%%%%%%%%%%%%%%%%%%%%%%%%%%%%%%%%%%%%%%%%%%%%%%

  \subsubsection{Methods}

    \label{stochastic:StochasticJob:__init__}
    \index{stochastic \textit{(module)}!StochasticJob \textit{(class)}!\_\_init\_\_ \textit{(method)}}
    \vspace{0.5ex}

    \begin{boxedminipage}{\textwidth}

    \raggedright \textbf{\_\_init\_\_}(\textit{self}, \textit{cp})

    \vspace{-1.5ex}

    \rule{\textwidth}{0.5\fboxrule}
    cp = ConfigParser object from which options are read.

    \vspace{1ex}

      Overrides: pipeline.CondorDAGJob.\_\_init\_\_

    \end{boxedminipage}

  \textbf{Inherited from AnalysisJob:}
    channel,
    get\_config
    \\
  \textbf{Inherited from CondorDAGJob:}
    add\_var\_arg,
    add\_var\_opt
    \\
  \textbf{Inherited from CondorJob:}
    add\_arg,
    add\_condor\_cmd,
    add\_ini\_opts,
    add\_opt,
    add\_short\_opt,
    get\_args,
    get\_opts,
    get\_short\_opts,
    get\_stderr\_file,
    get\_stdout\_file,
    get\_sub\_file,
    set\_log\_file,
    set\_notification,
    set\_stderr\_file,
    set\_stdout\_file,
    set\_sub\_file,
    write\_sub\_file
    \index{stochastic \textit{(module)}!StochasticJob \textit{(class)}|)}
    \index{stochastic \textit{(module)}!StochasticNode \textit{(class)}|(}

%%%%%%%%%%%%%%%%%%%%%%%%%%%%%%%%%%%%%%%%%%%%%%%%%%%%%%%%%%%%%%%%%%%%%%%%%%%
%%                           Class Description                           %%
%%%%%%%%%%%%%%%%%%%%%%%%%%%%%%%%%%%%%%%%%%%%%%%%%%%%%%%%%%%%%%%%%%%%%%%%%%%

\subsection{Class StochasticNode}

    \label{stochastic:StochasticNode}
\begin{tabular}{cccccccc}
% Line for pipeline.CondorDAGNode, linespec=[False, False]
\multicolumn{2}{r}{\settowidth{\BCL}{pipeline.CondorDAGNode}\multirow{2}{\BCL}{pipeline.CondorDAGNode}}
&&
&&
  \\\cline{3-3}
  &&\multicolumn{1}{c|}{}
&&
&&
  \\
% Line for pipeline.AnalysisNode, linespec=[False]
\multicolumn{4}{r}{\settowidth{\BCL}{pipeline.AnalysisNode}\multirow{2}{\BCL}{pipeline.AnalysisNode}}
&&
  \\\cline{5-5}
  &&&&\multicolumn{1}{c|}{}
&&
  \\
% Line for pipeline.CondorDAGNode, linespec=[True]
\multicolumn{4}{r}{\settowidth{\BCL}{pipeline.CondorDAGNode}\multirow{2}{\BCL}{pipeline.CondorDAGNode}}
&&\multicolumn{1}{|c}{}
  \\\cline{5-5}
  &&&&\multicolumn{1}{c|}{}
&\multicolumn{1}{|c}{}&
  \\
&&&&\multicolumn{2}{l}{\textbf{StochasticNode}}
\end{tabular}

An StochaticNode runs an instance of the stochastic code in a Condor DAG.


%%%%%%%%%%%%%%%%%%%%%%%%%%%%%%%%%%%%%%%%%%%%%%%%%%%%%%%%%%%%%%%%%%%%%%%%%%%
%%                                Methods                                %%
%%%%%%%%%%%%%%%%%%%%%%%%%%%%%%%%%%%%%%%%%%%%%%%%%%%%%%%%%%%%%%%%%%%%%%%%%%%

  \subsubsection{Methods}

    \label{stochastic:StochasticNode:__init__}
    \index{stochastic \textit{(module)}!StochasticNode \textit{(class)}!\_\_init\_\_ \textit{(method)}}
    \vspace{0.5ex}

    \begin{boxedminipage}{\textwidth}

    \raggedright \textbf{\_\_init\_\_}(\textit{self}, \textit{job})

    \vspace{-1.5ex}

    \rule{\textwidth}{0.5\fboxrule}
    job = A CondorDAGJob that can run an instance of lalapps\_stochastic.

    \vspace{1ex}

      Overrides: pipeline.AnalysisNode.\_\_init\_\_

    \end{boxedminipage}

    \label{stochastic:StochasticNode:get_ifo_one}
    \index{stochastic \textit{(module)}!StochasticNode \textit{(class)}!get\_ifo\_one \textit{(method)}}
    \vspace{0.5ex}

    \begin{boxedminipage}{\textwidth}

    \raggedright \textbf{get\_ifo\_one}(\textit{self})

    \vspace{-1.5ex}

    \rule{\textwidth}{0.5\fboxrule}
    Returns the IFO code of the primary interferometer.

    \vspace{1ex}

    \end{boxedminipage}

    \label{stochastic:StochasticNode:get_ifo_two}
    \index{stochastic \textit{(module)}!StochasticNode \textit{(class)}!get\_ifo\_two \textit{(method)}}
    \vspace{0.5ex}

    \begin{boxedminipage}{\textwidth}

    \raggedright \textbf{get\_ifo\_two}(\textit{self})

    \vspace{-1.5ex}

    \rule{\textwidth}{0.5\fboxrule}
    Returns the IFO code of the primary interferometer.

    \vspace{1ex}

    \end{boxedminipage}

    \label{stochastic:StochasticNode:get_output}
    \index{stochastic \textit{(module)}!StochasticNode \textit{(class)}!get\_output \textit{(method)}}
    \vspace{0.5ex}

    \begin{boxedminipage}{\textwidth}

    \raggedright \textbf{get\_output}(\textit{self})

    \vspace{-1.5ex}

    \rule{\textwidth}{0.5\fboxrule}
    Returns the file name of output from the stochastic code. This must 
    be kept synchronized with the name of the output file in 
    stochastic.c.

    \vspace{1ex}

      Overrides: pipeline.AnalysisNode.get\_output

    \end{boxedminipage}

    \label{stochastic:StochasticNode:set_cache_one}
    \index{stochastic \textit{(module)}!StochasticNode \textit{(class)}!set\_cache\_one \textit{(method)}}
    \vspace{0.5ex}

    \begin{boxedminipage}{\textwidth}

    \raggedright \textbf{set\_cache\_one}(\textit{self}, \textit{file})

    \vspace{-1.5ex}

    \rule{\textwidth}{0.5\fboxrule}
    Set the LAL frame cache to to use. The frame cache is passed to the 
    job with the --frame-cache-one argument. file = calibration file to 
    use.

    \vspace{1ex}

    \end{boxedminipage}

    \label{stochastic:StochasticNode:set_cache_two}
    \index{stochastic \textit{(module)}!StochasticNode \textit{(class)}!set\_cache\_two \textit{(method)}}
    \vspace{0.5ex}

    \begin{boxedminipage}{\textwidth}

    \raggedright \textbf{set\_cache\_two}(\textit{self}, \textit{file})

    \vspace{-1.5ex}

    \rule{\textwidth}{0.5\fboxrule}
    Set the LAL frame cache to to use. The frame cache is passed to the 
    job with the --frame-cache-two argument. file = calibration file to 
    use.

    \vspace{1ex}

    \end{boxedminipage}

    \label{stochastic:StochasticNode:set_calibration_one}
    \index{stochastic \textit{(module)}!StochasticNode \textit{(class)}!set\_calibration\_one \textit{(method)}}
    \vspace{0.5ex}

    \begin{boxedminipage}{\textwidth}

    \raggedright \textbf{set\_calibration\_one}(\textit{self}, \textit{ifo}, \textit{start})

    \vspace{-1.5ex}

    \rule{\textwidth}{0.5\fboxrule}
    Set the path to the calibration cache file for the given IFO. During 
    S2, the Hanford 2km IFO had two calibration epochs, so if the start 
    time is during S2, we use the correct cache file.

    \vspace{1ex}

    \end{boxedminipage}

    \label{stochastic:StochasticNode:set_calibration_two}
    \index{stochastic \textit{(module)}!StochasticNode \textit{(class)}!set\_calibration\_two \textit{(method)}}
    \vspace{0.5ex}

    \begin{boxedminipage}{\textwidth}

    \raggedright \textbf{set\_calibration\_two}(\textit{self}, \textit{ifo}, \textit{start})

    \vspace{-1.5ex}

    \rule{\textwidth}{0.5\fboxrule}
    Set the path to the calibration cache file for the given IFO. During 
    S2, the Hanford 2km IFO had two calibration epochs, so if the start 
    time is during S2, we use the correct cache file.

    \vspace{1ex}

    \end{boxedminipage}

    \label{stochastic:StochasticNode:set_ifo_one}
    \index{stochastic \textit{(module)}!StochasticNode \textit{(class)}!set\_ifo\_one \textit{(method)}}
    \vspace{0.5ex}

    \begin{boxedminipage}{\textwidth}

    \raggedright \textbf{set\_ifo\_one}(\textit{self}, \textit{ifo})

    \vspace{-1.5ex}

    \rule{\textwidth}{0.5\fboxrule}
    Set the interferometer code to use as IFO One. ifo = IFO code (e.g. 
    L1, H1 or H2).

    \vspace{1ex}

    \end{boxedminipage}

    \label{stochastic:StochasticNode:set_ifo_two}
    \index{stochastic \textit{(module)}!StochasticNode \textit{(class)}!set\_ifo\_two \textit{(method)}}
    \vspace{0.5ex}

    \begin{boxedminipage}{\textwidth}

    \raggedright \textbf{set\_ifo\_two}(\textit{self}, \textit{ifo})

    \vspace{-1.5ex}

    \rule{\textwidth}{0.5\fboxrule}
    Set the interferometer code to use as IFO Two. ifo = IFO code (e.g. 
    L1, H1 or H2).

    \vspace{1ex}

    \end{boxedminipage}

  \textbf{Inherited from AnalysisNode:}
    calibration,
    calibration\_cache\_path,
    get\_calibration,
    get\_end,
    get\_ifo,
    get\_ifo\_tag,
    get\_input,
    get\_start,
    set\_cache,
    set\_end,
    set\_ifo,
    set\_ifo\_tag,
    set\_input,
    set\_output,
    set\_start
    \\
  \textbf{Inherited from CondorDAGNode:}
    \_\_repr\_\_,
    add\_input\_file,
    add\_macro,
    add\_output\_file,
    add\_parent,
    add\_post\_script\_arg,
    add\_pre\_script\_arg,
    add\_var\_arg,
    add\_var\_opt,
    get\_args,
    get\_cmd\_line,
    get\_input\_files,
    get\_opts,
    get\_output\_files,
    job,
    set\_log\_file,
    set\_name,
    set\_post\_script,
    set\_pre\_script,
    set\_retry,
    write\_input\_files,
    write\_job,
    write\_output\_files,
    write\_parents,
    write\_post\_script,
    write\_pre\_script,
    write\_vars
    \index{stochastic \textit{(module)}!StochasticNode \textit{(class)}|)}
    \index{stochastic \textit{(module)}|)}

\clearpage

\section{Stochastic Search Programs}
\label{section:stochastic}

This section of \textsc{LALApps} contains programs that can be used to
search interferometer data for stochastic gravitational wave
backgrounds.

\clearpage
% $Id$

\section{Program \texttt{lalapps\_olapredfcn}}
\label{program:lalapps-olapredfcn}
\idx[Program]{lalapps\_olapredfcn}

\begin{entry}

\item[Name]
%
  \verb$lalapps_olapredfcn$ --- computes overlap reduction function given
  a pair of known detectors.

\item[Synopsis]
%
  \verb$lalapps_olapredfcn $[\verb$-h$]\verb$ $[\verb$-q$]\verb$ $[\verb$-v$]
  \verb$ $[\verb$-d debugLevel $]\verb+ \+\newline
  \verb$   $
  \verb$-s siteID1 $[\verb$-a azimuth1$]
  \verb$-t siteID2 $[\verb$-b azimuth2$]\verb+ \+\newline
  \verb$   $
  [\verb$-f fLow$]\verb$ -e deltaF$\verb$ -n numPoints$\verb$ -o outfile$
                         
\item[Description]
%
  \verb$lalapps_olapredfcn$ computes the overlap reduction function
  $\gamma(f)$ for a pair of known gravitational wave detectors.  It
  uses the LAL function \verb$LALOverlapReductionFunction()$, which is
  documented in the LAL Software Documentation under the
  \texttt{stochastic} package.

\item[Options]\leavevmode
\begin{entry}
\item[\texttt{-h}]
  Print a help message.
\item[\texttt{-q}]
  Run silently (redirect standard input and error to \texttt{/dev/null}).
\item[\texttt{-v}]
  Run in verbose mode.
\item[\texttt{-d} \textit{debugLevel}]
  Set the LAL debug level to \textit{debugLevel}.
\item[\texttt{-s} \textit{siteID1} \texttt{-t} \textit{siteID2}]
  Use detector sites identified by \textit{siteID1} and
  \textit{siteID2}; ID numbers between \texttt{LALNumCachedDetectors}
  (defined in the \texttt{tools} package of LAL) refer to detectors
  cached in the constant array \verb$lalCachedDetectors[]$.  (At this
  point, these are all interferometers.)  Additionally, the five
  resonant bar detectors of the IGEC collaboration can be specified.
  The bar geometry data (summarized in table~\ref{table:cachedBars})
  is used by the fucntion \verb$LALCreateDetector()$ from the
  \texttt{tools} package of LAL to generate the Cartesian position
  vector and response tensor which are used to calculate the overlap
  reduction function.  The ID numbers for the bars depend on the value
  of \texttt{LALNumCachedDetectors}; the correct ID numbers can be
  obtained by with the command
\begin{verbatim}
./lalapps_olapredfcn -h
\end{verbatim}
\item[\texttt{-a} \textit{azimuth1} \texttt{-b} \textit{azimuth2}]
%
  If \textit{siteID1} (\textit{siteID2}) is a bar detector, assume it
  has an azimuth of \textit{azimuth1} (\textit{azimuth2}) degrees East
  of North rather than the default IGEC orientation given in
  table~\ref{table:cachedBars}.  Note that this convention, measuring
  azimuth in degrees clockwise from North is not the same as that used
  in LAL (which comes from the frame spec).  Note also that any
  specified azimuth angle is ignored if the corresponding detector is
  an interferometer.
\item[\texttt{-f} \textit{fLow}]
  Begin the frequency series at a frequency of \textit{fLow}\,Hz; if this
  is omitted, the default value of 0\,Hz is used.
\item[\texttt{-e} \textit{deltaF}]
  Construct the frequency series with a frequency spacing of
  \textit{deltaF}\,Hz
\item[\texttt{-n} \textit{numPoints}]
  Construct a frequency series with \textit{numPoints} points.
\item[\texttt{-o} \textit{outfile}]
  Write the output to file \textit{outfile}.  The format of this file
  is that output by the routine \verb$LALPrintFrequencySeries()$ in
  the \texttt{support} package of LAL, which consists of a header
  describing metadata followed by two-column rows, each containing the
  doublet $\{f,\gamma(f)\}$.
\end{entry}

\begin{table}[tbp]
  \begin{center}
    \begin{tabular}{|c|c|c|c|}
\hline
      Name & Longitude & Latitude & Azimuth
\\ \hline
\verb$AURIGA$ & $11^\circ56'54''$E & $45^\circ21'12''$N & N$44^\circ$E 
\\ \hline
\verb$NAUTILUS$ & $12^\circ40'21''$E & $41^\circ49'26''$N & N$44^\circ$E 
\\ \hline
\verb$EXPLORER$ & $6^\circ12'$E & $46^\circ27'$N & N$39^\circ$E 
\\ \hline
\verb$ALLEGRO$ & $91^\circ10'43.\!\!''766$W & $30^\circ24'45.\!\!''110$N 
& N$40^\circ$W
\\ \hline
\verb$NIOBE$ & $115^\circ49'$E & $31^\circ56'$S & N$0^\circ$E 
\\ \hline
    \end{tabular}
    \caption{Location and orientation data for the five IGEC resonant
      bar detectors, stored in the \texttt{lalCachedBars[]}
      array.  The data are taken from
      \texttt{http://igec.lnl.infn.it/cgi-bin/browser.pl?Level=0,3,1}
      except for the latitude and longitude of ALLEGRO, which were
      taken from Finn \& Lazzarini, gr-qc/0104040.  Note that the
      elevation above the WGS-84 reference ellipsoid and altitude
      angle for each bar is not given, and therefore set to zero.}
    \label{table:cachedBars}
  \end{center}
\end{table}


\item[Example usage]
  To compute the overlap reduction function for LIGO Hanford and
  LIGO Livingston, with a resolution of 1\,Hz from 0\,Hz to 1024\,Hz:
\begin{verbatim}
lalapps_olapredfcn -s 0 -t 1 -e 1 -n 1025 -o LHOLLO.dat
\end{verbatim}
  
  To compute the overlap reduction function for ALLEGRO in its optimal
  orientation of $72.\!\!^\circ08$ West of South (see Finn \& Lazzarini,
  gr-qc/0104040) and LIGO Livingston, with a resolution of 0.5\,Hz from
  782.5\,Hz to 1032\,Hz (assuming \texttt{lalNumCachedBars} is 6):
\begin{verbatim}
lalapps_olapredfcn -s 9 -a 252.08 -t 1 -f 782.5 -e 0.5 -n 500 -o ALLEGROLHO.dat
\end{verbatim}

\item[Author]
John T.~Whelan

\end{entry}


\clearpage
\subsection{Program \texttt{lalapps\_stochastic\_pipe}}
\label{program:stochastic-pipeline}
\idx[Program]{stochastic\_pipeline.py}

\begin{entry}
\item[Name]
\verb$lalapps_stochastic_pipe$ --- python script to generate Condor DAGs
to run the stochastic pipeline.

\item[Synopsis]
\begin{verbatim}
  -h, --help               display this message
  -v, --version            print version information and exit

  -d, --datafind           run LSCdataFind to create frame cache files
  -s, --stochastic         run lalapps_stochastic

  -P, --priority PRIO      run jobs with condor priority PRIO

  -f, --config-file FILE   use configuration file FILE
  -l, --log-path PATH      directory to write condor log file
\end{verbatim}

\item[Description] \verb$lalapps_stochastic_pipe$ generates a Condor DAG
to run the stochastic search pipeline. The configuration should specify
the parameters needed to run the jobs and must be specified with the
\verb$--config-file$ option. A file containing science segments to
analysed should be specified in the \verb$[input]$ section of the
configuration file with a line such as
\begin{verbatim}
segments = S2H1L1v03_selectedsegs.txt
\end{verbatim}
This should contain four whitespace separated columns:
\begin{verbatim}
  segment_id    gps_start_time    gps_end_time    duration
\end{verbatim}
that define the science segments to be used. Lines starting with an
octothorpe are ignored.

\item[Example]
Generate a DAG to run a stochastic search on a pair of interferometers
specified in the configuration file. The generated DAG is then submitted
with \texttt{condor\_submit\_dag}
\begin{verbatim}
lalapps_inspiral_pipe --log-path /home/ram/dag_logs \
--datafind --stochastic --config-file stochastic_H1L1.ini

condor_submit_dag stochastic_H1L1.dag
\end{verbatim}

\item[Author]
Adam Mercer
\end{entry}

\clearpage
\subsection{Program \texttt{lalapps\_stochastic}}
\label{program:lalapps-stochastic-dev}
\idx[Program]{lalapps\_stochastic\_dev}

\begin{entry}
\item[Name]
\verb$lalapps_stochastic_dev$ --- standalone stochastic analysis code.

\item[Synopsis]
\begin{verbatim}
 --help                        print this message
 --version                     display version
 --verbose                     verbose mode
 --debug                       save out intermediate products
 --debug-level N               set lalDebugLevel
 --gps-start-time N            GPS start time
 --gps-end-time N              GPS end time
 --segment-duration N          segment duration
 --sample-rate N               sample rate
 --resample-rate N             resample rate
 --f-min N                     minimal frequency
 --f-max N                     maximal frequency
 --ifo-one IFO                 ifo for first stream
 --ifo-two IFO                 ifo for second stream
 --frame-cache-one FILE        cache file for first stream
 --frame-cache-two FILE        cache file for second stream
 --calibration-cache-one FILE  first stream calibration cache
 --calibration-cache-two FILE  second stream calibration cache
 --apply-mask                  apply frequency masking
 --mask-bin N                  number of bins to mask
 --overlap-hann                overlaping hann windows
 --hann-duration N             hann duration
 --high-pass-filter            apply high pass filtering
 --hpf-frequency N             high pass filter knee frequency
 --hpf-attenuation N           high pass filter attenuation
 --hpf-order N                 high pass filter order
 --filter-omega-alpha N        omega_gw exponent
 --filter-omega-fref N         omega_gw reference frequency
 --filter-omega0 N             omega_0
\end{verbatim}

\item[Description] \verb$lalapps_stochastic$ runs the standalone
stochastic analysis code.

\item[Author] 
Adam Mercer, Tania Regimbau
\end{entry}
