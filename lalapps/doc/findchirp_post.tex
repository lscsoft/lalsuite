\section{Program \texttt{lalapps\_findchirp\_post}}
\label{program:lalapps-findchirp-post}
\idx[Program]{lalapps\_findchirp\_post}

\begin{entry}

\item[Name]
\verb$lalapps_findchirp_post$ --- post process event files from findchirp

\item[Synopsis]
\verb$lalapps_inspinj$ [\verb$-h$] [\verb$-V$] [\verb$-v$]
[\verb$-d$ \textit{dbglvl}]
[\verb$-m$]
[\verb$-s$ \textit{minsnr}\texttt{-}\textit{maxsnr}]
[\verb$-c$ \textit{minchisq}\texttt{-}\textit{maxchisq}]
[\verb$-b$ \textit{bins}]
[\verb$-e$ \textit{eventfile}]

\item[Description]

\verb$lalapps_inspinj$ post process event files from findchirp.

\item[Options]\leavevmode
\begin{entry}
\item[\texttt{-h}]
Print a help message.
\item[\texttt{-V}]
Print the version information.
\item[\texttt{-v}]
Verbose output.
\item[\texttt{-d} \textit{dbglvl}]
Set LAL debug level to \textit{dbglvl}.
\item[\texttt{-m}]
Maximize events over injections. This options caused only the loudest event
(in SNR) in each window of length $100/\pi = 31.8309886183791$ to be processed.
The window starts at the time of the first event.
\item[\texttt{-s} \textit{minsnr}\texttt{-}\textit{mmaxsnr}]
Set minimum and maximum of the range of signal to noise ratio for creation
of the histogram bins.  (Default is 7.0\-14.0.)
\item[\texttt{-c} \textit{minchisq}\texttt{-}\textit{mmaxchisq}]
Set minimum and maximum of the range of chi squared veto statistic for
creation of the histogram bins.  (Default is 0.0\-50.0.)
\item[\texttt{-b} \textit{bins}]
Set the number of bins in the two dimensional histogram. (Default is 10.)
\item[\texttt{-e} \textit{eventfile}]
Set the path to the file containing the list of inspiral events. 
(Default is \verb|eventfile.dat|.) The event file must be of the format
\verb|END_TIME END_TIME_NS EFF_DISTANCE MASS1 MASS2 MCHIRP ETA SNR CHISQ SIGMASQ|.
Each event should be separated by a new line. Lines beginning with an
octothorpe are ignored.
\end{entry}

\item[Debug levels]
The LAL debug level can be specified as an integer or as a string of flags:
\begin{entry}
\item[\texttt{NDEBUG}]
No debugging information is printed and memory debugging code is disabled.
\item[\texttt{ERROR}]
Error messages are printed.
\item[\texttt{WARNING}]
Warning messages are printed.
\item[\texttt{INFO}]
Information messages are printed.
\item[\texttt{TRACE}]
Function call tracing messages are printed.
\item[\texttt{MEMINFO}]
Memory  allocation  information messages are printed.
\item[\texttt{MEMDBG}]
Debugging of memory allocation routines is enabled but no messages are printed.
\end{entry}
The following composite levels are available:
\begin{entry}
\item[\texttt{MSGLVL1}]
Equivalent to \verb$ERROR$
\item[\texttt{MSGLVL2}]
Equivalent to \verb$ERROR | WARNING$
\item[\texttt{MSGLVL3}]
Equivalent to \verb$ERROR | WARNING | INFO$
\item[\texttt{ALLDBG}]
All debugging messages are printed.
\end{entry}

For example, the command
\begin{indented}
\verb$lalapps_inspinj -d "ERROR | INFO"$
\end{indented}
will set the debug level so that error and information messages are printed.

\item[Environment]\leavevmode

\begin{entry}
\item[\texttt{LAL\_DEBUG\_LEVEL}]
Default LAL debug level to use.
\end{entry}

\item[Author]
Duncan Brown

\end{entry}

