\section{Program \texttt{lalapps\_thinca}}
\label{program:lalapps-thinca}
\idx[Program]{lalapps\_thinca}

\begin{entry}
\item[Name]
\verb$lalapps_thinca$ --- program does the inspiral coincidence analysis.

\item[Synopsis]
\prog{lalapps\_thinca} \newline \hspace*{0.5in}
[\option{--help}] \newline \hspace*{0.5in} 
[\option{--verbose}] \newline \hspace*{0.5in} 
[\option{--version}] \newline \hspace*{0.5in}
\option{--debug-level}~\parm{level} \newline \hspace*{0.5in} 
[\option{--user-tag}~\parm{usertag}] \newline \hspace*{0.5in} 
\option{--gps-start-time}~\parm{start\_time} \newline \hspace*{0.5in}           
\option{--gps-end-time}~\parm{end\_time} \newline \hspace*{0.5in} 
[\option{--check-times}] \newline \hspace*{0.5in} 
[\option{--g1-triggers}]  \newline \hspace*{0.5in}
[\option{--h1-triggers}]  \newline \hspace*{0.5in}
[\option{--h2-triggers}]  \newline \hspace*{0.5in}
[\option{--l1-triggers}]  \newline \hspace*{0.5in}
[\option{--t1-triggers}]  \newline \hspace*{0.5in}
[\option{--v1-triggers}]  \newline \hspace*{0.5in}
[\option{--g1-time-accuracy}~\parm{g1\_dt}]  \newline \hspace*{0.5in}
[\option{--h1-time-accuracy}~\parm{h1\_dt}]  \newline \hspace*{0.5in}
[\option{--h2-time-accuracy}~\parm{h2\_dt}]  \newline \hspace*{0.5in}
[\option{--l1-time-accuracy}~\parm{l1\_dt}]  \newline \hspace*{0.5in}
[\option{--t1-time-accuracy}~\parm{t1\_dt}]  \newline \hspace*{0.5in}
[\option{--v1-time-accuracy}~\parm{v1\_dt}]  \newline \hspace*{0.5in}
[\option{--g1-mass-accuracy}~\parm{g1\_dm}]  \newline \hspace*{0.5in}
[\option{--h1-mass-accuracy}~\parm{h1\_dm}]  \newline \hspace*{0.5in}
[\option{--h2-mass-accuracy}~\parm{h2\_dm}]  \newline \hspace*{0.5in}
[\option{--l1-mass-accuracy}~\parm{l1\_dm}]  \newline \hspace*{0.5in}
[\option{--t1-mass-accuracy}~\parm{t1\_dm}]  \newline \hspace*{0.5in}
[\option{--v1-mass-accuracy}~\parm{v1\_dm}]  \newline \hspace*{0.5in}
[\option{--g1-mchirp-accuracy}~\parm{g1\_dmchirp}]  \newline \hspace*{0.5in}
[\option{--h1-mchirp-accuracy}~\parm{h1\_dmchirp}]  \newline \hspace*{0.5in}
[\option{--h2-mchirp-accuracy}~\parm{h2\_dmchirp}]  \newline \hspace*{0.5in}
[\option{--l1-mchirp-accuracy}~\parm{l1\_dmchirp}]  \newline \hspace*{0.5in}
[\option{--t1-mchirp-accuracy}~\parm{t1\_dmchirp}]  \newline \hspace*{0.5in}
[\option{--v1-mchirp-accuracy}~\parm{v1\_dmchirp}]  \newline \hspace*{0.5in}
[\option{--g1-eta-accuracy}~\parm{g1\_deta}]  \newline \hspace*{0.5in}
[\option{--h1-eta-accuracy}~\parm{h1\_deta}]  \newline \hspace*{0.5in}
[\option{--h2-eta-accuracy}~\parm{h2\_deta}]  \newline \hspace*{0.5in}
[\option{--l1-eta-accuracy}~\parm{l1\_deta}]  \newline \hspace*{0.5in}
[\option{--t1-eta-accuracy}~\parm{t1\_deta}]  \newline \hspace*{0.5in}
[\option{--v1-eta-accuracy}~\parm{v1\_deta}]  \newline \hspace*{0.5in}
\option{--parameter-test}
\parm{(m1\_and\_m2|psi0\_and\_psi3|mchirp\_and\_eta)}
\newline \hspace*{0.5in}
\option{--data-type}~\parm{(playground\_only|exclude\_play|all\_data)} 
\newline \hspace*{0.5in}
\option{LIGOLW XML input files} 

\item[Description --- General] 

\verb$lalapps_thinca$ performs a coincidence test between triggers from
different interferometers.  It reads in triggers from up to four instruments
and returns coincident triggers.  At present, the code returns all double
coincident triggers --- it does not check for triple or quadruple coincidences
even if there are more than two active detectors.

The user specifies which instruments there will be triggers input from with
the \texttt{g1-triggers}, \texttt{h1-triggers} etc. options.  If less than two
of these are specified, the program exits as there cannot be coincidence.  The
triggers are then read in from the list of LIGO Lightweight XML files given
after the last command line argument.  The code only keeps triggers which
occur between the \textsc{start\_time} and the \textsc{end\_time}.  If the
\texttt{check-times} option is specified, then the input search summary tables
are checked to ensure that we have searched all data between the
\textsc{start\_time} and \textsc{end\_time} in all relevant ifos.  Following
this, we discard the non-playground triggers if \textsc{playground\_only} was
specified and any playground triggers if \textsc{exclude\_play} was specified.
At this stage, we check that there are triggers from at least two instruments,
if not, the code exits without testing for coincidences. 

The code now tests for any pairs of coincident triggers.  This is done in the
function \texttt{LALCreateTwoIFOCoincList()}.  Triggers are considered
coincident if their end times and mass parameters pass coincidence.  We test
either on the two component masses if \texttt{m1\_and\_m2} is specified or the
chirp mass and mass ratio if \texttt{mchirp\_and\_eta} is specified.  To pass
time coincidence, the end times must differ by less than \textsc{ifoa\_dt} $+$
\textsc{ifob\_dt} $+$ light travel time.  Similarly, we require that the mass
parameters agree within \textsc{ifoa\_dm} $+$ \textsc{ifob\_dm}.
At the end of the process, we
have a list of pairs of triggers from different instruments which pass the
time and mass coincidence tests.  These need to be searched for triple and
quadruple coincidences, but this has not yet been implemented.

The coincident triggers are written into a single LIGO Lightweight XML file.
In order that the coincident triggers can be easily located, the
\textsc{event\_id} field is populated.  This is a \texttt{UINT8} which is
populated with \textsc{start\_time} $\times 10^{9} +$ an integer identifier.
The integer identifier is unique within the file, so the overall \textsc{id}
will be unique.  

The output file is named
\begin{center}
\texttt{IFOS-THINCA\_USERTAG-GPSSTARTTIME-DURATION.xml}\\
\end{center}
where \textsc{IFOS} is a list of the active ifos in alphabetical order.  The
file contains \texttt{process}, \texttt{process\_params},
\texttt{search\_summvars} and \texttt{search\_summary} tables that describe
the search.  Additionally there is a \texttt{summ\_value} table which contains
the summ values which were contained in the input files (in aniticipation of
performing a distance cut) as well as the \texttt{sngl\_inspiral} table
containing the coincident events.

\item[Options]\leavevmode
\begin{entry}

\item[\texttt{--data-type}(playground\_only|exclude\_play|all\_data)]
Required.  Specify whether the code should use only the playground, exclude
the playground or use all the data. 

\item[\texttt{--gps-start-time} \textsc{start\_time}] Required.  Look
for coincident triggers with end times after \textsc{start\_time}.

\item[\texttt{--gps-end-time} \textsc{end\_time}] Required.  Look for
coincident triggers with end times before \textsc{end\_time}.


\item[\texttt{--g1-triggers}] Optional.  Specify that triggers from G1 will be
provided.

\item[\texttt{--h1-triggers}] Optional.  Specify that triggers from H1 will be
provided.
\item[\texttt{--h2-triggers}] Optional.  Specify that triggers from H2 will be
provided.

\item[\texttt{--l1-triggers}] Optional.  Specify that triggers from L1 will be
provided.
\item[\texttt{--t1-triggers}] Optional.  Specify that triggers from T1 will be
provided.
\item[\texttt{--v1-triggers}] Optional.  Specify that triggers from V1 will be
provided.  Note: while having triggers from each of the instruments is
optional, the code requires triggers from at least two instruments, otherwise
it is impossible to do coincidence.

\item[\texttt{--check-times}] Optional.  If this flag is set, the code checks
the input search summary tables to verify that the data for each of the
requested interferometers was analyzed once and only once between the
\textsc{start\_time} and \textsc{end\_time}.  By default, the code will not
perform this check.

\item[\texttt{--parameter-test}
(m1\_and\_m2|psi0\_and\_psi3|mchirp\_and\_eta)]
Required. Choose which parameters to use when testing for coincidenc.
Depending on which test is chosen, the allowed windows on the appropriate
parameters should be set as described below.

\item[\texttt{--ifo-time-accuracy} \textsc{ifo\_dt}] Required for any ifo for
which we have triggers. Set the accuracy with which the given \texttt{ifo} can
recover the end time of a signal.  The timing accuracy is specified in
milliseconds. Here, \texttt{ifo} is one of g1, h1, h2, l1, t1, v1.

\item[\texttt{--ifo-mass-accuracy} \textsc{ifo\_dm}] Optional. Set the
accuracy with which the given \texttt{ifo} can recover the component masses of
a signal.  The mass accuracy is set in solar masses.

\item[\texttt{--ifo-mchirp-accuracy} \textsc{ifo\_dmchirp}] Optional. Set the
accuracy with which the given \texttt{ifo} can recover the chirp mass of a
signal.  The chirp mass accuracy is set in solar masses.

\item[\texttt{--ifo-eta-accuracy} \textsc{ifo\_deta}] Optional. Set the
accuracy with which the given \texttt{ifo} can recover the mass ratio $\eta$
of a signal.

\item[\texttt{--comment} \textsc{string}] Optional. Add \textsc{string}
to the comment field in the process table. If not specified, no comment
is added. 

\item[\texttt{--user-tag} \textsc{usertag}] Optional. Set the user tag for
this job to be \textsc{usertag}. May also be specified on the command line as
\texttt{-userTag} for LIGO database compatibility.  This will affect the
naming of the output file.

\item[\texttt{--verbose}] Enable the output of informational messages.

\item[\texttt{--help}] Optional.  Print a help message and exit.

\item[\texttt{--version}] Optional.  Print out the author, CVS version and
tag information and exit.

\item[\texttt{--debug-level} \textsc{level}] Optional. Set the LAL debug
level to \textsc{level}. If not specified the default is 1.

\end{entry}

\item[Arguments]\leavevmode
\begin{entry}
\item[\texttt{[LIGO Lightweight XML files]}] The arguments to the program
should be a list of LIGO Lightweight XML files containing the triggers from
the two interferometers. The input files can be in any order and do not need
to be time ordered as \texttt{thinca} will sort all the triggers once they are
read in. If the program encounters a LIGO Lightweight XML containing triggers
from an unknown interferometer (i.e. not IFO A or IFO B) it will exit with an
error.
\end{entry}

\item[Example]
\begin{verbatim}
lalapps_thinca \
--data-type playground_only --h1-triggers --h2-triggers --l1-triggers \
--h1-time-accuracy 1 --h2-time-accuracy 1.5 --l1-time accuracy 1 \
--parameter-test mchirp_and_eta --h1-mchirp-accuracy 0.02 \
--h2-mchirp-accuracy 0.03 --l1-mchirp-accuracy 0.04 \ 
--h1-eta-accuracy 1 --h2-eta-accuracy 1 --l1-eta-accuracy 1 \
--gps-start-time 777001000 --gps-end-time 777002000 \
H1-INSPIRAL-777000500-2048.xml H2-INSPIRAL-777000700-2048.xml \
L1-INSPIRAL-777000500-2048.xml
\end{verbatim}

\item[Algorithm]
Not yet documented.


\item[Author] 
Steve Fairhurst
\end{entry}


