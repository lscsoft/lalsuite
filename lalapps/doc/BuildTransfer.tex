\section{Program \texttt{lalapps\_BuildTransfer}}
\label{program:lalapps-BuildTransfer}
\idx[Program]{lalapps\_BuildTransfer}

\begin{entry}

\item[Name]
\verb$lal_BuildTransfer$ --- computes instrumental response function given the 
transfer function between EXC\_ETM and AS\_Q.   

\item[Synopsis]
\verb$lal_BuildTransfer$ [\verb$-r tffile$] [\verb$-a ETM calibration$]
                         [\verb$-f npoints fmin fmax$] [\verb$-t$]
                         [\verb$-o outfile$] [\verb$-i ilwdfile$]
                         
\item[Description]
\verb$lal_BuildTransfer$ computes the frequency domain response
function $R(f)$ given the measured swept-sine data and the ETM
calibration (normalized to units of strain). See the LAL Software
Documentation under \texttt{tools} for the definition of the transfer
function.

\item[Options]\leavevmode
\begin{entry}
\item[\texttt{-h}]
Print a help message.
\item[\texttt{-t}]
If this flag is used,  then the ilwd will indicate units of
counts/attostrain.  The user must correctly choose the zeros, poles,
and gains,  it is not handled automatically.
\item[\texttt{-f npoints fmin fmax}] 
Information about the frequency series.  \verb$npoints$ is the number
of points in the series,  \verb$fmin$ is the lowest frequency to be
represented,  and \verb$fmax$ is the highest frequency to be
represented.   
\item[\texttt{-o} \textit{outfile}]
Write the output to file \textit{outfile} as 3 columns.  Each row
represents the triplet: $\{f,\; \textrm{Re}[T(f)],\; \textrm{Im}[T(f)]\}$.
\item[\texttt{-i} \textit{outfile}]
Write the output to file \textit{outfile} in ILWD format suitable for
ingestion by the datacondAPI in LDAS.
\end{entry}

\item[Example usage]
To compute the transfer function 4 zeros at $0,0,0.74,0.74$Hz,  5
poles at $50.35,50.35,12.1,12.1,186.0$Hz,  and gain $1.47e7$ at 1025
sample points from $0.0$Hz up to $1024.0$Hz and write the ouput as
ILWD:
\begin{verbatim}
./lalapps_ZPG2Transfer -z 4 0.0 0.0 0.74 0.74 \
   -p 5 50.35 50.35 12.1 12.1 186.0 -g 1.47e7 \
   -f 1025 0.0 1024.0 -o response.txt
\end{verbatim}

\item[Author]
Patrick Brady

\end{entry}
