f\subsection{Program \texttt{lalapps\_BankEfficiency}}
\label{program:lalapps-BankEfficiency}
\idx[Program]{lalapps\_BankEfficiency}
{\LARGE{{\bf Documentation In progress}}}

\begin{entry}

\item[Name]
\verb$lalapps_BankEfficiency$ --- produces Monte Carlo simulation to test the bank efficiency
code in the framework of inspiral binaries. It is dedicated to BCV bank validation tough 
SPA bank might be tested as well.

\item[Synopsis]
\verb$lalapps_BankEfficiency$ [\verb$-h$] 
[\verb$--alpha-bank$   \textit{$\alpha_{bank}$}]
[\verb$--alpha-signal$ \textit{$\alpha_{signal}$}]
[\verb$--fend-bcv$     \textit{$f_{cut}Min$}\texttt{\&}\textit{$f_{cut}Max$}]
[\verb$--fl$	\textit{lower cutoff frequency}]
[\verb$--mm$ \textit{MinimalMatch}]
[\verb$--mass-range$ \textit{mMin}\texttt{\&}\textit{mMax}]
[\verb$--n$     \textit{number of simulations}]  - 
[\verb$--number-fcut$ \textit{number of layers } \textit{ in $f_{cut}$ dimension}]
[\verb$--print-bank$]
[\verb$--psi0-range$ \textit{$\psi0_{Min}$}\texttt{\&}\textit{$\psi0_{Max}$}]
[\verb$--psi3-range$ \textit{$\psi3_{Min}$}\texttt{\&}\textit{$\psi3_{Max}$}]
[\verb$--s$ \textit{seed}]
[\verb$--sampling$ \textit{sampling frequency}]
[\verb$--signal$ \textit{injected signal approximant}]
[\verb$--signal-order$ \textit{injected signal order}]
[\verb$--template$ \textit{template approximant}]





\item[Description]
\verb$lalapps_BankEfficiency$ This application create a bank of templates based on the
SPA bank or BCV bank. Then, it performs a Monte Carlo Simulation by injecting known signals
such as TaylorT1, PadeT1 or EOB and computing the overlap between the signal and the 
bank of templates.
 
The argument \verb$--template$  decides which bank has to be used: BCV approximant leads 
to BCV templates and BCV bank whereas any other approximant leads to a SPA bank but with 
templates based on the given approximant. 

If BCV bank is chosen, then overlap is based on BCV and takes into account the alpha
maximization. If SPA bank is chosen then the filter in quadrature is used.

In the case of BCV bank, the user have to provide the following parameters: 

\begin{itemize}
\item Minimal Match \verb$--mm 0.97$.
\item $\alpha$ parameter \verb$--alpha-bank 0.01$ .
\item a range of $\psi_0$ \verb$--psi0-range 1 100000$.
\item a range of $\psi_3$ \verb$--psi3-range -1 -3500$.
\end{itemize}
 and possibly
\begin{itemize}
\item number of layers \verb$--number-fcut 5$ in the $f_{cut}$ dimension for the bank generation. 
\item range of $f_{cut}$ \verb$--fend-bcv 4 8$.
\item cutoff frequency for the moment computation \verb$--fcut-moment$
\end{itemize}

For the time being the Bank code is provided by the bank package through the function
\texttt{LALCreateCoarseBank}. If the option "template" is BCV then the bank will be a BCV
 bank using the fucntion \texttt{LALInspiralBCVFcutBank} (see bank package for details).





\item[Options related to the simulations]\leavevmode
\begin{entry}
\item[\texttt{--alpha-bank}]
alpha parameter to compute the metric.
\item[\texttt{--alpha-signal}]
alpha parameter used for generating the  injection of BCV waveform.
\item[\texttt{--fend-bcv}]
Range of cutoff frequency for the BCV template creation. User must provide two parameters: 
the lowest and upper frequency cutoff. Those frequency are not expressed in Hz but in term
of distance betwwen the two objets. for instance at the last stable orbit, the distance 
between the two objects is 6GM. In the case of an EOB description, the distance between 
the two objects is closer to 3GM. A good compromise is to use fend-bcv between 8GM and 3GM
and using the options --number-fcut = 5. 
\item[\texttt{--fl}]
Lower cutoff frequency for templates and signal. Might add two options later. 
\item[\texttt{--h}]
Print a help message.
\item[\texttt{--mm}]
Fixe the Minimal Match of the bank.
\item[\texttt{--mass-range}]
Mass range of the signal to inject (minimum and maximum values). One has to adjust
 the parameters of the bank generation. For instance, if the minimal mass is 3 solar
mass then psi0 max should be of the order of 500.000.
\item[\texttt{--n}]
Number of trials, i.e number of injections.
\item[\texttt{--number-fcut}]
Number of layers to create in the BCV bank in the fend-bcv dimension. 
Given the lower and upper frequency (see option fend-bcv), the bank creates
n layers each of them having a frequency between lowest fend-bcv and highest fend-bcv.
\item[\texttt{--psi0-range}]
Range of paramter psi0. User must give two positive numbers.
\item[\texttt{--psi3-range}]
Range of paramter psi3. User must give two negative numbers.
\item[\texttt{--sampling}]
Sampling frequency.
\item[\texttt{--seed}]
Random seed.
\item[\texttt{--signal}]
Approximant of the injected signal. See LAL documentation for details. It must be a string
such as TaylorT1, TaylorT2, EOB ...
\item[\texttt{--signal-order}]
Order of the injected waveforms.
\item[\texttt{--template}]
Template to be used for the detection. If BCV if given then both bank and templates will be
based on BCV. If the approximant is not BCV then the bank is a SPA bank of templates and the 
templates corresponds to the given argument.
\end{entry}




\item[Options related to output]\leavevmode
\begin{entry}
\item[\texttt{--print-bank}]
Print the bank of templates on stdout. If a BCV-template bank is used then the ouput is composed of
5 columns: psi0, psi3, number of layer, total mass and final frequency. If it is a SPA bank
then 4 columns are print: tau0, tau3, and corresponding mass1 and mass2. The number of templates is 
also printed.

\end{entry}

\item[Example]\leavevmode
The following command
\begin{verbatim}
lalapps_BankEfficiency --psi0-range 1 100000 --psi3-range -1 -3500 -n 1000 
--template BCV --signal TaylorT1 --alpha-bank 0 -mm 0.97
\end{verbatim}

will compute overlaps between injected TaylorT1 waveforms and a BCV abnk of templates. 


\item[Author]
Thomas Cokelaer

\end{entry}
