\section{Program \texttt{lalapps\_trigbank}}
\label{program:lalapps-trigbank}
\idx[Program]{lalapps\_trigbank}

\begin{entry}
\item[Name]
\verb$lalapps_trigbank$ --- program to make triggered template banks.

\item[Synopsis]
\prog{lalapps\_trigbank} \newline \hspace*{0.5in}
[\option{--help}] \newline \hspace*{0.5in}
[\option{--verbose}] \newline \hspace*{0.5in}
[\option{--version}] \newline \hspace*{0.5in}
[\option{--user-tag}~\parm{usertag}] \newline \hspace*{0.5in}
[\option{--ifo-tag}~\parm{ifotag}] \newline \hspace*{0.5in}
[\option{--comment}] \newline \hspace*{0.5in}
\option{--gps-start-time}~\parm{start\_time} \newline \hspace*{0.5in}
\option{--gps-end-time}~\parm{end\_time} \newline \hspace*{0.5in}   
[\option{--check-times}] \newline \hspace*{0.5in}
\option{--input-ifo}~\parm{inputifo} \newline \hspace*{0.5in}
\option{--output-ifo}~\parm{outputifo} \newline \hspace*{0.5in}
\option{--parameter-test}~\parm{(m1\_and\_m2|psi0\_and\_psi3|mchirp\_and\_eta)}
\newline \hspace*{0.5in}
\option{--data-type}~\parm{(playground\_only|exclude\_play|all\_data)} 
\newline \hspace*{0.5in}
\option{LIGOLW XML input files}


\textsc{(LIGO Lightweight XML files)}

\item[Description --- General] 

\verb$lalapps_trigbank$ produces a triggered template bank from the input xml
files.  The code takes in a list of triggers from the specified input files.
First, it discards any triggers which do not come from the specified
\textsc{input\_ifo}.  Then, it keeps only those triggers which occur between
the \textsc{start\_time} and the \textsc{end\_time}. If the
\texttt{check-times} option is specified, then the input search summary tables
are checked to ensure that we have searched all data between the
\textsc{start\_time} and \textsc{end\_time} in all relevant ifos.  Following
this, we discard the non-playground triggers if \textsc{playground\_only} was
specified and any playground triggers if \textsc{exclude\_play} was specified.

The remaining triggers are sorted according to the given
\texttt{parameter-test}, which must be one of \texttt{m1\_and\_m2} or
\texttt{psi0\_and\_psi3}.  Duplicate templates (those with identical m1
and m2 or psio and psi3) are discarded and what
remains is output as the triggered template bank.  


The output file contains \texttt{process}, \texttt{process\_params},
\texttt{search\_summary} and \texttt{sngl\_inspiral} tables.  The output
file name is 
\begin{center}
\texttt{OUTPUTIFO-TRIGBANK\_IFOTAG\_USERTAG-GPSSTARTTIME-DURATION.xml}\\
\end{center}
where the input and output instruments are specified on the command line.

\item[Options]\leavevmode
\begin{entry}

\item[\texttt{--data-type} (playground\_only|exclude\_play|all\_data)]
Required.  Specify whether the code should use only the playground, exclude
the playground or use all the data. 

\item[\texttt{--gps-start-time} \textsc{start\_time}] Required.  Look
for coincident triggers with end times after \textsc{start\_time}.

\item[\texttt{--gps-end-time} \textsc{end\_time}] Required.  Look for
coincident triggers with end times before \textsc{end\_time}.

\item[\texttt{--check-times}] Optional.  If this flag is set, the code checks
the input search summary tables to verify that the data for the input
interferometers was analyzed once and only once between the
\textsc{start\_time} and \textsc{end\_time}.  By default, the code will not
perform this check.

\item[\texttt{--parameter-test} (m1\_and\_m2|psi0\_and\_psi3)] Required. Choose
which parameters to use when testing for repeated triggers.  If two triggers
have the same value of \texttt{mass1} and \texttt{mass2} or \texttt{psi0} and
\texttt{psi3} then only one copy is put into the triggered bank.

\item[\texttt{--input-ifo} \textsc{inputifo}] Required. The triggers from
\textsc{INPUTIFO} are uset to create the triggered bank.

\item[\texttt{--output-ifo} \textsc{outputifo}] Required. This gives the
instrument for which we are creating the triggered bank.  It is only used in
naming the output file.

\item[\texttt{--ifo-tag} \textsc{ifotag}] Optional.  Used in naming the output
file.

\item[\texttt{--comment} \textsc{string}] Optional. Add \textsc{string}
to the comment field in the process table. If not specified, no comment
is added. 

\item[\texttt{--user-tag} \textsc{usertag}] Optional. Set the user tag for
this job to be \textsc{usertag}. May also be specified on the command line as
\texttt{-userTag} for LIGO database compatibility.  This will affect the
naming of the output file.

\item[\texttt{--verbose}] Enable the output of informational messages.

\item[\texttt{--help}] Optional.  Print a help message and exit.

\item[\texttt{--version}] Optional.  Print out the author, CVS version and
tag information and exit.

\end{entry}

\item[Arguments]\leavevmode
\begin{entry}
\item[\texttt{[LIGO Lightweight XML files]}] The arguments to the program
should be a list of LIGO Lightweight XML files containing the triggers from the
two interferometers. The input files can be in any order and do not need to be
time ordered as \texttt{trigbank} will sort all the triggers once they are read
in. 
\end{entry}

\item[Example]
\begin{verbatim}
lalapps_trigbank \
--data-type playground_only --input-ifo H1 --output-ifo H1 --ifo-tag H1H2 \
--gps-start-time 734323079 --gps-end-time 734324999 \
H1-INSPIRAL-734323015-2048.xml
\end{verbatim}

\item[Algorithm]
None.

\item[Author] 
Patrick Brady, Duncan Brown and Steve Fairhurst
\end{entry}


