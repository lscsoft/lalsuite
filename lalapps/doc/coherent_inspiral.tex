\section{Program \texttt{lalapps\_coherent\_inspiral}}
\label{program:lalapps-coherent-inspiral}
\idx[Program]{lalapps\_coherent\_inspiral}

\begin{entry}
\item[Name]
\verb$lalapps_coherent_inspiral$ --- coherent search code

\item[Synopsis] \prog{lalapps\_inspiral} \newline \hspace*{0.5in}
[\option{--help}] \newline \hspace*{0.5in} 
[\option{--verbose}] \newline \hspace*{0.5in} 
[\option{--version}] \newline \hspace*{0.5in}
\option{--channel-number}~\parm{CHANNUMBER} \newline \hspace*{0.5in}           
\option{--bank-file}~\parm{FILE} \newline \hspace*{0.5in}           
\option{--sample-rate}~\parm{F} \newline \hspace*{0.5in}         
\option{--segment-length}~\parm{N} \newline \hspace*{0.5in}            
\option{--low-frequency-cutoff}~\parm{F} \newline \hspace*{0.5in}            
\option{--cohsnr-threshold}~\parm{RHO} \newline \hspace*{0.5in} 
[\option{--output-path}~\parm{PATH}] \newline \hspace*{0.5in}              
[\option{--write-cohsnr}] \newline \hspace*{0.5in}
[\option{--maximize-over-chirp}] \newline \hspace*{0.5in}
\option{--H1-framefile}~\parm{FILE} \newline \hspace*{0.5in}
\option{--H2-framefile}~\parm{FILE} \newline \hspace*{0.5in}
\option{--L-framefile}~\parm{FILE} \newline \hspace*{0.5in}
\option{--V-framefile}~\parm{FILE} \newline \hspace*{0.5in}
\option{--G-framefile}~\parm{FILE} \newline \hspace*{0.5in}
\option{--T-framefile}~\parm{FILE} \newline \hspace*{0.5in}              

\item[Description] 
\prog{lalapps\_coherent\_inspiral} is a code that optimally combines 
pre-filtered data for up to 4 detectors to produce an event table in the form 
of an xml file as well as the coherent SNR time series in the form of a frame 
file if the user so desires.  The user must provide 2-4 frame files 
(1 for each detector in the network) that contain the C data for each detector.
To produce these files, run inspiral.c on the raw data for 2-4 detectors 
(1 at a time of course) with the --write-cdata option specified. Each of these
raw data segments should be filtered with the same template for a meaningful 
coherent SNR. For some networks with 3 or 4 detectors, the coherent code 
requires that the beam-pattern coefficient files be in the same directory where
the code is executed.  These files were generated by a mathematica notebook  
and their filenames are: HBeam.dat, LBeam.dat, VIRGOBeam.dat, TAMABeam.dat, 
and GEOBeam.dat.     

\item[Options]\leavevmode
\begin{entry}
\item[\option{--help}] Optional. Display a brief usage summary.

\item[\option{--verbose}] Optional. Print progress information.

\item[\option{--version}] Optional. Print version information and exit.

\item[\option{--channel-number}~\parm{CHANNUMBER}] Read data from 
C data channel number \parm{CHANNUMBER}. After running inspiral.c with the 
--write-cdata option specified, the output frame file will contain a channel 
for each segment that the raw data is broken up into as it is filtered by 
inspiral.c. Here, the user specifies the number on the end of said channel, or in effect the segment number of the filtered data.  (e.g. the 5 in L1:LSC-AS\_Q\_CData\_5) 

\item[\option{--bank-file}~\parm{FILE}] Read template bank parameters 
from file \parm{FILE}. The raw data for each detector should have been filtered
with the same bank for each detector using inspiral.c.

\item[\option{--sample-rate}~\parm{F}] The sample rate that was used to 
pre-filter the raw data to obtain the C data frame files using inspiral.c.  

\item[\option{--segment-length}~\parm{N}]  The data segment length in terms of \parm{N}, the number of timepoints.

\item[\option{--segment-overlap}~\parm{N}] Overlap data segments by 
a number ~\parm{N} of points.

\item[\option{--low-frequency-cutoff}~\parm{F}] The low frequency cutoff that 
was used to pre-filter the data using inspiral.c .

\item[\option{--cohsnr-threshold}~\parm{RHO}] Set the coherent signal-to-noise 
threshold to \parm{RHO}.

\item[\option{--output-path}~\parm{PATH}] Optional. Write output data to 
a given path \parm{PATH}. The default is to write the output to the same 
directory in which the executable is run.

\item[\option{--write-cohsnr}~\parm{FILE}] Optional. Write the coherent snr timeseries to a frame file \parm{FILE} (.gwf).

\item[\option{--maximize-over-chirp}] Optional. Do event clustering.

\item[\option{--H1-framefile}~\parm{FILE}] Specify the frame file containing
the C data for H1 if it is to be included in the network whose coherent snr
is being computed.

\item[\option{--H2-framefile}~\parm{FILE}] Specify the frame file containing
the C data for H2 if it is to be included in the network whose coherent snr
is being computed. 

\item[\option{--L-framefile}~\parm{FILE}] Specify the frame file containing
the C data for Livingston if it is to be included in the network whose 
coherent snr is being computed.

\item[\option{--V-framefile}~\parm{FILE}] Specify the frame file containing
the C data for Virgo if it is to be included in the network whose coherent snr
is being computed.

\item[\option{--G-framefile}~\parm{FILE}] Specify the frame file containing
the C data for Geo if it is to be included in the network whose coherent snr
is being computed.

\item[\option{--T-framefile}~\parm{FILE}] Specify the frame file containing
the C data for Tama if it is to be included in the network whose coherent snr
is being computed.

\end{entry}

\item[Example]
To run the program, type (for example):
\begin{verbatim}
lalapps_coherent_inspiral \
--verbose \
--low-frequency-cutoff 70.0 \
--bank-file L1-TMPLTBANK-751957200-512.xml \
--sample-rate 4096 \
--segment-length 524288 \
--cohsnr-threshold 10.0 \
--H1-framefile H1-INSPIRAL-751957200-512ZDATA.gwf \
--L-framefile L1-INSPIRAL-751957200-512ZDATA.gwf \
--channel-number 0 \
--write-cohsnr H1-L-COHSNR-751957200-512_0.gwf \
--output-path /home/sseader/LIGO/INSPIRAL/S3/playground \
\end{verbatim} 


\item[Author] Sukanta Bose and Shawn Seader 
\end{entry}
