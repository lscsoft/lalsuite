% $Id$

\section{Program \prog{lalapps\_stochastic\_bayes}}
\label{program:lalapps-stochastic-bayes}
\idx[Program]{lalapps\_stochastic\_bayes}

\begin{entry}
\item[Name]
\prog{lalapps\_stochastic\_bayes} --- python script to generate Condor DAGs
to run the stochastic bayesian pipeline.

\item[Synopsis]
\prog{lalapps\_stochastic\_bayes} \newline \hspace*{0.5in}
[\option{--help}] \newline \hspace*{0.5in}
[\option{--version}] \newline \hspace*{0.5in}
[\option{--user-tag}~\parm{TAG}] \newline \hspace*{0.5in}
[\option{--datafind}] \newline \hspace*{0.5in}
[\option{--stochastic}] \newline \hspace*{0.5in}
[\option{--priority}~\parm{PRIO}] \newline \hspace*{0.5in}
\option{--config-file}~\parm{FILE} \newline \hspace*{0.5in}
\option{--log-path}~\parm{PATH}

\item[Description]
\prog{lalapps\_stochastic\_bayes} builds a stochastic search DAG suitable
for running at the various LSC Data Grid sites. The configuration file
specifies the parameters needed to run the analysis jobs contained
within the pipeline. It is specified with the \option{--config-file}
option. A typical .ini file can be seen in\\
\texttt{\$LALAPPS\_PREFIX\$/share/lalapps/stochastic\_S3\_H1L1\_bayes.ini}.

The .ini file contains several sections. The \verb$[condor]$ section
contains the names of executables which will run the various stages of
the pipeline. The \verb$[pipeline]$ section gives the CVS details of the
pipeline. The \verb$[input]$ section specifies the segment list and the
minimum and maximum segment duration for the jobs. The \verb$[datafind]$
section specifies the frame types for the two input streams for passing
onto LSCdataFind. The \verb$[detectors]$ section specifies the detector
pair to cross correlate. The \verb$[calibration]$ section specifies the
location and name of the various calibration cache files. The
\verb$[frequency]$ section specifies the minimum and maximum frequencies
and the band over which the bayesian search runs. Finally
\verb$[stochastic]$ section specifies static options to pass to
\prog{lalapps\_stochastic}, i.e.~options that are not automatically
generated.

\item[Options]\leavevmode
\begin{entry}
\item[\option{--help}] Display usage information
\item[\option{--version}] Display version information
\item[\option{--user-tag}~\parm{TAG}] The tag for the job
\item[\option{--datafind}] Run LSCdataFind as part of the DAG to create
the cache files for each science segment
\item[\option{--stochastic}] Run \prog{lalapps\_stochastic} on the data
\item[\option{--priority}~\parm{PRIO}] Run jobs with condor priority \parm{PRIO}
\item[\option{--config-file}~\parm{FILE}] Use configuration file \parm{FILE}
\item[\option{--log-path}~\parm{PATH}] Directory to write condor log file
\end{entry}

\item[Configuration File Options]
The configuration file is a standard python configuration that can be
easily parsed using the standard ConfigParser module from python. The
configuration file consists of sections, led by a ``[section]'' header
and followed by ``name: value'' entries. The optional values can contain
format strings which refer to other values in the same section. Lines
beginning with ``\#'' or ``;'' are ignored and may be used to provide
comments.

The first section required is ``[condor]'', this specfies all the
parameters associated with condor.
\begin{entry}
\item[universe]
Specfies the condor universe under which to run, this should be set to
''standard'' if LALApps has been compiled with the
\texttt{--enable-condor}, otherwise it should be set to ``vanilla''.
\item[datafind]
Specifies the location of the \prog{LSCdataFind} script.
\item[stochastic]
Specifies the location of the \prog{lalapps\_stochastic} binary.
\end{entry}

The next section is ``[pipeline]'', this specifies version information
of the pipeline, currently only the version of configuration file is
specified.
\begin{entry}
\item[version]
A CVS \texttt{\$Id\$} keyword specifying the version of the
configuration file.
\end{entry}

The next section is ''[input]'' which specifies parameters regarding
the input data.
\begin{entry}
\item[segments]
Specifies the location of the segment list, outputed from
\prog{SegWizard} listing the Science Segments to analyse.
\item[min\_length]
Specifies the minimum length of Science Segment to analyse
\item[max\_length]
Specifies the maximum length at which the split the Science Segments
into.
\item[channel]
Currently the pipeline infrastructure requires that the ``[input]''
section contains the option ''channel'', this can be set to anything as
it not used within the \texttt{LSCdataFind} class.
\end{entry}

The next section is ``[datafind]'' which specifies parameters for
\prog{LSCdataFind}.
\begin{entry}
\item[type-one]
Specifies the frame type of the first data stream for \prog{LSCdataFind}
to find.
\item[type-two]
Specifies the frame type of the second data stream for
\prog{LSCdataFind} to find.
\end{entry}

The next section is ''[detectors]'' which specifies the detectors to use
for the search.
\begin{entry}
\item[detector-one]
Specifies the detector for the first data stream.
\item[detector-two]
Specifies the detector for the second data stream.
\end{entry}

The next section is ``[calibration]'' which specifies the location of
the calibration files.
\begin{entry}
\item[path]
Specifies the path to the calibration cache files.
\item[L1]
Specifies the cache file for the Livingston 4~km detector.
\item[H1]
Specifies the cache file for the Hanford 4~km detector.
\item[H2]
Specifies the cache file for the Hanford 2~km detector.
\end{entry}

The next section is ``[frequency]'' which specifies the frequency band
information for the bayesian search.
\begin{entry}
\item[f-min]
Specifies the minimum frequency of the total band to search.
\item[f-max]
Specifies the maximum frequency of the total band to search.
\item[f-band]
Specifies the sub band to search over.
\end{entry}

The next section is ``[stochastic]'', this is used to specify any static
options to pass onto \prog{lalapps\_stochastic}, i.e.~options that are
not automatically generated. See
Section~\ref{program:lalapps-stochastic} for the accepted options for
\prog{lalapps\_stochastic}. The options that are automatically generated
are, \texttt{--gps-start-time}, \texttt{--gps-end-time},
\texttt{--ifo-one}, \texttt{--ifo-two}, \texttt{--frame-cache-one},
\texttt{--frame-cache-two}, \texttt{--calibration-cache-one},\\
\texttt{--calibration-cache-two}, \texttt{--f-min}, and
\texttt{--f-max}.

\item[Example]
Generate a DAG to run a stochastic search on a pair of interferometers
specified in the configuration file. The generated DAG is then submitted
with \texttt{condor\_submit\_dag}

\begin{verbatim}
> lalapps_stochastic_bayes --log-path /home/ram/dag_logs \
    --datafind --stochastic --config-file stochastic_H1L1_bayes.ini
> condor_submit_dag stochastic_H1L1_bayes.dag
\end{verbatim}

\item[Author]
Adam Mercer
\end{entry}
