\section{Program \texttt{lalapps\_inspiral\_pipe}}
\label{program:inspiral-pipeline}
\idx[Program]{inspiral\_pipeline.py}

\begin{entry}
\item[Name]
\verb$lalapps_inspiral_pipe$ --- python script to generate Condor DAGs to
run the inspiral pipeline.

\item[Synopsis]
\begin{verbatim}
  -h, --help               display this message
  -v, --version            print version information and exit
  -u, --user-tag TAG       tag the job with TAG (overrides value in ini file)

  -d, --datafind           run LALdataFind to create frame cache files
  -t, --template-bank      run lalapps_tmpltbank to generate a template bank
  -i, --inspiral           run lalapps_inspiral on the first IFO
  -T, --triggered-bank     run lalapps_trigtotmplt to generate a triggered bank
  -I, --triggered-inspiral run lalapps_inspiral on the second IFO
  -C, --coincidence        run lalapps_inca on the triggers from both IFOs

  -j, --injections         add simulated inspirals from injection file

  -p, --playground-only    only create chunks that overlap with playground
  -P, --priority PRIO      run jobs with condor priority PRIO

  -f, --config-file FILE   use configuration file FILE
  -l, --log-path PATH      directory to write condor log file
\end{verbatim}

\item[Description] \verb$lalapps_inspiral_pipe$ generates a Condor DAG to run
the inspiral analysis pipeline. The configuration file should specify the
parameters needed to run the jobs and must be specified with the
\verb$--config-file$ option.  A file containing science segments to be
analyzed should be specified in the \verb$[input]$ section of the
configuration file with a line such as
\begin{verbatim}
segments = S2H1L1v03_selectedsegs.txt
\end{verbatim}
This should contain four whitespace separated columns:
\begin{verbatim}
  segment_id    gps_start_time  gps_end_time    duration
\end{verbatim}
that define the science segments to be used. Lines starting with an octothorpe
are ignored.

The analysis chunk size is determined from the number of data segments and
their length and overlap specified in config file. A chunk length is typically
this is 1024 seconds for S2.  The chunks start and stop times are computed
from the science segment times and used to build the DAG.

The once the DAG file has been created it should be submitted to the Condor
pool with the \verb$condor_submit_dag$ command.

\item[Options]\leavevmode
\begin{entry}
\item[\texttt{--help}] Display a brief usage summary.
\end{entry}

\item[Example]
Generate a DAG to run an inspiral search on the first IFO. The generated
DAG is then submitted with \texttt{condor\_submit\_dag}
\begin{verbatim}
lalapps_inspiral_pipe --log-path /people/duncan/dag_logs \
--datafind --template-bank --inspiral --playground-only \
--config-file l1_s2.ini

condor_submit_dag l1_s2.dag
\end{verbatim}

\item[Author] 
Duncan Brown
\end{entry}

