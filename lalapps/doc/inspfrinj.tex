\section{Program \texttt{lalapps\_inspfrinj}}
\label{program:lalapps-inspfrinj}
\idx[Program]{lalapps\_inspfrinj}

\begin{entry}
\item[Name]
\verb$lalapps_inspfrinj$ --- performs inspiral injections into frame data.

\item[Synopsis]
\begin{verbatim}
lalapps_inspfrinj [options]
 
  --help                    display this message
  --verbose                 print progress information
  --version                 print version information and exit
  --debug-level LEVEL       set the LAL debug level to LEVEL
  --user-tag STRING         set the process_params usertag to STRING
  --comment STRING          set the process table comment to STRING
 
  --gps-start-time SEC      GPS second of data start time
  --gps-start-time-ns NS    GPS nanosecond of data start time
  --gps-end-time SEC        GPS second of data end time
  --gps-end-time-ns NS      GPS nanosecond of data end time
 
  --frame-cache             obtain frame data from LAL frame cache FILE
  --calibration-cache FILE  obtain calibration from LAL frame cache FILE
  --calibrated-data TYPE    calibrated data of TYPE real_4 or real_8
  --num-resp-points N       num points to determine response function (4194304)
  --channel-name CHAN       read data from interferometer channel CHAN
 
  --injection-file FILE     inject simulated inspiral signals from FILE
  --inject-overhead         inject signals from overhead detector
  --inject-safety SEC       inject signals ending up to SEC after gps end time
 
  --write-raw-data          write out the raw frame files
  --write-inj-only          write out frames containing only injections
  --write-raw-plus-inj      write out frames containing raw data with inj
 
  --output-frame-length SEC write out data in frames of length SEC
  --output-file-name OUTPUT set output file names to OUTPUT-GPSTIME-LENGTH.gwf
                   if not set, default to IFO-INSPFRINJ-GPSTIME-LENGTH.gwf
 
  --ifo  IFO                specify the IFO (only if not reading frames)
  --sample-rate             data sample rate (only if not reading frames)
\end{verbatim}

\item[Description] 

\verb$lalapps_frinspinj$ performs inspiral injections into frame files.
This code is essentially a copy of those parts of
\verb$lalapps_inspiral$ responsible for reading in the data and
performing the injections.  The injection details are read in from the 
\verb$sim_inspiral$ table of a LIGO lightweight xml file.  

The code operates in two different ways.  If a \verb$frame-cache$ file
is specified, then the raw data is read in.  This data can either be
uncalibrated, in which case a calibration cache must be specified, or
calibrated, in which case the \verb$--calibrated-data$ flag must be set.
At present the code only works with real 4 data.  The injection data is
then calculated from the signals in the \verb$sim_inspiral$ table.  The
response function is either unity for strain data or is computed from
the calibration cache.  When working with uncalibrated data, the number
of points used to determine the response function will have an effect,
albeit minor, on the injection signals.  By default, the value
\verb$--num-resp-points$ is set to 4194304.  This matches the value
used by lalapps\_inspiral when working with 256 second segments and a
channel sampled at 16384 Hz.  The output frames can contain any of
the following: raw data, injection only and raw plus injection.  

If no input frame data is given, the code can produce strain data
injections for the relevant interferometer.  In this case, it is
necessary to specify the sample rate for the output data.  Since no
input data is read in, the output data can only contain the injection
only channel.

The output of \verb$lalapps_frinspinj$ is an xml file and one or several
frame files.  The xml file contains
\verb$process$, \verb$process_params$, \verb$search_summary$ and
\verb$sim_inspiral$ tables and is named (unless set on the command line):
\begin{center}
\texttt{IFO-INSPFRINJ\_USERTAG-GPSSTARTTIME-DURATION.xml}\\
\end{center}
where \texttt{GPSSTARTTIME} and \texttt{DURATION} are the start time and
duration passed to the code.  If \verb$--output-file-name$ is set to
\verb$OUTPUT$ on the command line, the xml will be named
\begin{center}
\texttt{OUTPUT\_USERTAG-GPSSTARTTIME-DURATION.xml}.\\
\end{center}
The \verb$sim_inspiral$ table contains a list of all injections which
were performed into the specified data.  The length of output frame
files is specified by \verb$output-frame-length$.  Unless otherwise set
on the command line, output frames are named: 
\begin{center}
\texttt{IFO-INSPFRINJ\_USERTAG-FRAMESTARTTIME-FRAMELENGTH.gwf}\\
\end{center}
The channels stored in the output frame are determined by which of
\verb$--write-raw-data$, \verb$--write-inj-only$ and
\verb$--write-raw-plus-inj$ are selected.  The channel names are
\verb$CHAN$, \verb$CHAN_INSP_INJ_ONLY$ and
\verb$CHAN_RAW_PLUS_INSP_INJ$.  In the case where no input data is
given, the single output channel name is \verb$IFO:STRAIN_INSP_INJ_ONLY$.

\item[Example]
\begin{verbatim}
lalapps_frinspinj --gps-start-time 732758030  --gps-end-time 732760078 \
  --frame-cache cache/L-732758022-732763688.cache --channel-name L1:LSC-AS_Q\	
  --calibration-cache cache_files/L1-CAL-V03-729273600-734367600.cache \
  --injection-file HL-INJECTIONS_4096-729273613-5094000.xml \
  --inject-safety 50 --output-frame-length 16 \
  --write-raw-data   --write-inj-only   --write-raw-plus-inj 
\end{verbatim}

\item[Author] 
Steve Fairhurst
\end{entry}

